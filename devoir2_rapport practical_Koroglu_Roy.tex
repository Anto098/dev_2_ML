\documentclass[11pt, french, english]{article}
\usepackage[margin=0.5in]{geometry}
\usepackage[T1]{fontenc}
\usepackage[utf8]{inputenc}
\usepackage{amsmath}
\usepackage{amssymb}
\usepackage{bbm}
\usepackage{mathtools}
\usepackage{ulem}
\usepackage{url}
\usepackage{graphicx}
\usepackage{lmodern}
\usepackage[english]{babel}

\allowdisplaybreaks

\makeatletter
\addto\extrasfrench{%
   \providecommand{\og}{\leavevmode\flqq~}%
   \providecommand{\fg}{\ifdim\lastskip>\z@\unskip\fi~\frqq}%
}
\makeatother

\usepackage{listings}
\usepackage{etoolbox}
\input{../common/commands}

%%%%%%%%% STUDENTS CHANGE THIS

\providetoggle{undergrad}
\settoggle{undergrad}{true}     %%% "true" if 3395 or "false" if 6390

\providetoggle{french}
\settoggle{french}{true}        %%% "true" if french or "false" if english

\providetoggle{final}            
\settoggle{final}{true}        %%% "true" for your final homework submission (removes instructions)

\newcommand{\question}[1]{\\ \textbf{Question.} #1 }
\usepackage[colorlinks=false]{}


\def\wkj{w_{k}^{j}}

\begin{document}
\setlength{\parskip}{0.3cm} \setlength{\parindent}{0cm}
\begin{center}
	\textbf{\proftitle{IFT 3395 Fondements de l'apprentissage machine \\ Prof. Guillaume Rabusseau}{IFT 6390 Fundamentals of Machine Learning \\ Ioannis Mitliagkas}}
	\par\end{center}{\large \par}

\begin{center}
	\textbf{\LARGE{\enfr{Homework É - Practical component - Answers}{Devoir 2 - Partie pratique - rapport}}} \\
	\par\end{center}{\LARGE \par}

Mathis Koroglu : 20223045\\
Antonin Roy : 20145595

\section*{\enfr{One-versus-all, L2 loss SVM}{Un-contre-tous, Perte L2 SVM}}

\begin{enumerate}
	\item \points{5 points}{5 points}\\
	      \begin{reponse}
		      \begin{align*}
			      \frac{\partial}{\partial w_k^j}\left(\frac{C}{2}\sum_{j'=1}^{m} \lVert \mathbf{w}^{j'} \rVert_2^2 \right)
			       & = \frac{C}{2} \frac{\partial}{\partial w_k^j}\left(\sum_{j^{\prime}=1}^{m}\sum_{k^{\prime}=1}^{p} {w_{k^{\prime}}^{j^{\prime}}}^{2}\right) \\
			       & = \frac{C}{2} \frac{\partial {w_{k}^{j}}^2}{\partial w_k^j}                                                                                \\
			       & = \frac{C}{2} 2 w_{k}^{j}                                                                                                                  \\
			       & = Cw_{k}^{j}
		      \end{align*}
	      \end{reponse}

	\item \points{10 points}{10 points}\\
	      \begin{reponse}

		      \begin{equation}
			      \label{eq:deriv_prod_scal}
			      \frac{\partial \langle \mathbf{w}^{j^{\prime}}, \mathbf{x}_i \rangle}{\partial\wkj}
			      = \begin{cases}
				      x_{i,k} & \text{ si } j^{\prime} = j \\
				      0       & \text{ sinon }
			      \end{cases}
		      \end{equation}

		      \begin{equation}
			      \label{eq:deriv_prod_scal_1}
			      \frac{\partial \langle \mathbf{w}^{j^{\prime}}, \mathbf{x}_i \rangle\mathbbm{1}_{\{y_i = j^{\prime}\}}}{\partial\wkj}
			      = \begin{cases}
				      x_{i,k}\mathbbm{1}_{\{y_i = j^{\prime}\}} & \text{ si } j^{\prime} = j \\
				      0                                         & \text{ sinon }
			      \end{cases}
		      \end{equation}

		      \begin{equation}
			      \label{eq:2-deriv_prod_scal_1}
			      \frac{\partial}{\partial\wkj}\left(2 - \langle \mathbf{w}^{j^{\prime}}, \mathbf{x}_i \rangle\mathbbm{1}_{\{y_i = j^{\prime}\}}\right)
			      = \begin{cases}
				      - x_{i,k}\mathbbm{1}_{\{y_i = j^{\prime}\}} & \text{ si } j^{\prime} = j \\
				      0                                           & \text{ sinon }
			      \end{cases}
		      \end{equation}

		      \begin{align}
			      \label{eq:deriv_max}
			      \begin{split}
				      \frac{\partial}{\partial\wkj}\max\left(0, 2 - \langle \mathbf{w}^{j^{\prime}}, \mathbf{x}_i \rangle\mathbbm{1}_{\{y_i = j^{\prime}\}}\right)
				      &= \underbrace{\frac{\partial\left(\max\left(0, 2 - \langle \mathbf{w}^{j^{\prime}}, \mathbf{x}_i \rangle\mathbbm{1}_{\{y_i = j^{\prime}\}}\right)\right)}{\partial\left(2 - \langle \mathbf{w}^{j^{\prime}}, \mathbf{x}_i \rangle\mathbbm{1}_{\{y_i = j^{\prime}\}}\right)}}_{\begin{cases}
						      1 & \text{ si } 2 >\langle \mathbf{w}^{j^{\prime}}, \mathbf{x}_i \rangle\mathbbm{1}_{\{y_i = j^{\prime}\}} \\
						      0 & \text{ sinon }
					      \end{cases}}\underbrace{\frac{\partial\left(2 - \langle \mathbf{w}^{j^{\prime}}, \mathbf{x}_i \rangle\mathbbm{1}_{\{y_i = j^{\prime}\}}\right)}{\partial\wkj}}_{\begin{cases}
						      - x_{i,k}\mathbbm{1}_{\{y_i = j^{\prime}\}} & \text{ si } j^{\prime} = j \\
						      0                                           & \text{ sinon }
					      \end{cases}}\\
				      &\overset{\ref{eq:2-deriv_prod_scal_1}}{=} \begin{cases}
					      - x_{i,k}\mathbbm{1}_{\{y_i = j^{\prime}\}} & \text{ si } j^{\prime} = j \text{ et } 2 >\langle \mathbf{w}^{j^{\prime}}, \mathbf{x}_i \rangle\mathbbm{1}_{\{y_i = j^{\prime}\}} \\
					      0                                           & \text{ sinon }
				      \end{cases}
			      \end{split}
		      \end{align}

		      \begin{align}
			      \label{eq:derivL}
			      \begin{split}
				      \frac{\partial\mathcal{L}\left(\mathbf{w}^{j^{\prime}}; (\mathbf{x}_i, y_i)\right)}{\partial\wkj}
				      &= \frac{\partial\left(\max\left(0, 2 - \langle \mathbf{w}^{j^{\prime}}, \mathbf{x}_i \rangle\mathbbm{1}_{\{y_i = j^{\prime}\}}\right)\right)^2}{\partial\wkj}\\
				      &= 2\max\left(0, 2 - \langle \mathbf{w}^{j^{\prime}}, \mathbf{x}_i \rangle\mathbbm{1}_{\{y_i = j^{\prime}\}}\right)\frac{\partial}{\partial\wkj}\max\left(0, 2 - \langle \mathbf{w}^{j^{\prime}}, \mathbf{x}_i \rangle\mathbbm{1}_{\{y_i = j^{\prime}\}}\right)\\
				      &\overset{\ref{eq:deriv_max}}{=} \begin{cases}
					      -2\underbrace{\max\left(0, 2 - \langle \mathbf{w}^{j^{\prime}}, \mathbf{x}_i \rangle\mathbbm{1}_{\{y_i = j^{\prime}\}}\right)}_{0 \text{ si } 2 \leq \langle \mathbf{w}^{j^{\prime}}, \mathbf{x}_i \rangle\mathbbm{1}_{\{y_i = j^{\prime}\}}}x_{i,k}\mathbbm{1}_{\{y_i = j^{\prime}\}} & \text{ si } j^{\prime} = j \text{ et } 2 >\langle \mathbf{w}^{j^{\prime}}, \mathbf{x}_i \rangle\mathbbm{1}_{\{y_i = j^{\prime}\}} \\
					      2\max\left(0, 2 - \langle \mathbf{w}^{j^{\prime}}, \mathbf{x}_i \rangle\mathbbm{1}_{\{y_i = j^{\prime}\}}\right) \cdot 0 = 0                                                                                                                                                           & \text{ sinon }
				      \end{cases}\\
				      &= \begin{cases}
					      -2\max\left(0, 2 - \langle \mathbf{w}^{j^{\prime}}, \mathbf{x}_i \rangle\mathbbm{1}_{\{y_i = j^{\prime}\}}\right)x_{i,k}\mathbbm{1}_{\{y_i = j^{\prime}\}} & \text{ si } j^{\prime} = j \\
					      0                                                                                                                                                          & \text{ sinon }
				      \end{cases}
			      \end{split}
		      \end{align}

		      \begin{align*}
			      \frac{\partial}{\partial\wkj} \left(\frac{1}{n}\sum_{(\mathbf{x}_i, y_i) \in S} \sum_{j'=1}^{m} \mathcal{L}\left(\mathbf{w}^{j'}; (\mathbf{x}_i, y_i))\right)\right)
			       & = \frac{\partial}{\partial w_{k}^{j}} \left(\frac{1}{n}\sum_{(\mathbf{x}_i, y_i) \in S} \left(\mathcal{L}\left(\mathbf{w}^{j}; (\mathbf{x}_i, y_i)\right) +\underbrace{\sum_{\substack{j^{\prime}=1                      \\ j^{\prime} \neq j}}^{m} \mathcal{L}\left(\mathbf{w}^{j^{\prime}}; (\mathbf{x}_i, y_i)\right)}_{0 \text{ d'après } \ref{eq:derivL}}\right)\right)\\
			       & = \frac{\partial}{\partial w_{k}^{j}} \left(\frac{1}{n}\sum_{(\mathbf{x}_i, y_i) \in S} \mathcal{L}\left(\mathbf{w}^{j}; (\mathbf{x}_i, y_i)\right)\right)                                                               \\
			       & = \frac{1}{n}\sum_{(\mathbf{x}_i, y_i) \in S}\frac{\partial\mathcal{L}\left(\mathbf{w}^{j}; (\mathbf{x}_i, y_i)\right)}{\partial\wkj}                                                                                    \\
			       & \overset{\ref{eq:derivL}}{=} -\frac{2}{n}\sum_{(\mathbf{x}_i, y_i) \in S}\max\left(0, 2 - \langle \mathbf{w}^{j}, \mathbf{x}_i \rangle\mathbbm{1}_{\{y_i = j^{\prime}\}}\right)x_{i,k}\mathbbm{1}_{\{y_i = j^{\prime}\}}
		      \end{align*}

	      \end{reponse}
	      \newpage

	\item[4.] Graphiques \points{10 points}{10 points}

		\begin{reponse}
			\begin{figure}[H]
				\begin{center}
					%% Creator: Matplotlib, PGF backend
%%
%% To include the figure in your LaTeX document, write
%%   \input{<filename>.pgf}
%%
%% Make sure the required packages are loaded in your preamble
%%   \usepackage{pgf}
%%
%% Figures using additional raster images can only be included by \input if
%% they are in the same directory as the main LaTeX file. For loading figures
%% from other directories you can use the `import` package
%%   \usepackage{import}
%%
%% and then include the figures with
%%   \import{<path to file>}{<filename>.pgf}
%%
%% Matplotlib used the following preamble
%%
\begingroup%
\makeatletter%
\begin{pgfpicture}%
\pgfpathrectangle{\pgfpointorigin}{\pgfqpoint{8.000000in}{8.000000in}}%
\pgfusepath{use as bounding box, clip}%
\begin{pgfscope}%
\pgfsetbuttcap%
\pgfsetmiterjoin%
\definecolor{currentfill}{rgb}{1.000000,1.000000,1.000000}%
\pgfsetfillcolor{currentfill}%
\pgfsetlinewidth{0.000000pt}%
\definecolor{currentstroke}{rgb}{1.000000,1.000000,1.000000}%
\pgfsetstrokecolor{currentstroke}%
\pgfsetdash{}{0pt}%
\pgfpathmoveto{\pgfqpoint{0.000000in}{0.000000in}}%
\pgfpathlineto{\pgfqpoint{8.000000in}{0.000000in}}%
\pgfpathlineto{\pgfqpoint{8.000000in}{8.000000in}}%
\pgfpathlineto{\pgfqpoint{0.000000in}{8.000000in}}%
\pgfpathclose%
\pgfusepath{fill}%
\end{pgfscope}%
\begin{pgfscope}%
\pgfsetbuttcap%
\pgfsetmiterjoin%
\definecolor{currentfill}{rgb}{1.000000,1.000000,1.000000}%
\pgfsetfillcolor{currentfill}%
\pgfsetlinewidth{0.000000pt}%
\definecolor{currentstroke}{rgb}{0.000000,0.000000,0.000000}%
\pgfsetstrokecolor{currentstroke}%
\pgfsetstrokeopacity{0.000000}%
\pgfsetdash{}{0pt}%
\pgfpathmoveto{\pgfqpoint{1.000000in}{3.960000in}}%
\pgfpathlineto{\pgfqpoint{4.100000in}{3.960000in}}%
\pgfpathlineto{\pgfqpoint{4.100000in}{7.040000in}}%
\pgfpathlineto{\pgfqpoint{1.000000in}{7.040000in}}%
\pgfpathclose%
\pgfusepath{fill}%
\end{pgfscope}%
\begin{pgfscope}%
\pgfpathrectangle{\pgfqpoint{1.000000in}{3.960000in}}{\pgfqpoint{3.100000in}{3.080000in}}%
\pgfusepath{clip}%
\pgfsetbuttcap%
\pgfsetroundjoin%
\pgfsetlinewidth{0.501875pt}%
\definecolor{currentstroke}{rgb}{0.000000,0.000000,0.000000}%
\pgfsetstrokecolor{currentstroke}%
\pgfsetstrokeopacity{0.450000}%
\pgfsetdash{{2.500000pt}{5.000000pt}}{0.000000pt}%
\pgfpathmoveto{\pgfqpoint{1.300000in}{3.960000in}}%
\pgfpathlineto{\pgfqpoint{1.300000in}{7.040000in}}%
\pgfusepath{stroke}%
\end{pgfscope}%
\begin{pgfscope}%
\pgfsetbuttcap%
\pgfsetroundjoin%
\definecolor{currentfill}{rgb}{0.000000,0.000000,0.000000}%
\pgfsetfillcolor{currentfill}%
\pgfsetlinewidth{0.803000pt}%
\definecolor{currentstroke}{rgb}{0.000000,0.000000,0.000000}%
\pgfsetstrokecolor{currentstroke}%
\pgfsetdash{}{0pt}%
\pgfsys@defobject{currentmarker}{\pgfqpoint{0.000000in}{0.000000in}}{\pgfqpoint{0.000000in}{0.048611in}}{%
\pgfpathmoveto{\pgfqpoint{0.000000in}{0.000000in}}%
\pgfpathlineto{\pgfqpoint{0.000000in}{0.048611in}}%
\pgfusepath{stroke,fill}%
}%
\begin{pgfscope}%
\pgfsys@transformshift{1.300000in}{3.960000in}%
\pgfsys@useobject{currentmarker}{}%
\end{pgfscope}%
\end{pgfscope}%
\begin{pgfscope}%
\pgfpathrectangle{\pgfqpoint{1.000000in}{3.960000in}}{\pgfqpoint{3.100000in}{3.080000in}}%
\pgfusepath{clip}%
\pgfsetbuttcap%
\pgfsetroundjoin%
\pgfsetlinewidth{0.501875pt}%
\definecolor{currentstroke}{rgb}{0.000000,0.000000,0.000000}%
\pgfsetstrokecolor{currentstroke}%
\pgfsetstrokeopacity{0.450000}%
\pgfsetdash{{2.500000pt}{5.000000pt}}{0.000000pt}%
\pgfpathmoveto{\pgfqpoint{1.928141in}{3.960000in}}%
\pgfpathlineto{\pgfqpoint{1.928141in}{7.040000in}}%
\pgfusepath{stroke}%
\end{pgfscope}%
\begin{pgfscope}%
\pgfsetbuttcap%
\pgfsetroundjoin%
\definecolor{currentfill}{rgb}{0.000000,0.000000,0.000000}%
\pgfsetfillcolor{currentfill}%
\pgfsetlinewidth{0.803000pt}%
\definecolor{currentstroke}{rgb}{0.000000,0.000000,0.000000}%
\pgfsetstrokecolor{currentstroke}%
\pgfsetdash{}{0pt}%
\pgfsys@defobject{currentmarker}{\pgfqpoint{0.000000in}{0.000000in}}{\pgfqpoint{0.000000in}{0.048611in}}{%
\pgfpathmoveto{\pgfqpoint{0.000000in}{0.000000in}}%
\pgfpathlineto{\pgfqpoint{0.000000in}{0.048611in}}%
\pgfusepath{stroke,fill}%
}%
\begin{pgfscope}%
\pgfsys@transformshift{1.928141in}{3.960000in}%
\pgfsys@useobject{currentmarker}{}%
\end{pgfscope}%
\end{pgfscope}%
\begin{pgfscope}%
\pgfpathrectangle{\pgfqpoint{1.000000in}{3.960000in}}{\pgfqpoint{3.100000in}{3.080000in}}%
\pgfusepath{clip}%
\pgfsetbuttcap%
\pgfsetroundjoin%
\pgfsetlinewidth{0.501875pt}%
\definecolor{currentstroke}{rgb}{0.000000,0.000000,0.000000}%
\pgfsetstrokecolor{currentstroke}%
\pgfsetstrokeopacity{0.450000}%
\pgfsetdash{{2.500000pt}{5.000000pt}}{0.000000pt}%
\pgfpathmoveto{\pgfqpoint{2.556281in}{3.960000in}}%
\pgfpathlineto{\pgfqpoint{2.556281in}{7.040000in}}%
\pgfusepath{stroke}%
\end{pgfscope}%
\begin{pgfscope}%
\pgfsetbuttcap%
\pgfsetroundjoin%
\definecolor{currentfill}{rgb}{0.000000,0.000000,0.000000}%
\pgfsetfillcolor{currentfill}%
\pgfsetlinewidth{0.803000pt}%
\definecolor{currentstroke}{rgb}{0.000000,0.000000,0.000000}%
\pgfsetstrokecolor{currentstroke}%
\pgfsetdash{}{0pt}%
\pgfsys@defobject{currentmarker}{\pgfqpoint{0.000000in}{0.000000in}}{\pgfqpoint{0.000000in}{0.048611in}}{%
\pgfpathmoveto{\pgfqpoint{0.000000in}{0.000000in}}%
\pgfpathlineto{\pgfqpoint{0.000000in}{0.048611in}}%
\pgfusepath{stroke,fill}%
}%
\begin{pgfscope}%
\pgfsys@transformshift{2.556281in}{3.960000in}%
\pgfsys@useobject{currentmarker}{}%
\end{pgfscope}%
\end{pgfscope}%
\begin{pgfscope}%
\pgfpathrectangle{\pgfqpoint{1.000000in}{3.960000in}}{\pgfqpoint{3.100000in}{3.080000in}}%
\pgfusepath{clip}%
\pgfsetbuttcap%
\pgfsetroundjoin%
\pgfsetlinewidth{0.501875pt}%
\definecolor{currentstroke}{rgb}{0.000000,0.000000,0.000000}%
\pgfsetstrokecolor{currentstroke}%
\pgfsetstrokeopacity{0.450000}%
\pgfsetdash{{2.500000pt}{5.000000pt}}{0.000000pt}%
\pgfpathmoveto{\pgfqpoint{3.184422in}{3.960000in}}%
\pgfpathlineto{\pgfqpoint{3.184422in}{7.040000in}}%
\pgfusepath{stroke}%
\end{pgfscope}%
\begin{pgfscope}%
\pgfsetbuttcap%
\pgfsetroundjoin%
\definecolor{currentfill}{rgb}{0.000000,0.000000,0.000000}%
\pgfsetfillcolor{currentfill}%
\pgfsetlinewidth{0.803000pt}%
\definecolor{currentstroke}{rgb}{0.000000,0.000000,0.000000}%
\pgfsetstrokecolor{currentstroke}%
\pgfsetdash{}{0pt}%
\pgfsys@defobject{currentmarker}{\pgfqpoint{0.000000in}{0.000000in}}{\pgfqpoint{0.000000in}{0.048611in}}{%
\pgfpathmoveto{\pgfqpoint{0.000000in}{0.000000in}}%
\pgfpathlineto{\pgfqpoint{0.000000in}{0.048611in}}%
\pgfusepath{stroke,fill}%
}%
\begin{pgfscope}%
\pgfsys@transformshift{3.184422in}{3.960000in}%
\pgfsys@useobject{currentmarker}{}%
\end{pgfscope}%
\end{pgfscope}%
\begin{pgfscope}%
\pgfpathrectangle{\pgfqpoint{1.000000in}{3.960000in}}{\pgfqpoint{3.100000in}{3.080000in}}%
\pgfusepath{clip}%
\pgfsetbuttcap%
\pgfsetroundjoin%
\pgfsetlinewidth{0.501875pt}%
\definecolor{currentstroke}{rgb}{0.000000,0.000000,0.000000}%
\pgfsetstrokecolor{currentstroke}%
\pgfsetstrokeopacity{0.450000}%
\pgfsetdash{{2.500000pt}{5.000000pt}}{0.000000pt}%
\pgfpathmoveto{\pgfqpoint{3.812563in}{3.960000in}}%
\pgfpathlineto{\pgfqpoint{3.812563in}{7.040000in}}%
\pgfusepath{stroke}%
\end{pgfscope}%
\begin{pgfscope}%
\pgfsetbuttcap%
\pgfsetroundjoin%
\definecolor{currentfill}{rgb}{0.000000,0.000000,0.000000}%
\pgfsetfillcolor{currentfill}%
\pgfsetlinewidth{0.803000pt}%
\definecolor{currentstroke}{rgb}{0.000000,0.000000,0.000000}%
\pgfsetstrokecolor{currentstroke}%
\pgfsetdash{}{0pt}%
\pgfsys@defobject{currentmarker}{\pgfqpoint{0.000000in}{0.000000in}}{\pgfqpoint{0.000000in}{0.048611in}}{%
\pgfpathmoveto{\pgfqpoint{0.000000in}{0.000000in}}%
\pgfpathlineto{\pgfqpoint{0.000000in}{0.048611in}}%
\pgfusepath{stroke,fill}%
}%
\begin{pgfscope}%
\pgfsys@transformshift{3.812563in}{3.960000in}%
\pgfsys@useobject{currentmarker}{}%
\end{pgfscope}%
\end{pgfscope}%
\begin{pgfscope}%
\pgfpathrectangle{\pgfqpoint{1.000000in}{3.960000in}}{\pgfqpoint{3.100000in}{3.080000in}}%
\pgfusepath{clip}%
\pgfsetbuttcap%
\pgfsetroundjoin%
\pgfsetlinewidth{0.501875pt}%
\definecolor{currentstroke}{rgb}{0.000000,0.000000,0.000000}%
\pgfsetstrokecolor{currentstroke}%
\pgfsetstrokeopacity{0.450000}%
\pgfsetdash{{2.500000pt}{5.000000pt}}{0.000000pt}%
\pgfpathmoveto{\pgfqpoint{1.000000in}{3.968029in}}%
\pgfpathlineto{\pgfqpoint{4.100000in}{3.968029in}}%
\pgfusepath{stroke}%
\end{pgfscope}%
\begin{pgfscope}%
\pgfsetbuttcap%
\pgfsetroundjoin%
\definecolor{currentfill}{rgb}{0.000000,0.000000,0.000000}%
\pgfsetfillcolor{currentfill}%
\pgfsetlinewidth{0.803000pt}%
\definecolor{currentstroke}{rgb}{0.000000,0.000000,0.000000}%
\pgfsetstrokecolor{currentstroke}%
\pgfsetdash{}{0pt}%
\pgfsys@defobject{currentmarker}{\pgfqpoint{-0.048611in}{0.000000in}}{\pgfqpoint{-0.000000in}{0.000000in}}{%
\pgfpathmoveto{\pgfqpoint{-0.000000in}{0.000000in}}%
\pgfpathlineto{\pgfqpoint{-0.048611in}{0.000000in}}%
\pgfusepath{stroke,fill}%
}%
\begin{pgfscope}%
\pgfsys@transformshift{1.000000in}{3.968029in}%
\pgfsys@useobject{currentmarker}{}%
\end{pgfscope}%
\end{pgfscope}%
\begin{pgfscope}%
\definecolor{textcolor}{rgb}{0.000000,0.000000,0.000000}%
\pgfsetstrokecolor{textcolor}%
\pgfsetfillcolor{textcolor}%
\pgftext[x=0.655863in, y=3.919804in, left, base]{\color{textcolor}\rmfamily\fontsize{10.400000}{12.480000}\selectfont \(\displaystyle {11.5}\)}%
\end{pgfscope}%
\begin{pgfscope}%
\pgfpathrectangle{\pgfqpoint{1.000000in}{3.960000in}}{\pgfqpoint{3.100000in}{3.080000in}}%
\pgfusepath{clip}%
\pgfsetbuttcap%
\pgfsetroundjoin%
\pgfsetlinewidth{0.501875pt}%
\definecolor{currentstroke}{rgb}{0.000000,0.000000,0.000000}%
\pgfsetstrokecolor{currentstroke}%
\pgfsetstrokeopacity{0.450000}%
\pgfsetdash{{2.500000pt}{5.000000pt}}{0.000000pt}%
\pgfpathmoveto{\pgfqpoint{1.000000in}{4.329085in}}%
\pgfpathlineto{\pgfqpoint{4.100000in}{4.329085in}}%
\pgfusepath{stroke}%
\end{pgfscope}%
\begin{pgfscope}%
\pgfsetbuttcap%
\pgfsetroundjoin%
\definecolor{currentfill}{rgb}{0.000000,0.000000,0.000000}%
\pgfsetfillcolor{currentfill}%
\pgfsetlinewidth{0.803000pt}%
\definecolor{currentstroke}{rgb}{0.000000,0.000000,0.000000}%
\pgfsetstrokecolor{currentstroke}%
\pgfsetdash{}{0pt}%
\pgfsys@defobject{currentmarker}{\pgfqpoint{-0.048611in}{0.000000in}}{\pgfqpoint{-0.000000in}{0.000000in}}{%
\pgfpathmoveto{\pgfqpoint{-0.000000in}{0.000000in}}%
\pgfpathlineto{\pgfqpoint{-0.048611in}{0.000000in}}%
\pgfusepath{stroke,fill}%
}%
\begin{pgfscope}%
\pgfsys@transformshift{1.000000in}{4.329085in}%
\pgfsys@useobject{currentmarker}{}%
\end{pgfscope}%
\end{pgfscope}%
\begin{pgfscope}%
\definecolor{textcolor}{rgb}{0.000000,0.000000,0.000000}%
\pgfsetstrokecolor{textcolor}%
\pgfsetfillcolor{textcolor}%
\pgftext[x=0.655863in, y=4.280859in, left, base]{\color{textcolor}\rmfamily\fontsize{10.400000}{12.480000}\selectfont \(\displaystyle {12.0}\)}%
\end{pgfscope}%
\begin{pgfscope}%
\pgfpathrectangle{\pgfqpoint{1.000000in}{3.960000in}}{\pgfqpoint{3.100000in}{3.080000in}}%
\pgfusepath{clip}%
\pgfsetbuttcap%
\pgfsetroundjoin%
\pgfsetlinewidth{0.501875pt}%
\definecolor{currentstroke}{rgb}{0.000000,0.000000,0.000000}%
\pgfsetstrokecolor{currentstroke}%
\pgfsetstrokeopacity{0.450000}%
\pgfsetdash{{2.500000pt}{5.000000pt}}{0.000000pt}%
\pgfpathmoveto{\pgfqpoint{1.000000in}{4.690141in}}%
\pgfpathlineto{\pgfqpoint{4.100000in}{4.690141in}}%
\pgfusepath{stroke}%
\end{pgfscope}%
\begin{pgfscope}%
\pgfsetbuttcap%
\pgfsetroundjoin%
\definecolor{currentfill}{rgb}{0.000000,0.000000,0.000000}%
\pgfsetfillcolor{currentfill}%
\pgfsetlinewidth{0.803000pt}%
\definecolor{currentstroke}{rgb}{0.000000,0.000000,0.000000}%
\pgfsetstrokecolor{currentstroke}%
\pgfsetdash{}{0pt}%
\pgfsys@defobject{currentmarker}{\pgfqpoint{-0.048611in}{0.000000in}}{\pgfqpoint{-0.000000in}{0.000000in}}{%
\pgfpathmoveto{\pgfqpoint{-0.000000in}{0.000000in}}%
\pgfpathlineto{\pgfqpoint{-0.048611in}{0.000000in}}%
\pgfusepath{stroke,fill}%
}%
\begin{pgfscope}%
\pgfsys@transformshift{1.000000in}{4.690141in}%
\pgfsys@useobject{currentmarker}{}%
\end{pgfscope}%
\end{pgfscope}%
\begin{pgfscope}%
\definecolor{textcolor}{rgb}{0.000000,0.000000,0.000000}%
\pgfsetstrokecolor{textcolor}%
\pgfsetfillcolor{textcolor}%
\pgftext[x=0.655863in, y=4.641915in, left, base]{\color{textcolor}\rmfamily\fontsize{10.400000}{12.480000}\selectfont \(\displaystyle {12.5}\)}%
\end{pgfscope}%
\begin{pgfscope}%
\pgfpathrectangle{\pgfqpoint{1.000000in}{3.960000in}}{\pgfqpoint{3.100000in}{3.080000in}}%
\pgfusepath{clip}%
\pgfsetbuttcap%
\pgfsetroundjoin%
\pgfsetlinewidth{0.501875pt}%
\definecolor{currentstroke}{rgb}{0.000000,0.000000,0.000000}%
\pgfsetstrokecolor{currentstroke}%
\pgfsetstrokeopacity{0.450000}%
\pgfsetdash{{2.500000pt}{5.000000pt}}{0.000000pt}%
\pgfpathmoveto{\pgfqpoint{1.000000in}{5.051196in}}%
\pgfpathlineto{\pgfqpoint{4.100000in}{5.051196in}}%
\pgfusepath{stroke}%
\end{pgfscope}%
\begin{pgfscope}%
\pgfsetbuttcap%
\pgfsetroundjoin%
\definecolor{currentfill}{rgb}{0.000000,0.000000,0.000000}%
\pgfsetfillcolor{currentfill}%
\pgfsetlinewidth{0.803000pt}%
\definecolor{currentstroke}{rgb}{0.000000,0.000000,0.000000}%
\pgfsetstrokecolor{currentstroke}%
\pgfsetdash{}{0pt}%
\pgfsys@defobject{currentmarker}{\pgfqpoint{-0.048611in}{0.000000in}}{\pgfqpoint{-0.000000in}{0.000000in}}{%
\pgfpathmoveto{\pgfqpoint{-0.000000in}{0.000000in}}%
\pgfpathlineto{\pgfqpoint{-0.048611in}{0.000000in}}%
\pgfusepath{stroke,fill}%
}%
\begin{pgfscope}%
\pgfsys@transformshift{1.000000in}{5.051196in}%
\pgfsys@useobject{currentmarker}{}%
\end{pgfscope}%
\end{pgfscope}%
\begin{pgfscope}%
\definecolor{textcolor}{rgb}{0.000000,0.000000,0.000000}%
\pgfsetstrokecolor{textcolor}%
\pgfsetfillcolor{textcolor}%
\pgftext[x=0.655863in, y=5.002971in, left, base]{\color{textcolor}\rmfamily\fontsize{10.400000}{12.480000}\selectfont \(\displaystyle {13.0}\)}%
\end{pgfscope}%
\begin{pgfscope}%
\pgfpathrectangle{\pgfqpoint{1.000000in}{3.960000in}}{\pgfqpoint{3.100000in}{3.080000in}}%
\pgfusepath{clip}%
\pgfsetbuttcap%
\pgfsetroundjoin%
\pgfsetlinewidth{0.501875pt}%
\definecolor{currentstroke}{rgb}{0.000000,0.000000,0.000000}%
\pgfsetstrokecolor{currentstroke}%
\pgfsetstrokeopacity{0.450000}%
\pgfsetdash{{2.500000pt}{5.000000pt}}{0.000000pt}%
\pgfpathmoveto{\pgfqpoint{1.000000in}{5.412252in}}%
\pgfpathlineto{\pgfqpoint{4.100000in}{5.412252in}}%
\pgfusepath{stroke}%
\end{pgfscope}%
\begin{pgfscope}%
\pgfsetbuttcap%
\pgfsetroundjoin%
\definecolor{currentfill}{rgb}{0.000000,0.000000,0.000000}%
\pgfsetfillcolor{currentfill}%
\pgfsetlinewidth{0.803000pt}%
\definecolor{currentstroke}{rgb}{0.000000,0.000000,0.000000}%
\pgfsetstrokecolor{currentstroke}%
\pgfsetdash{}{0pt}%
\pgfsys@defobject{currentmarker}{\pgfqpoint{-0.048611in}{0.000000in}}{\pgfqpoint{-0.000000in}{0.000000in}}{%
\pgfpathmoveto{\pgfqpoint{-0.000000in}{0.000000in}}%
\pgfpathlineto{\pgfqpoint{-0.048611in}{0.000000in}}%
\pgfusepath{stroke,fill}%
}%
\begin{pgfscope}%
\pgfsys@transformshift{1.000000in}{5.412252in}%
\pgfsys@useobject{currentmarker}{}%
\end{pgfscope}%
\end{pgfscope}%
\begin{pgfscope}%
\definecolor{textcolor}{rgb}{0.000000,0.000000,0.000000}%
\pgfsetstrokecolor{textcolor}%
\pgfsetfillcolor{textcolor}%
\pgftext[x=0.655863in, y=5.364027in, left, base]{\color{textcolor}\rmfamily\fontsize{10.400000}{12.480000}\selectfont \(\displaystyle {13.5}\)}%
\end{pgfscope}%
\begin{pgfscope}%
\pgfpathrectangle{\pgfqpoint{1.000000in}{3.960000in}}{\pgfqpoint{3.100000in}{3.080000in}}%
\pgfusepath{clip}%
\pgfsetbuttcap%
\pgfsetroundjoin%
\pgfsetlinewidth{0.501875pt}%
\definecolor{currentstroke}{rgb}{0.000000,0.000000,0.000000}%
\pgfsetstrokecolor{currentstroke}%
\pgfsetstrokeopacity{0.450000}%
\pgfsetdash{{2.500000pt}{5.000000pt}}{0.000000pt}%
\pgfpathmoveto{\pgfqpoint{1.000000in}{5.773308in}}%
\pgfpathlineto{\pgfqpoint{4.100000in}{5.773308in}}%
\pgfusepath{stroke}%
\end{pgfscope}%
\begin{pgfscope}%
\pgfsetbuttcap%
\pgfsetroundjoin%
\definecolor{currentfill}{rgb}{0.000000,0.000000,0.000000}%
\pgfsetfillcolor{currentfill}%
\pgfsetlinewidth{0.803000pt}%
\definecolor{currentstroke}{rgb}{0.000000,0.000000,0.000000}%
\pgfsetstrokecolor{currentstroke}%
\pgfsetdash{}{0pt}%
\pgfsys@defobject{currentmarker}{\pgfqpoint{-0.048611in}{0.000000in}}{\pgfqpoint{-0.000000in}{0.000000in}}{%
\pgfpathmoveto{\pgfqpoint{-0.000000in}{0.000000in}}%
\pgfpathlineto{\pgfqpoint{-0.048611in}{0.000000in}}%
\pgfusepath{stroke,fill}%
}%
\begin{pgfscope}%
\pgfsys@transformshift{1.000000in}{5.773308in}%
\pgfsys@useobject{currentmarker}{}%
\end{pgfscope}%
\end{pgfscope}%
\begin{pgfscope}%
\definecolor{textcolor}{rgb}{0.000000,0.000000,0.000000}%
\pgfsetstrokecolor{textcolor}%
\pgfsetfillcolor{textcolor}%
\pgftext[x=0.655863in, y=5.725083in, left, base]{\color{textcolor}\rmfamily\fontsize{10.400000}{12.480000}\selectfont \(\displaystyle {14.0}\)}%
\end{pgfscope}%
\begin{pgfscope}%
\pgfpathrectangle{\pgfqpoint{1.000000in}{3.960000in}}{\pgfqpoint{3.100000in}{3.080000in}}%
\pgfusepath{clip}%
\pgfsetbuttcap%
\pgfsetroundjoin%
\pgfsetlinewidth{0.501875pt}%
\definecolor{currentstroke}{rgb}{0.000000,0.000000,0.000000}%
\pgfsetstrokecolor{currentstroke}%
\pgfsetstrokeopacity{0.450000}%
\pgfsetdash{{2.500000pt}{5.000000pt}}{0.000000pt}%
\pgfpathmoveto{\pgfqpoint{1.000000in}{6.134364in}}%
\pgfpathlineto{\pgfqpoint{4.100000in}{6.134364in}}%
\pgfusepath{stroke}%
\end{pgfscope}%
\begin{pgfscope}%
\pgfsetbuttcap%
\pgfsetroundjoin%
\definecolor{currentfill}{rgb}{0.000000,0.000000,0.000000}%
\pgfsetfillcolor{currentfill}%
\pgfsetlinewidth{0.803000pt}%
\definecolor{currentstroke}{rgb}{0.000000,0.000000,0.000000}%
\pgfsetstrokecolor{currentstroke}%
\pgfsetdash{}{0pt}%
\pgfsys@defobject{currentmarker}{\pgfqpoint{-0.048611in}{0.000000in}}{\pgfqpoint{-0.000000in}{0.000000in}}{%
\pgfpathmoveto{\pgfqpoint{-0.000000in}{0.000000in}}%
\pgfpathlineto{\pgfqpoint{-0.048611in}{0.000000in}}%
\pgfusepath{stroke,fill}%
}%
\begin{pgfscope}%
\pgfsys@transformshift{1.000000in}{6.134364in}%
\pgfsys@useobject{currentmarker}{}%
\end{pgfscope}%
\end{pgfscope}%
\begin{pgfscope}%
\definecolor{textcolor}{rgb}{0.000000,0.000000,0.000000}%
\pgfsetstrokecolor{textcolor}%
\pgfsetfillcolor{textcolor}%
\pgftext[x=0.655863in, y=6.086139in, left, base]{\color{textcolor}\rmfamily\fontsize{10.400000}{12.480000}\selectfont \(\displaystyle {14.5}\)}%
\end{pgfscope}%
\begin{pgfscope}%
\pgfpathrectangle{\pgfqpoint{1.000000in}{3.960000in}}{\pgfqpoint{3.100000in}{3.080000in}}%
\pgfusepath{clip}%
\pgfsetbuttcap%
\pgfsetroundjoin%
\pgfsetlinewidth{0.501875pt}%
\definecolor{currentstroke}{rgb}{0.000000,0.000000,0.000000}%
\pgfsetstrokecolor{currentstroke}%
\pgfsetstrokeopacity{0.450000}%
\pgfsetdash{{2.500000pt}{5.000000pt}}{0.000000pt}%
\pgfpathmoveto{\pgfqpoint{1.000000in}{6.495420in}}%
\pgfpathlineto{\pgfqpoint{4.100000in}{6.495420in}}%
\pgfusepath{stroke}%
\end{pgfscope}%
\begin{pgfscope}%
\pgfsetbuttcap%
\pgfsetroundjoin%
\definecolor{currentfill}{rgb}{0.000000,0.000000,0.000000}%
\pgfsetfillcolor{currentfill}%
\pgfsetlinewidth{0.803000pt}%
\definecolor{currentstroke}{rgb}{0.000000,0.000000,0.000000}%
\pgfsetstrokecolor{currentstroke}%
\pgfsetdash{}{0pt}%
\pgfsys@defobject{currentmarker}{\pgfqpoint{-0.048611in}{0.000000in}}{\pgfqpoint{-0.000000in}{0.000000in}}{%
\pgfpathmoveto{\pgfqpoint{-0.000000in}{0.000000in}}%
\pgfpathlineto{\pgfqpoint{-0.048611in}{0.000000in}}%
\pgfusepath{stroke,fill}%
}%
\begin{pgfscope}%
\pgfsys@transformshift{1.000000in}{6.495420in}%
\pgfsys@useobject{currentmarker}{}%
\end{pgfscope}%
\end{pgfscope}%
\begin{pgfscope}%
\definecolor{textcolor}{rgb}{0.000000,0.000000,0.000000}%
\pgfsetstrokecolor{textcolor}%
\pgfsetfillcolor{textcolor}%
\pgftext[x=0.655863in, y=6.447194in, left, base]{\color{textcolor}\rmfamily\fontsize{10.400000}{12.480000}\selectfont \(\displaystyle {15.0}\)}%
\end{pgfscope}%
\begin{pgfscope}%
\pgfpathrectangle{\pgfqpoint{1.000000in}{3.960000in}}{\pgfqpoint{3.100000in}{3.080000in}}%
\pgfusepath{clip}%
\pgfsetbuttcap%
\pgfsetroundjoin%
\pgfsetlinewidth{0.501875pt}%
\definecolor{currentstroke}{rgb}{0.000000,0.000000,0.000000}%
\pgfsetstrokecolor{currentstroke}%
\pgfsetstrokeopacity{0.450000}%
\pgfsetdash{{2.500000pt}{5.000000pt}}{0.000000pt}%
\pgfpathmoveto{\pgfqpoint{1.000000in}{6.856475in}}%
\pgfpathlineto{\pgfqpoint{4.100000in}{6.856475in}}%
\pgfusepath{stroke}%
\end{pgfscope}%
\begin{pgfscope}%
\pgfsetbuttcap%
\pgfsetroundjoin%
\definecolor{currentfill}{rgb}{0.000000,0.000000,0.000000}%
\pgfsetfillcolor{currentfill}%
\pgfsetlinewidth{0.803000pt}%
\definecolor{currentstroke}{rgb}{0.000000,0.000000,0.000000}%
\pgfsetstrokecolor{currentstroke}%
\pgfsetdash{}{0pt}%
\pgfsys@defobject{currentmarker}{\pgfqpoint{-0.048611in}{0.000000in}}{\pgfqpoint{-0.000000in}{0.000000in}}{%
\pgfpathmoveto{\pgfqpoint{-0.000000in}{0.000000in}}%
\pgfpathlineto{\pgfqpoint{-0.048611in}{0.000000in}}%
\pgfusepath{stroke,fill}%
}%
\begin{pgfscope}%
\pgfsys@transformshift{1.000000in}{6.856475in}%
\pgfsys@useobject{currentmarker}{}%
\end{pgfscope}%
\end{pgfscope}%
\begin{pgfscope}%
\definecolor{textcolor}{rgb}{0.000000,0.000000,0.000000}%
\pgfsetstrokecolor{textcolor}%
\pgfsetfillcolor{textcolor}%
\pgftext[x=0.655863in, y=6.808250in, left, base]{\color{textcolor}\rmfamily\fontsize{10.400000}{12.480000}\selectfont \(\displaystyle {15.5}\)}%
\end{pgfscope}%
\begin{pgfscope}%
\definecolor{textcolor}{rgb}{0.000000,0.000000,0.000000}%
\pgfsetstrokecolor{textcolor}%
\pgfsetfillcolor{textcolor}%
\pgftext[x=0.600308in,y=5.500000in,,bottom,rotate=90.000000]{\color{textcolor}\rmfamily\fontsize{12.800000}{15.360000}\selectfont loss}%
\end{pgfscope}%
\begin{pgfscope}%
\pgfpathrectangle{\pgfqpoint{1.000000in}{3.960000in}}{\pgfqpoint{3.100000in}{3.080000in}}%
\pgfusepath{clip}%
\pgfsetroundcap%
\pgfsetroundjoin%
\pgfsetlinewidth{1.204500pt}%
\definecolor{currentstroke}{rgb}{0.121569,0.466667,0.705882}%
\pgfsetstrokecolor{currentstroke}%
\pgfsetstrokeopacity{0.850000}%
\pgfsetdash{}{0pt}%
\pgfpathmoveto{\pgfqpoint{1.300000in}{6.737452in}}%
\pgfpathlineto{\pgfqpoint{1.312563in}{6.500510in}}%
\pgfpathlineto{\pgfqpoint{1.325126in}{6.338040in}}%
\pgfpathlineto{\pgfqpoint{1.337688in}{6.210629in}}%
\pgfpathlineto{\pgfqpoint{1.350251in}{6.105530in}}%
\pgfpathlineto{\pgfqpoint{1.362814in}{6.016633in}}%
\pgfpathlineto{\pgfqpoint{1.375377in}{5.940154in}}%
\pgfpathlineto{\pgfqpoint{1.387940in}{5.873484in}}%
\pgfpathlineto{\pgfqpoint{1.400503in}{5.814723in}}%
\pgfpathlineto{\pgfqpoint{1.413065in}{5.762439in}}%
\pgfpathlineto{\pgfqpoint{1.425628in}{5.715528in}}%
\pgfpathlineto{\pgfqpoint{1.450754in}{5.634564in}}%
\pgfpathlineto{\pgfqpoint{1.475879in}{5.566814in}}%
\pgfpathlineto{\pgfqpoint{1.501005in}{5.508934in}}%
\pgfpathlineto{\pgfqpoint{1.526131in}{5.458592in}}%
\pgfpathlineto{\pgfqpoint{1.551256in}{5.414123in}}%
\pgfpathlineto{\pgfqpoint{1.576382in}{5.374305in}}%
\pgfpathlineto{\pgfqpoint{1.601508in}{5.338225in}}%
\pgfpathlineto{\pgfqpoint{1.626633in}{5.305189in}}%
\pgfpathlineto{\pgfqpoint{1.664322in}{5.260219in}}%
\pgfpathlineto{\pgfqpoint{1.702010in}{5.219613in}}%
\pgfpathlineto{\pgfqpoint{1.739698in}{5.182443in}}%
\pgfpathlineto{\pgfqpoint{1.777387in}{5.148046in}}%
\pgfpathlineto{\pgfqpoint{1.827638in}{5.105667in}}%
\pgfpathlineto{\pgfqpoint{1.877889in}{5.066518in}}%
\pgfpathlineto{\pgfqpoint{1.928141in}{5.030000in}}%
\pgfpathlineto{\pgfqpoint{1.990955in}{4.987388in}}%
\pgfpathlineto{\pgfqpoint{2.053769in}{4.947564in}}%
\pgfpathlineto{\pgfqpoint{2.129146in}{4.902809in}}%
\pgfpathlineto{\pgfqpoint{2.204523in}{4.860813in}}%
\pgfpathlineto{\pgfqpoint{2.292462in}{4.814722in}}%
\pgfpathlineto{\pgfqpoint{2.380402in}{4.771266in}}%
\pgfpathlineto{\pgfqpoint{2.480905in}{4.724323in}}%
\pgfpathlineto{\pgfqpoint{2.581407in}{4.679848in}}%
\pgfpathlineto{\pgfqpoint{2.694472in}{4.632333in}}%
\pgfpathlineto{\pgfqpoint{2.807538in}{4.587126in}}%
\pgfpathlineto{\pgfqpoint{2.933166in}{4.539250in}}%
\pgfpathlineto{\pgfqpoint{3.071357in}{4.489095in}}%
\pgfpathlineto{\pgfqpoint{3.209548in}{4.441261in}}%
\pgfpathlineto{\pgfqpoint{3.360302in}{4.391436in}}%
\pgfpathlineto{\pgfqpoint{3.523618in}{4.339952in}}%
\pgfpathlineto{\pgfqpoint{3.686935in}{4.290816in}}%
\pgfpathlineto{\pgfqpoint{3.800000in}{4.258065in}}%
\pgfpathlineto{\pgfqpoint{3.800000in}{4.258065in}}%
\pgfusepath{stroke}%
\end{pgfscope}%
\begin{pgfscope}%
\pgfpathrectangle{\pgfqpoint{1.000000in}{3.960000in}}{\pgfqpoint{3.100000in}{3.080000in}}%
\pgfusepath{clip}%
\pgfsetroundcap%
\pgfsetroundjoin%
\pgfsetlinewidth{1.204500pt}%
\definecolor{currentstroke}{rgb}{1.000000,0.498039,0.054902}%
\pgfsetstrokecolor{currentstroke}%
\pgfsetstrokeopacity{0.850000}%
\pgfsetdash{}{0pt}%
\pgfpathmoveto{\pgfqpoint{1.300000in}{6.738849in}}%
\pgfpathlineto{\pgfqpoint{1.312563in}{6.504437in}}%
\pgfpathlineto{\pgfqpoint{1.325126in}{6.344874in}}%
\pgfpathlineto{\pgfqpoint{1.337688in}{6.220648in}}%
\pgfpathlineto{\pgfqpoint{1.350251in}{6.118920in}}%
\pgfpathlineto{\pgfqpoint{1.362814in}{6.033498in}}%
\pgfpathlineto{\pgfqpoint{1.375377in}{5.960541in}}%
\pgfpathlineto{\pgfqpoint{1.387940in}{5.897402in}}%
\pgfpathlineto{\pgfqpoint{1.400503in}{5.842156in}}%
\pgfpathlineto{\pgfqpoint{1.413065in}{5.793357in}}%
\pgfpathlineto{\pgfqpoint{1.425628in}{5.749891in}}%
\pgfpathlineto{\pgfqpoint{1.438191in}{5.710893in}}%
\pgfpathlineto{\pgfqpoint{1.463317in}{5.643690in}}%
\pgfpathlineto{\pgfqpoint{1.488442in}{5.587669in}}%
\pgfpathlineto{\pgfqpoint{1.513568in}{5.540066in}}%
\pgfpathlineto{\pgfqpoint{1.538693in}{5.498936in}}%
\pgfpathlineto{\pgfqpoint{1.563819in}{5.462879in}}%
\pgfpathlineto{\pgfqpoint{1.588945in}{5.430866in}}%
\pgfpathlineto{\pgfqpoint{1.614070in}{5.402125in}}%
\pgfpathlineto{\pgfqpoint{1.639196in}{5.376070in}}%
\pgfpathlineto{\pgfqpoint{1.676884in}{5.341067in}}%
\pgfpathlineto{\pgfqpoint{1.714573in}{5.309994in}}%
\pgfpathlineto{\pgfqpoint{1.752261in}{5.282054in}}%
\pgfpathlineto{\pgfqpoint{1.789950in}{5.256672in}}%
\pgfpathlineto{\pgfqpoint{1.840201in}{5.226090in}}%
\pgfpathlineto{\pgfqpoint{1.890452in}{5.198569in}}%
\pgfpathlineto{\pgfqpoint{1.940704in}{5.173576in}}%
\pgfpathlineto{\pgfqpoint{1.990955in}{5.150714in}}%
\pgfpathlineto{\pgfqpoint{2.053769in}{5.124672in}}%
\pgfpathlineto{\pgfqpoint{2.116583in}{5.101025in}}%
\pgfpathlineto{\pgfqpoint{2.191960in}{5.075317in}}%
\pgfpathlineto{\pgfqpoint{2.267337in}{5.052084in}}%
\pgfpathlineto{\pgfqpoint{2.355276in}{5.027631in}}%
\pgfpathlineto{\pgfqpoint{2.443216in}{5.005615in}}%
\pgfpathlineto{\pgfqpoint{2.543719in}{4.982996in}}%
\pgfpathlineto{\pgfqpoint{2.644221in}{4.962700in}}%
\pgfpathlineto{\pgfqpoint{2.757286in}{4.942248in}}%
\pgfpathlineto{\pgfqpoint{2.870352in}{4.923968in}}%
\pgfpathlineto{\pgfqpoint{2.995980in}{4.905851in}}%
\pgfpathlineto{\pgfqpoint{3.134171in}{4.888222in}}%
\pgfpathlineto{\pgfqpoint{3.284925in}{4.871346in}}%
\pgfpathlineto{\pgfqpoint{3.448241in}{4.855430in}}%
\pgfpathlineto{\pgfqpoint{3.624121in}{4.840629in}}%
\pgfpathlineto{\pgfqpoint{3.800000in}{4.827887in}}%
\pgfpathlineto{\pgfqpoint{3.800000in}{4.827887in}}%
\pgfusepath{stroke}%
\end{pgfscope}%
\begin{pgfscope}%
\pgfpathrectangle{\pgfqpoint{1.000000in}{3.960000in}}{\pgfqpoint{3.100000in}{3.080000in}}%
\pgfusepath{clip}%
\pgfsetroundcap%
\pgfsetroundjoin%
\pgfsetlinewidth{1.204500pt}%
\definecolor{currentstroke}{rgb}{0.172549,0.627451,0.172549}%
\pgfsetstrokecolor{currentstroke}%
\pgfsetstrokeopacity{0.850000}%
\pgfsetdash{}{0pt}%
\pgfpathmoveto{\pgfqpoint{1.300000in}{6.741935in}}%
\pgfpathlineto{\pgfqpoint{1.312563in}{6.513054in}}%
\pgfpathlineto{\pgfqpoint{1.325126in}{6.359772in}}%
\pgfpathlineto{\pgfqpoint{1.337688in}{6.242350in}}%
\pgfpathlineto{\pgfqpoint{1.350251in}{6.147739in}}%
\pgfpathlineto{\pgfqpoint{1.362814in}{6.069569in}}%
\pgfpathlineto{\pgfqpoint{1.375377in}{6.003879in}}%
\pgfpathlineto{\pgfqpoint{1.387940in}{5.947940in}}%
\pgfpathlineto{\pgfqpoint{1.400503in}{5.899778in}}%
\pgfpathlineto{\pgfqpoint{1.413065in}{5.857920in}}%
\pgfpathlineto{\pgfqpoint{1.425628in}{5.821240in}}%
\pgfpathlineto{\pgfqpoint{1.438191in}{5.788859in}}%
\pgfpathlineto{\pgfqpoint{1.450754in}{5.760089in}}%
\pgfpathlineto{\pgfqpoint{1.463317in}{5.734376in}}%
\pgfpathlineto{\pgfqpoint{1.488442in}{5.690409in}}%
\pgfpathlineto{\pgfqpoint{1.513568in}{5.654243in}}%
\pgfpathlineto{\pgfqpoint{1.538693in}{5.623994in}}%
\pgfpathlineto{\pgfqpoint{1.563819in}{5.598322in}}%
\pgfpathlineto{\pgfqpoint{1.588945in}{5.576254in}}%
\pgfpathlineto{\pgfqpoint{1.614070in}{5.557072in}}%
\pgfpathlineto{\pgfqpoint{1.639196in}{5.540236in}}%
\pgfpathlineto{\pgfqpoint{1.676884in}{5.518507in}}%
\pgfpathlineto{\pgfqpoint{1.714573in}{5.500130in}}%
\pgfpathlineto{\pgfqpoint{1.752261in}{5.484386in}}%
\pgfpathlineto{\pgfqpoint{1.789950in}{5.470758in}}%
\pgfpathlineto{\pgfqpoint{1.840201in}{5.455231in}}%
\pgfpathlineto{\pgfqpoint{1.890452in}{5.442127in}}%
\pgfpathlineto{\pgfqpoint{1.953266in}{5.428435in}}%
\pgfpathlineto{\pgfqpoint{2.016080in}{5.417107in}}%
\pgfpathlineto{\pgfqpoint{2.091457in}{5.405955in}}%
\pgfpathlineto{\pgfqpoint{2.179397in}{5.395553in}}%
\pgfpathlineto{\pgfqpoint{2.279899in}{5.386272in}}%
\pgfpathlineto{\pgfqpoint{2.392965in}{5.378298in}}%
\pgfpathlineto{\pgfqpoint{2.518593in}{5.371674in}}%
\pgfpathlineto{\pgfqpoint{2.669347in}{5.365933in}}%
\pgfpathlineto{\pgfqpoint{2.857789in}{5.361058in}}%
\pgfpathlineto{\pgfqpoint{3.096482in}{5.357212in}}%
\pgfpathlineto{\pgfqpoint{3.423116in}{5.354346in}}%
\pgfpathlineto{\pgfqpoint{3.800000in}{5.352781in}}%
\pgfpathlineto{\pgfqpoint{3.800000in}{5.352781in}}%
\pgfusepath{stroke}%
\end{pgfscope}%
\begin{pgfscope}%
\pgfsetrectcap%
\pgfsetmiterjoin%
\pgfsetlinewidth{0.803000pt}%
\definecolor{currentstroke}{rgb}{0.000000,0.000000,0.000000}%
\pgfsetstrokecolor{currentstroke}%
\pgfsetdash{}{0pt}%
\pgfpathmoveto{\pgfqpoint{1.000000in}{3.960000in}}%
\pgfpathlineto{\pgfqpoint{1.000000in}{7.040000in}}%
\pgfusepath{stroke}%
\end{pgfscope}%
\begin{pgfscope}%
\pgfsetrectcap%
\pgfsetmiterjoin%
\pgfsetlinewidth{0.803000pt}%
\definecolor{currentstroke}{rgb}{0.000000,0.000000,0.000000}%
\pgfsetstrokecolor{currentstroke}%
\pgfsetdash{}{0pt}%
\pgfpathmoveto{\pgfqpoint{4.100000in}{3.960000in}}%
\pgfpathlineto{\pgfqpoint{4.100000in}{7.040000in}}%
\pgfusepath{stroke}%
\end{pgfscope}%
\begin{pgfscope}%
\pgfsetrectcap%
\pgfsetmiterjoin%
\pgfsetlinewidth{0.803000pt}%
\definecolor{currentstroke}{rgb}{0.000000,0.000000,0.000000}%
\pgfsetstrokecolor{currentstroke}%
\pgfsetdash{}{0pt}%
\pgfpathmoveto{\pgfqpoint{1.000000in}{3.960000in}}%
\pgfpathlineto{\pgfqpoint{4.100000in}{3.960000in}}%
\pgfusepath{stroke}%
\end{pgfscope}%
\begin{pgfscope}%
\pgfsetrectcap%
\pgfsetmiterjoin%
\pgfsetlinewidth{0.803000pt}%
\definecolor{currentstroke}{rgb}{0.000000,0.000000,0.000000}%
\pgfsetstrokecolor{currentstroke}%
\pgfsetdash{}{0pt}%
\pgfpathmoveto{\pgfqpoint{1.000000in}{7.040000in}}%
\pgfpathlineto{\pgfqpoint{4.100000in}{7.040000in}}%
\pgfusepath{stroke}%
\end{pgfscope}%
\begin{pgfscope}%
\definecolor{textcolor}{rgb}{0.000000,0.000000,0.000000}%
\pgfsetstrokecolor{textcolor}%
\pgfsetfillcolor{textcolor}%
\pgftext[x=2.550000in,y=7.123333in,,base]{\color{textcolor}\rmfamily\fontsize{12.800000}{15.360000}\selectfont Train}%
\end{pgfscope}%
\begin{pgfscope}%
\pgfsetbuttcap%
\pgfsetmiterjoin%
\definecolor{currentfill}{rgb}{1.000000,1.000000,1.000000}%
\pgfsetfillcolor{currentfill}%
\pgfsetlinewidth{0.000000pt}%
\definecolor{currentstroke}{rgb}{0.000000,0.000000,0.000000}%
\pgfsetstrokecolor{currentstroke}%
\pgfsetstrokeopacity{0.000000}%
\pgfsetdash{}{0pt}%
\pgfpathmoveto{\pgfqpoint{4.100000in}{3.960000in}}%
\pgfpathlineto{\pgfqpoint{7.200000in}{3.960000in}}%
\pgfpathlineto{\pgfqpoint{7.200000in}{7.040000in}}%
\pgfpathlineto{\pgfqpoint{4.100000in}{7.040000in}}%
\pgfpathclose%
\pgfusepath{fill}%
\end{pgfscope}%
\begin{pgfscope}%
\pgfpathrectangle{\pgfqpoint{4.100000in}{3.960000in}}{\pgfqpoint{3.100000in}{3.080000in}}%
\pgfusepath{clip}%
\pgfsetbuttcap%
\pgfsetroundjoin%
\pgfsetlinewidth{0.501875pt}%
\definecolor{currentstroke}{rgb}{0.000000,0.000000,0.000000}%
\pgfsetstrokecolor{currentstroke}%
\pgfsetstrokeopacity{0.450000}%
\pgfsetdash{{2.500000pt}{5.000000pt}}{0.000000pt}%
\pgfpathmoveto{\pgfqpoint{4.400000in}{3.960000in}}%
\pgfpathlineto{\pgfqpoint{4.400000in}{7.040000in}}%
\pgfusepath{stroke}%
\end{pgfscope}%
\begin{pgfscope}%
\pgfsetbuttcap%
\pgfsetroundjoin%
\definecolor{currentfill}{rgb}{0.000000,0.000000,0.000000}%
\pgfsetfillcolor{currentfill}%
\pgfsetlinewidth{0.803000pt}%
\definecolor{currentstroke}{rgb}{0.000000,0.000000,0.000000}%
\pgfsetstrokecolor{currentstroke}%
\pgfsetdash{}{0pt}%
\pgfsys@defobject{currentmarker}{\pgfqpoint{0.000000in}{0.000000in}}{\pgfqpoint{0.000000in}{0.048611in}}{%
\pgfpathmoveto{\pgfqpoint{0.000000in}{0.000000in}}%
\pgfpathlineto{\pgfqpoint{0.000000in}{0.048611in}}%
\pgfusepath{stroke,fill}%
}%
\begin{pgfscope}%
\pgfsys@transformshift{4.400000in}{3.960000in}%
\pgfsys@useobject{currentmarker}{}%
\end{pgfscope}%
\end{pgfscope}%
\begin{pgfscope}%
\pgfpathrectangle{\pgfqpoint{4.100000in}{3.960000in}}{\pgfqpoint{3.100000in}{3.080000in}}%
\pgfusepath{clip}%
\pgfsetbuttcap%
\pgfsetroundjoin%
\pgfsetlinewidth{0.501875pt}%
\definecolor{currentstroke}{rgb}{0.000000,0.000000,0.000000}%
\pgfsetstrokecolor{currentstroke}%
\pgfsetstrokeopacity{0.450000}%
\pgfsetdash{{2.500000pt}{5.000000pt}}{0.000000pt}%
\pgfpathmoveto{\pgfqpoint{5.028141in}{3.960000in}}%
\pgfpathlineto{\pgfqpoint{5.028141in}{7.040000in}}%
\pgfusepath{stroke}%
\end{pgfscope}%
\begin{pgfscope}%
\pgfsetbuttcap%
\pgfsetroundjoin%
\definecolor{currentfill}{rgb}{0.000000,0.000000,0.000000}%
\pgfsetfillcolor{currentfill}%
\pgfsetlinewidth{0.803000pt}%
\definecolor{currentstroke}{rgb}{0.000000,0.000000,0.000000}%
\pgfsetstrokecolor{currentstroke}%
\pgfsetdash{}{0pt}%
\pgfsys@defobject{currentmarker}{\pgfqpoint{0.000000in}{0.000000in}}{\pgfqpoint{0.000000in}{0.048611in}}{%
\pgfpathmoveto{\pgfqpoint{0.000000in}{0.000000in}}%
\pgfpathlineto{\pgfqpoint{0.000000in}{0.048611in}}%
\pgfusepath{stroke,fill}%
}%
\begin{pgfscope}%
\pgfsys@transformshift{5.028141in}{3.960000in}%
\pgfsys@useobject{currentmarker}{}%
\end{pgfscope}%
\end{pgfscope}%
\begin{pgfscope}%
\pgfpathrectangle{\pgfqpoint{4.100000in}{3.960000in}}{\pgfqpoint{3.100000in}{3.080000in}}%
\pgfusepath{clip}%
\pgfsetbuttcap%
\pgfsetroundjoin%
\pgfsetlinewidth{0.501875pt}%
\definecolor{currentstroke}{rgb}{0.000000,0.000000,0.000000}%
\pgfsetstrokecolor{currentstroke}%
\pgfsetstrokeopacity{0.450000}%
\pgfsetdash{{2.500000pt}{5.000000pt}}{0.000000pt}%
\pgfpathmoveto{\pgfqpoint{5.656281in}{3.960000in}}%
\pgfpathlineto{\pgfqpoint{5.656281in}{7.040000in}}%
\pgfusepath{stroke}%
\end{pgfscope}%
\begin{pgfscope}%
\pgfsetbuttcap%
\pgfsetroundjoin%
\definecolor{currentfill}{rgb}{0.000000,0.000000,0.000000}%
\pgfsetfillcolor{currentfill}%
\pgfsetlinewidth{0.803000pt}%
\definecolor{currentstroke}{rgb}{0.000000,0.000000,0.000000}%
\pgfsetstrokecolor{currentstroke}%
\pgfsetdash{}{0pt}%
\pgfsys@defobject{currentmarker}{\pgfqpoint{0.000000in}{0.000000in}}{\pgfqpoint{0.000000in}{0.048611in}}{%
\pgfpathmoveto{\pgfqpoint{0.000000in}{0.000000in}}%
\pgfpathlineto{\pgfqpoint{0.000000in}{0.048611in}}%
\pgfusepath{stroke,fill}%
}%
\begin{pgfscope}%
\pgfsys@transformshift{5.656281in}{3.960000in}%
\pgfsys@useobject{currentmarker}{}%
\end{pgfscope}%
\end{pgfscope}%
\begin{pgfscope}%
\pgfpathrectangle{\pgfqpoint{4.100000in}{3.960000in}}{\pgfqpoint{3.100000in}{3.080000in}}%
\pgfusepath{clip}%
\pgfsetbuttcap%
\pgfsetroundjoin%
\pgfsetlinewidth{0.501875pt}%
\definecolor{currentstroke}{rgb}{0.000000,0.000000,0.000000}%
\pgfsetstrokecolor{currentstroke}%
\pgfsetstrokeopacity{0.450000}%
\pgfsetdash{{2.500000pt}{5.000000pt}}{0.000000pt}%
\pgfpathmoveto{\pgfqpoint{6.284422in}{3.960000in}}%
\pgfpathlineto{\pgfqpoint{6.284422in}{7.040000in}}%
\pgfusepath{stroke}%
\end{pgfscope}%
\begin{pgfscope}%
\pgfsetbuttcap%
\pgfsetroundjoin%
\definecolor{currentfill}{rgb}{0.000000,0.000000,0.000000}%
\pgfsetfillcolor{currentfill}%
\pgfsetlinewidth{0.803000pt}%
\definecolor{currentstroke}{rgb}{0.000000,0.000000,0.000000}%
\pgfsetstrokecolor{currentstroke}%
\pgfsetdash{}{0pt}%
\pgfsys@defobject{currentmarker}{\pgfqpoint{0.000000in}{0.000000in}}{\pgfqpoint{0.000000in}{0.048611in}}{%
\pgfpathmoveto{\pgfqpoint{0.000000in}{0.000000in}}%
\pgfpathlineto{\pgfqpoint{0.000000in}{0.048611in}}%
\pgfusepath{stroke,fill}%
}%
\begin{pgfscope}%
\pgfsys@transformshift{6.284422in}{3.960000in}%
\pgfsys@useobject{currentmarker}{}%
\end{pgfscope}%
\end{pgfscope}%
\begin{pgfscope}%
\pgfpathrectangle{\pgfqpoint{4.100000in}{3.960000in}}{\pgfqpoint{3.100000in}{3.080000in}}%
\pgfusepath{clip}%
\pgfsetbuttcap%
\pgfsetroundjoin%
\pgfsetlinewidth{0.501875pt}%
\definecolor{currentstroke}{rgb}{0.000000,0.000000,0.000000}%
\pgfsetstrokecolor{currentstroke}%
\pgfsetstrokeopacity{0.450000}%
\pgfsetdash{{2.500000pt}{5.000000pt}}{0.000000pt}%
\pgfpathmoveto{\pgfqpoint{6.912563in}{3.960000in}}%
\pgfpathlineto{\pgfqpoint{6.912563in}{7.040000in}}%
\pgfusepath{stroke}%
\end{pgfscope}%
\begin{pgfscope}%
\pgfsetbuttcap%
\pgfsetroundjoin%
\definecolor{currentfill}{rgb}{0.000000,0.000000,0.000000}%
\pgfsetfillcolor{currentfill}%
\pgfsetlinewidth{0.803000pt}%
\definecolor{currentstroke}{rgb}{0.000000,0.000000,0.000000}%
\pgfsetstrokecolor{currentstroke}%
\pgfsetdash{}{0pt}%
\pgfsys@defobject{currentmarker}{\pgfqpoint{0.000000in}{0.000000in}}{\pgfqpoint{0.000000in}{0.048611in}}{%
\pgfpathmoveto{\pgfqpoint{0.000000in}{0.000000in}}%
\pgfpathlineto{\pgfqpoint{0.000000in}{0.048611in}}%
\pgfusepath{stroke,fill}%
}%
\begin{pgfscope}%
\pgfsys@transformshift{6.912563in}{3.960000in}%
\pgfsys@useobject{currentmarker}{}%
\end{pgfscope}%
\end{pgfscope}%
\begin{pgfscope}%
\pgfpathrectangle{\pgfqpoint{4.100000in}{3.960000in}}{\pgfqpoint{3.100000in}{3.080000in}}%
\pgfusepath{clip}%
\pgfsetbuttcap%
\pgfsetroundjoin%
\pgfsetlinewidth{0.501875pt}%
\definecolor{currentstroke}{rgb}{0.000000,0.000000,0.000000}%
\pgfsetstrokecolor{currentstroke}%
\pgfsetstrokeopacity{0.450000}%
\pgfsetdash{{2.500000pt}{5.000000pt}}{0.000000pt}%
\pgfpathmoveto{\pgfqpoint{4.100000in}{3.968029in}}%
\pgfpathlineto{\pgfqpoint{7.200000in}{3.968029in}}%
\pgfusepath{stroke}%
\end{pgfscope}%
\begin{pgfscope}%
\pgfsetbuttcap%
\pgfsetroundjoin%
\definecolor{currentfill}{rgb}{0.000000,0.000000,0.000000}%
\pgfsetfillcolor{currentfill}%
\pgfsetlinewidth{0.803000pt}%
\definecolor{currentstroke}{rgb}{0.000000,0.000000,0.000000}%
\pgfsetstrokecolor{currentstroke}%
\pgfsetdash{}{0pt}%
\pgfsys@defobject{currentmarker}{\pgfqpoint{-0.024306in}{-0.000000in}}{\pgfqpoint{0.024306in}{0.000000in}}{%
\pgfpathmoveto{\pgfqpoint{0.024306in}{-0.000000in}}%
\pgfpathlineto{\pgfqpoint{-0.024306in}{0.000000in}}%
\pgfusepath{stroke,fill}%
}%
\begin{pgfscope}%
\pgfsys@transformshift{4.100000in}{3.968029in}%
\pgfsys@useobject{currentmarker}{}%
\end{pgfscope}%
\end{pgfscope}%
\begin{pgfscope}%
\pgfpathrectangle{\pgfqpoint{4.100000in}{3.960000in}}{\pgfqpoint{3.100000in}{3.080000in}}%
\pgfusepath{clip}%
\pgfsetbuttcap%
\pgfsetroundjoin%
\pgfsetlinewidth{0.501875pt}%
\definecolor{currentstroke}{rgb}{0.000000,0.000000,0.000000}%
\pgfsetstrokecolor{currentstroke}%
\pgfsetstrokeopacity{0.450000}%
\pgfsetdash{{2.500000pt}{5.000000pt}}{0.000000pt}%
\pgfpathmoveto{\pgfqpoint{4.100000in}{4.329085in}}%
\pgfpathlineto{\pgfqpoint{7.200000in}{4.329085in}}%
\pgfusepath{stroke}%
\end{pgfscope}%
\begin{pgfscope}%
\pgfsetbuttcap%
\pgfsetroundjoin%
\definecolor{currentfill}{rgb}{0.000000,0.000000,0.000000}%
\pgfsetfillcolor{currentfill}%
\pgfsetlinewidth{0.803000pt}%
\definecolor{currentstroke}{rgb}{0.000000,0.000000,0.000000}%
\pgfsetstrokecolor{currentstroke}%
\pgfsetdash{}{0pt}%
\pgfsys@defobject{currentmarker}{\pgfqpoint{-0.024306in}{-0.000000in}}{\pgfqpoint{0.024306in}{0.000000in}}{%
\pgfpathmoveto{\pgfqpoint{0.024306in}{-0.000000in}}%
\pgfpathlineto{\pgfqpoint{-0.024306in}{0.000000in}}%
\pgfusepath{stroke,fill}%
}%
\begin{pgfscope}%
\pgfsys@transformshift{4.100000in}{4.329085in}%
\pgfsys@useobject{currentmarker}{}%
\end{pgfscope}%
\end{pgfscope}%
\begin{pgfscope}%
\pgfpathrectangle{\pgfqpoint{4.100000in}{3.960000in}}{\pgfqpoint{3.100000in}{3.080000in}}%
\pgfusepath{clip}%
\pgfsetbuttcap%
\pgfsetroundjoin%
\pgfsetlinewidth{0.501875pt}%
\definecolor{currentstroke}{rgb}{0.000000,0.000000,0.000000}%
\pgfsetstrokecolor{currentstroke}%
\pgfsetstrokeopacity{0.450000}%
\pgfsetdash{{2.500000pt}{5.000000pt}}{0.000000pt}%
\pgfpathmoveto{\pgfqpoint{4.100000in}{4.690141in}}%
\pgfpathlineto{\pgfqpoint{7.200000in}{4.690141in}}%
\pgfusepath{stroke}%
\end{pgfscope}%
\begin{pgfscope}%
\pgfsetbuttcap%
\pgfsetroundjoin%
\definecolor{currentfill}{rgb}{0.000000,0.000000,0.000000}%
\pgfsetfillcolor{currentfill}%
\pgfsetlinewidth{0.803000pt}%
\definecolor{currentstroke}{rgb}{0.000000,0.000000,0.000000}%
\pgfsetstrokecolor{currentstroke}%
\pgfsetdash{}{0pt}%
\pgfsys@defobject{currentmarker}{\pgfqpoint{-0.024306in}{-0.000000in}}{\pgfqpoint{0.024306in}{0.000000in}}{%
\pgfpathmoveto{\pgfqpoint{0.024306in}{-0.000000in}}%
\pgfpathlineto{\pgfqpoint{-0.024306in}{0.000000in}}%
\pgfusepath{stroke,fill}%
}%
\begin{pgfscope}%
\pgfsys@transformshift{4.100000in}{4.690141in}%
\pgfsys@useobject{currentmarker}{}%
\end{pgfscope}%
\end{pgfscope}%
\begin{pgfscope}%
\pgfpathrectangle{\pgfqpoint{4.100000in}{3.960000in}}{\pgfqpoint{3.100000in}{3.080000in}}%
\pgfusepath{clip}%
\pgfsetbuttcap%
\pgfsetroundjoin%
\pgfsetlinewidth{0.501875pt}%
\definecolor{currentstroke}{rgb}{0.000000,0.000000,0.000000}%
\pgfsetstrokecolor{currentstroke}%
\pgfsetstrokeopacity{0.450000}%
\pgfsetdash{{2.500000pt}{5.000000pt}}{0.000000pt}%
\pgfpathmoveto{\pgfqpoint{4.100000in}{5.051196in}}%
\pgfpathlineto{\pgfqpoint{7.200000in}{5.051196in}}%
\pgfusepath{stroke}%
\end{pgfscope}%
\begin{pgfscope}%
\pgfsetbuttcap%
\pgfsetroundjoin%
\definecolor{currentfill}{rgb}{0.000000,0.000000,0.000000}%
\pgfsetfillcolor{currentfill}%
\pgfsetlinewidth{0.803000pt}%
\definecolor{currentstroke}{rgb}{0.000000,0.000000,0.000000}%
\pgfsetstrokecolor{currentstroke}%
\pgfsetdash{}{0pt}%
\pgfsys@defobject{currentmarker}{\pgfqpoint{-0.024306in}{-0.000000in}}{\pgfqpoint{0.024306in}{0.000000in}}{%
\pgfpathmoveto{\pgfqpoint{0.024306in}{-0.000000in}}%
\pgfpathlineto{\pgfqpoint{-0.024306in}{0.000000in}}%
\pgfusepath{stroke,fill}%
}%
\begin{pgfscope}%
\pgfsys@transformshift{4.100000in}{5.051196in}%
\pgfsys@useobject{currentmarker}{}%
\end{pgfscope}%
\end{pgfscope}%
\begin{pgfscope}%
\pgfpathrectangle{\pgfqpoint{4.100000in}{3.960000in}}{\pgfqpoint{3.100000in}{3.080000in}}%
\pgfusepath{clip}%
\pgfsetbuttcap%
\pgfsetroundjoin%
\pgfsetlinewidth{0.501875pt}%
\definecolor{currentstroke}{rgb}{0.000000,0.000000,0.000000}%
\pgfsetstrokecolor{currentstroke}%
\pgfsetstrokeopacity{0.450000}%
\pgfsetdash{{2.500000pt}{5.000000pt}}{0.000000pt}%
\pgfpathmoveto{\pgfqpoint{4.100000in}{5.412252in}}%
\pgfpathlineto{\pgfqpoint{7.200000in}{5.412252in}}%
\pgfusepath{stroke}%
\end{pgfscope}%
\begin{pgfscope}%
\pgfsetbuttcap%
\pgfsetroundjoin%
\definecolor{currentfill}{rgb}{0.000000,0.000000,0.000000}%
\pgfsetfillcolor{currentfill}%
\pgfsetlinewidth{0.803000pt}%
\definecolor{currentstroke}{rgb}{0.000000,0.000000,0.000000}%
\pgfsetstrokecolor{currentstroke}%
\pgfsetdash{}{0pt}%
\pgfsys@defobject{currentmarker}{\pgfqpoint{-0.024306in}{-0.000000in}}{\pgfqpoint{0.024306in}{0.000000in}}{%
\pgfpathmoveto{\pgfqpoint{0.024306in}{-0.000000in}}%
\pgfpathlineto{\pgfqpoint{-0.024306in}{0.000000in}}%
\pgfusepath{stroke,fill}%
}%
\begin{pgfscope}%
\pgfsys@transformshift{4.100000in}{5.412252in}%
\pgfsys@useobject{currentmarker}{}%
\end{pgfscope}%
\end{pgfscope}%
\begin{pgfscope}%
\pgfpathrectangle{\pgfqpoint{4.100000in}{3.960000in}}{\pgfqpoint{3.100000in}{3.080000in}}%
\pgfusepath{clip}%
\pgfsetbuttcap%
\pgfsetroundjoin%
\pgfsetlinewidth{0.501875pt}%
\definecolor{currentstroke}{rgb}{0.000000,0.000000,0.000000}%
\pgfsetstrokecolor{currentstroke}%
\pgfsetstrokeopacity{0.450000}%
\pgfsetdash{{2.500000pt}{5.000000pt}}{0.000000pt}%
\pgfpathmoveto{\pgfqpoint{4.100000in}{5.773308in}}%
\pgfpathlineto{\pgfqpoint{7.200000in}{5.773308in}}%
\pgfusepath{stroke}%
\end{pgfscope}%
\begin{pgfscope}%
\pgfsetbuttcap%
\pgfsetroundjoin%
\definecolor{currentfill}{rgb}{0.000000,0.000000,0.000000}%
\pgfsetfillcolor{currentfill}%
\pgfsetlinewidth{0.803000pt}%
\definecolor{currentstroke}{rgb}{0.000000,0.000000,0.000000}%
\pgfsetstrokecolor{currentstroke}%
\pgfsetdash{}{0pt}%
\pgfsys@defobject{currentmarker}{\pgfqpoint{-0.024306in}{-0.000000in}}{\pgfqpoint{0.024306in}{0.000000in}}{%
\pgfpathmoveto{\pgfqpoint{0.024306in}{-0.000000in}}%
\pgfpathlineto{\pgfqpoint{-0.024306in}{0.000000in}}%
\pgfusepath{stroke,fill}%
}%
\begin{pgfscope}%
\pgfsys@transformshift{4.100000in}{5.773308in}%
\pgfsys@useobject{currentmarker}{}%
\end{pgfscope}%
\end{pgfscope}%
\begin{pgfscope}%
\pgfpathrectangle{\pgfqpoint{4.100000in}{3.960000in}}{\pgfqpoint{3.100000in}{3.080000in}}%
\pgfusepath{clip}%
\pgfsetbuttcap%
\pgfsetroundjoin%
\pgfsetlinewidth{0.501875pt}%
\definecolor{currentstroke}{rgb}{0.000000,0.000000,0.000000}%
\pgfsetstrokecolor{currentstroke}%
\pgfsetstrokeopacity{0.450000}%
\pgfsetdash{{2.500000pt}{5.000000pt}}{0.000000pt}%
\pgfpathmoveto{\pgfqpoint{4.100000in}{6.134364in}}%
\pgfpathlineto{\pgfqpoint{7.200000in}{6.134364in}}%
\pgfusepath{stroke}%
\end{pgfscope}%
\begin{pgfscope}%
\pgfsetbuttcap%
\pgfsetroundjoin%
\definecolor{currentfill}{rgb}{0.000000,0.000000,0.000000}%
\pgfsetfillcolor{currentfill}%
\pgfsetlinewidth{0.803000pt}%
\definecolor{currentstroke}{rgb}{0.000000,0.000000,0.000000}%
\pgfsetstrokecolor{currentstroke}%
\pgfsetdash{}{0pt}%
\pgfsys@defobject{currentmarker}{\pgfqpoint{-0.024306in}{-0.000000in}}{\pgfqpoint{0.024306in}{0.000000in}}{%
\pgfpathmoveto{\pgfqpoint{0.024306in}{-0.000000in}}%
\pgfpathlineto{\pgfqpoint{-0.024306in}{0.000000in}}%
\pgfusepath{stroke,fill}%
}%
\begin{pgfscope}%
\pgfsys@transformshift{4.100000in}{6.134364in}%
\pgfsys@useobject{currentmarker}{}%
\end{pgfscope}%
\end{pgfscope}%
\begin{pgfscope}%
\pgfpathrectangle{\pgfqpoint{4.100000in}{3.960000in}}{\pgfqpoint{3.100000in}{3.080000in}}%
\pgfusepath{clip}%
\pgfsetbuttcap%
\pgfsetroundjoin%
\pgfsetlinewidth{0.501875pt}%
\definecolor{currentstroke}{rgb}{0.000000,0.000000,0.000000}%
\pgfsetstrokecolor{currentstroke}%
\pgfsetstrokeopacity{0.450000}%
\pgfsetdash{{2.500000pt}{5.000000pt}}{0.000000pt}%
\pgfpathmoveto{\pgfqpoint{4.100000in}{6.495420in}}%
\pgfpathlineto{\pgfqpoint{7.200000in}{6.495420in}}%
\pgfusepath{stroke}%
\end{pgfscope}%
\begin{pgfscope}%
\pgfsetbuttcap%
\pgfsetroundjoin%
\definecolor{currentfill}{rgb}{0.000000,0.000000,0.000000}%
\pgfsetfillcolor{currentfill}%
\pgfsetlinewidth{0.803000pt}%
\definecolor{currentstroke}{rgb}{0.000000,0.000000,0.000000}%
\pgfsetstrokecolor{currentstroke}%
\pgfsetdash{}{0pt}%
\pgfsys@defobject{currentmarker}{\pgfqpoint{-0.024306in}{-0.000000in}}{\pgfqpoint{0.024306in}{0.000000in}}{%
\pgfpathmoveto{\pgfqpoint{0.024306in}{-0.000000in}}%
\pgfpathlineto{\pgfqpoint{-0.024306in}{0.000000in}}%
\pgfusepath{stroke,fill}%
}%
\begin{pgfscope}%
\pgfsys@transformshift{4.100000in}{6.495420in}%
\pgfsys@useobject{currentmarker}{}%
\end{pgfscope}%
\end{pgfscope}%
\begin{pgfscope}%
\pgfpathrectangle{\pgfqpoint{4.100000in}{3.960000in}}{\pgfqpoint{3.100000in}{3.080000in}}%
\pgfusepath{clip}%
\pgfsetbuttcap%
\pgfsetroundjoin%
\pgfsetlinewidth{0.501875pt}%
\definecolor{currentstroke}{rgb}{0.000000,0.000000,0.000000}%
\pgfsetstrokecolor{currentstroke}%
\pgfsetstrokeopacity{0.450000}%
\pgfsetdash{{2.500000pt}{5.000000pt}}{0.000000pt}%
\pgfpathmoveto{\pgfqpoint{4.100000in}{6.856475in}}%
\pgfpathlineto{\pgfqpoint{7.200000in}{6.856475in}}%
\pgfusepath{stroke}%
\end{pgfscope}%
\begin{pgfscope}%
\pgfsetbuttcap%
\pgfsetroundjoin%
\definecolor{currentfill}{rgb}{0.000000,0.000000,0.000000}%
\pgfsetfillcolor{currentfill}%
\pgfsetlinewidth{0.803000pt}%
\definecolor{currentstroke}{rgb}{0.000000,0.000000,0.000000}%
\pgfsetstrokecolor{currentstroke}%
\pgfsetdash{}{0pt}%
\pgfsys@defobject{currentmarker}{\pgfqpoint{-0.024306in}{-0.000000in}}{\pgfqpoint{0.024306in}{0.000000in}}{%
\pgfpathmoveto{\pgfqpoint{0.024306in}{-0.000000in}}%
\pgfpathlineto{\pgfqpoint{-0.024306in}{0.000000in}}%
\pgfusepath{stroke,fill}%
}%
\begin{pgfscope}%
\pgfsys@transformshift{4.100000in}{6.856475in}%
\pgfsys@useobject{currentmarker}{}%
\end{pgfscope}%
\end{pgfscope}%
\begin{pgfscope}%
\pgfpathrectangle{\pgfqpoint{4.100000in}{3.960000in}}{\pgfqpoint{3.100000in}{3.080000in}}%
\pgfusepath{clip}%
\pgfsetroundcap%
\pgfsetroundjoin%
\pgfsetlinewidth{1.204500pt}%
\definecolor{currentstroke}{rgb}{0.121569,0.466667,0.705882}%
\pgfsetstrokecolor{currentstroke}%
\pgfsetstrokeopacity{0.850000}%
\pgfsetdash{}{0pt}%
\pgfpathmoveto{\pgfqpoint{4.400000in}{6.703210in}}%
\pgfpathlineto{\pgfqpoint{4.412563in}{6.452234in}}%
\pgfpathlineto{\pgfqpoint{4.425126in}{6.283901in}}%
\pgfpathlineto{\pgfqpoint{4.437688in}{6.154238in}}%
\pgfpathlineto{\pgfqpoint{4.450251in}{6.048612in}}%
\pgfpathlineto{\pgfqpoint{4.462814in}{5.960081in}}%
\pgfpathlineto{\pgfqpoint{4.475377in}{5.884477in}}%
\pgfpathlineto{\pgfqpoint{4.487940in}{5.818982in}}%
\pgfpathlineto{\pgfqpoint{4.500503in}{5.761585in}}%
\pgfpathlineto{\pgfqpoint{4.513065in}{5.710785in}}%
\pgfpathlineto{\pgfqpoint{4.525628in}{5.665438in}}%
\pgfpathlineto{\pgfqpoint{4.538191in}{5.624651in}}%
\pgfpathlineto{\pgfqpoint{4.563317in}{5.554078in}}%
\pgfpathlineto{\pgfqpoint{4.588442in}{5.494858in}}%
\pgfpathlineto{\pgfqpoint{4.613568in}{5.444123in}}%
\pgfpathlineto{\pgfqpoint{4.638693in}{5.399866in}}%
\pgfpathlineto{\pgfqpoint{4.663819in}{5.360661in}}%
\pgfpathlineto{\pgfqpoint{4.688945in}{5.325444in}}%
\pgfpathlineto{\pgfqpoint{4.714070in}{5.293424in}}%
\pgfpathlineto{\pgfqpoint{4.751759in}{5.250131in}}%
\pgfpathlineto{\pgfqpoint{4.789447in}{5.211277in}}%
\pgfpathlineto{\pgfqpoint{4.827136in}{5.175855in}}%
\pgfpathlineto{\pgfqpoint{4.864824in}{5.143153in}}%
\pgfpathlineto{\pgfqpoint{4.915075in}{5.102912in}}%
\pgfpathlineto{\pgfqpoint{4.965327in}{5.065746in}}%
\pgfpathlineto{\pgfqpoint{5.015578in}{5.031056in}}%
\pgfpathlineto{\pgfqpoint{5.078392in}{4.990534in}}%
\pgfpathlineto{\pgfqpoint{5.141206in}{4.952607in}}%
\pgfpathlineto{\pgfqpoint{5.216583in}{4.909914in}}%
\pgfpathlineto{\pgfqpoint{5.291960in}{4.869792in}}%
\pgfpathlineto{\pgfqpoint{5.379899in}{4.825703in}}%
\pgfpathlineto{\pgfqpoint{5.467839in}{4.784092in}}%
\pgfpathlineto{\pgfqpoint{5.568342in}{4.739110in}}%
\pgfpathlineto{\pgfqpoint{5.668844in}{4.696474in}}%
\pgfpathlineto{\pgfqpoint{5.781910in}{4.650916in}}%
\pgfpathlineto{\pgfqpoint{5.907538in}{4.602876in}}%
\pgfpathlineto{\pgfqpoint{6.033166in}{4.557200in}}%
\pgfpathlineto{\pgfqpoint{6.171357in}{4.509350in}}%
\pgfpathlineto{\pgfqpoint{6.322111in}{4.459677in}}%
\pgfpathlineto{\pgfqpoint{6.472864in}{4.412356in}}%
\pgfpathlineto{\pgfqpoint{6.636181in}{4.363477in}}%
\pgfpathlineto{\pgfqpoint{6.799497in}{4.316848in}}%
\pgfpathlineto{\pgfqpoint{6.900000in}{4.289188in}}%
\pgfpathlineto{\pgfqpoint{6.900000in}{4.289188in}}%
\pgfusepath{stroke}%
\end{pgfscope}%
\begin{pgfscope}%
\pgfpathrectangle{\pgfqpoint{4.100000in}{3.960000in}}{\pgfqpoint{3.100000in}{3.080000in}}%
\pgfusepath{clip}%
\pgfsetroundcap%
\pgfsetroundjoin%
\pgfsetlinewidth{1.204500pt}%
\definecolor{currentstroke}{rgb}{1.000000,0.498039,0.054902}%
\pgfsetstrokecolor{currentstroke}%
\pgfsetstrokeopacity{0.850000}%
\pgfsetdash{}{0pt}%
\pgfpathmoveto{\pgfqpoint{4.400000in}{6.704653in}}%
\pgfpathlineto{\pgfqpoint{4.412563in}{6.456289in}}%
\pgfpathlineto{\pgfqpoint{4.425126in}{6.290927in}}%
\pgfpathlineto{\pgfqpoint{4.437688in}{6.164488in}}%
\pgfpathlineto{\pgfqpoint{4.450251in}{6.062250in}}%
\pgfpathlineto{\pgfqpoint{4.462814in}{5.977196in}}%
\pgfpathlineto{\pgfqpoint{4.475377in}{5.905100in}}%
\pgfpathlineto{\pgfqpoint{4.487940in}{5.843111in}}%
\pgfpathlineto{\pgfqpoint{4.500503in}{5.789192in}}%
\pgfpathlineto{\pgfqpoint{4.513065in}{5.741832in}}%
\pgfpathlineto{\pgfqpoint{4.525628in}{5.699875in}}%
\pgfpathlineto{\pgfqpoint{4.538191in}{5.662425in}}%
\pgfpathlineto{\pgfqpoint{4.550754in}{5.628775in}}%
\pgfpathlineto{\pgfqpoint{4.575879in}{5.570699in}}%
\pgfpathlineto{\pgfqpoint{4.601005in}{5.522204in}}%
\pgfpathlineto{\pgfqpoint{4.626131in}{5.480923in}}%
\pgfpathlineto{\pgfqpoint{4.651256in}{5.445185in}}%
\pgfpathlineto{\pgfqpoint{4.676382in}{5.413795in}}%
\pgfpathlineto{\pgfqpoint{4.701508in}{5.385864in}}%
\pgfpathlineto{\pgfqpoint{4.726633in}{5.360726in}}%
\pgfpathlineto{\pgfqpoint{4.764322in}{5.327191in}}%
\pgfpathlineto{\pgfqpoint{4.802010in}{5.297611in}}%
\pgfpathlineto{\pgfqpoint{4.839698in}{5.271135in}}%
\pgfpathlineto{\pgfqpoint{4.877387in}{5.247155in}}%
\pgfpathlineto{\pgfqpoint{4.927638in}{5.218319in}}%
\pgfpathlineto{\pgfqpoint{4.977889in}{5.192396in}}%
\pgfpathlineto{\pgfqpoint{5.028141in}{5.168857in}}%
\pgfpathlineto{\pgfqpoint{5.090955in}{5.142211in}}%
\pgfpathlineto{\pgfqpoint{5.153769in}{5.118143in}}%
\pgfpathlineto{\pgfqpoint{5.216583in}{5.096246in}}%
\pgfpathlineto{\pgfqpoint{5.291960in}{5.072405in}}%
\pgfpathlineto{\pgfqpoint{5.367337in}{5.050832in}}%
\pgfpathlineto{\pgfqpoint{5.455276in}{5.028110in}}%
\pgfpathlineto{\pgfqpoint{5.543216in}{5.007649in}}%
\pgfpathlineto{\pgfqpoint{5.643719in}{4.986640in}}%
\pgfpathlineto{\pgfqpoint{5.744221in}{4.967811in}}%
\pgfpathlineto{\pgfqpoint{5.857286in}{4.948871in}}%
\pgfpathlineto{\pgfqpoint{5.982915in}{4.930221in}}%
\pgfpathlineto{\pgfqpoint{6.121106in}{4.912194in}}%
\pgfpathlineto{\pgfqpoint{6.259296in}{4.896395in}}%
\pgfpathlineto{\pgfqpoint{6.410050in}{4.881330in}}%
\pgfpathlineto{\pgfqpoint{6.573367in}{4.867192in}}%
\pgfpathlineto{\pgfqpoint{6.749246in}{4.854123in}}%
\pgfpathlineto{\pgfqpoint{6.900000in}{4.844440in}}%
\pgfpathlineto{\pgfqpoint{6.900000in}{4.844440in}}%
\pgfusepath{stroke}%
\end{pgfscope}%
\begin{pgfscope}%
\pgfpathrectangle{\pgfqpoint{4.100000in}{3.960000in}}{\pgfqpoint{3.100000in}{3.080000in}}%
\pgfusepath{clip}%
\pgfsetroundcap%
\pgfsetroundjoin%
\pgfsetlinewidth{1.204500pt}%
\definecolor{currentstroke}{rgb}{0.172549,0.627451,0.172549}%
\pgfsetstrokecolor{currentstroke}%
\pgfsetstrokeopacity{0.850000}%
\pgfsetdash{}{0pt}%
\pgfpathmoveto{\pgfqpoint{4.400000in}{6.707841in}}%
\pgfpathlineto{\pgfqpoint{4.412563in}{6.465188in}}%
\pgfpathlineto{\pgfqpoint{4.425126in}{6.306247in}}%
\pgfpathlineto{\pgfqpoint{4.437688in}{6.186698in}}%
\pgfpathlineto{\pgfqpoint{4.450251in}{6.091615in}}%
\pgfpathlineto{\pgfqpoint{4.462814in}{6.013814in}}%
\pgfpathlineto{\pgfqpoint{4.475377in}{5.948954in}}%
\pgfpathlineto{\pgfqpoint{4.487940in}{5.894109in}}%
\pgfpathlineto{\pgfqpoint{4.500503in}{5.847198in}}%
\pgfpathlineto{\pgfqpoint{4.513065in}{5.806682in}}%
\pgfpathlineto{\pgfqpoint{4.525628in}{5.771395in}}%
\pgfpathlineto{\pgfqpoint{4.538191in}{5.740431in}}%
\pgfpathlineto{\pgfqpoint{4.550754in}{5.713080in}}%
\pgfpathlineto{\pgfqpoint{4.563317in}{5.688776in}}%
\pgfpathlineto{\pgfqpoint{4.588442in}{5.647555in}}%
\pgfpathlineto{\pgfqpoint{4.613568in}{5.613989in}}%
\pgfpathlineto{\pgfqpoint{4.638693in}{5.586168in}}%
\pgfpathlineto{\pgfqpoint{4.663819in}{5.562743in}}%
\pgfpathlineto{\pgfqpoint{4.688945in}{5.542744in}}%
\pgfpathlineto{\pgfqpoint{4.714070in}{5.525463in}}%
\pgfpathlineto{\pgfqpoint{4.739196in}{5.510374in}}%
\pgfpathlineto{\pgfqpoint{4.776884in}{5.490998in}}%
\pgfpathlineto{\pgfqpoint{4.814573in}{5.474687in}}%
\pgfpathlineto{\pgfqpoint{4.852261in}{5.460765in}}%
\pgfpathlineto{\pgfqpoint{4.902513in}{5.445101in}}%
\pgfpathlineto{\pgfqpoint{4.952764in}{5.432032in}}%
\pgfpathlineto{\pgfqpoint{5.015578in}{5.418509in}}%
\pgfpathlineto{\pgfqpoint{5.078392in}{5.407421in}}%
\pgfpathlineto{\pgfqpoint{5.153769in}{5.396597in}}%
\pgfpathlineto{\pgfqpoint{5.241709in}{5.386596in}}%
\pgfpathlineto{\pgfqpoint{5.342211in}{5.377770in}}%
\pgfpathlineto{\pgfqpoint{5.455276in}{5.370293in}}%
\pgfpathlineto{\pgfqpoint{5.580905in}{5.364193in}}%
\pgfpathlineto{\pgfqpoint{5.744221in}{5.358685in}}%
\pgfpathlineto{\pgfqpoint{5.945226in}{5.354384in}}%
\pgfpathlineto{\pgfqpoint{6.209045in}{5.351220in}}%
\pgfpathlineto{\pgfqpoint{6.573367in}{5.349260in}}%
\pgfpathlineto{\pgfqpoint{6.900000in}{5.348610in}}%
\pgfpathlineto{\pgfqpoint{6.900000in}{5.348610in}}%
\pgfusepath{stroke}%
\end{pgfscope}%
\begin{pgfscope}%
\pgfsetrectcap%
\pgfsetmiterjoin%
\pgfsetlinewidth{0.803000pt}%
\definecolor{currentstroke}{rgb}{0.000000,0.000000,0.000000}%
\pgfsetstrokecolor{currentstroke}%
\pgfsetdash{}{0pt}%
\pgfpathmoveto{\pgfqpoint{4.100000in}{3.960000in}}%
\pgfpathlineto{\pgfqpoint{4.100000in}{7.040000in}}%
\pgfusepath{stroke}%
\end{pgfscope}%
\begin{pgfscope}%
\pgfsetrectcap%
\pgfsetmiterjoin%
\pgfsetlinewidth{0.803000pt}%
\definecolor{currentstroke}{rgb}{0.000000,0.000000,0.000000}%
\pgfsetstrokecolor{currentstroke}%
\pgfsetdash{}{0pt}%
\pgfpathmoveto{\pgfqpoint{7.200000in}{3.960000in}}%
\pgfpathlineto{\pgfqpoint{7.200000in}{7.040000in}}%
\pgfusepath{stroke}%
\end{pgfscope}%
\begin{pgfscope}%
\pgfsetrectcap%
\pgfsetmiterjoin%
\pgfsetlinewidth{0.803000pt}%
\definecolor{currentstroke}{rgb}{0.000000,0.000000,0.000000}%
\pgfsetstrokecolor{currentstroke}%
\pgfsetdash{}{0pt}%
\pgfpathmoveto{\pgfqpoint{4.100000in}{3.960000in}}%
\pgfpathlineto{\pgfqpoint{7.200000in}{3.960000in}}%
\pgfusepath{stroke}%
\end{pgfscope}%
\begin{pgfscope}%
\pgfsetrectcap%
\pgfsetmiterjoin%
\pgfsetlinewidth{0.803000pt}%
\definecolor{currentstroke}{rgb}{0.000000,0.000000,0.000000}%
\pgfsetstrokecolor{currentstroke}%
\pgfsetdash{}{0pt}%
\pgfpathmoveto{\pgfqpoint{4.100000in}{7.040000in}}%
\pgfpathlineto{\pgfqpoint{7.200000in}{7.040000in}}%
\pgfusepath{stroke}%
\end{pgfscope}%
\begin{pgfscope}%
\definecolor{textcolor}{rgb}{0.000000,0.000000,0.000000}%
\pgfsetstrokecolor{textcolor}%
\pgfsetfillcolor{textcolor}%
\pgftext[x=5.650000in,y=7.123333in,,base]{\color{textcolor}\rmfamily\fontsize{12.800000}{15.360000}\selectfont Test}%
\end{pgfscope}%
\begin{pgfscope}%
\pgfsetbuttcap%
\pgfsetmiterjoin%
\definecolor{currentfill}{rgb}{1.000000,1.000000,1.000000}%
\pgfsetfillcolor{currentfill}%
\pgfsetlinewidth{0.000000pt}%
\definecolor{currentstroke}{rgb}{0.000000,0.000000,0.000000}%
\pgfsetstrokecolor{currentstroke}%
\pgfsetstrokeopacity{0.000000}%
\pgfsetdash{}{0pt}%
\pgfpathmoveto{\pgfqpoint{1.000000in}{0.880000in}}%
\pgfpathlineto{\pgfqpoint{4.100000in}{0.880000in}}%
\pgfpathlineto{\pgfqpoint{4.100000in}{3.960000in}}%
\pgfpathlineto{\pgfqpoint{1.000000in}{3.960000in}}%
\pgfpathclose%
\pgfusepath{fill}%
\end{pgfscope}%
\begin{pgfscope}%
\pgfpathrectangle{\pgfqpoint{1.000000in}{0.880000in}}{\pgfqpoint{3.100000in}{3.080000in}}%
\pgfusepath{clip}%
\pgfsetbuttcap%
\pgfsetroundjoin%
\pgfsetlinewidth{0.501875pt}%
\definecolor{currentstroke}{rgb}{0.000000,0.000000,0.000000}%
\pgfsetstrokecolor{currentstroke}%
\pgfsetstrokeopacity{0.450000}%
\pgfsetdash{{2.500000pt}{5.000000pt}}{0.000000pt}%
\pgfpathmoveto{\pgfqpoint{1.300000in}{0.880000in}}%
\pgfpathlineto{\pgfqpoint{1.300000in}{3.960000in}}%
\pgfusepath{stroke}%
\end{pgfscope}%
\begin{pgfscope}%
\pgfsetbuttcap%
\pgfsetroundjoin%
\definecolor{currentfill}{rgb}{0.000000,0.000000,0.000000}%
\pgfsetfillcolor{currentfill}%
\pgfsetlinewidth{0.803000pt}%
\definecolor{currentstroke}{rgb}{0.000000,0.000000,0.000000}%
\pgfsetstrokecolor{currentstroke}%
\pgfsetdash{}{0pt}%
\pgfsys@defobject{currentmarker}{\pgfqpoint{0.000000in}{-0.048611in}}{\pgfqpoint{0.000000in}{0.000000in}}{%
\pgfpathmoveto{\pgfqpoint{0.000000in}{0.000000in}}%
\pgfpathlineto{\pgfqpoint{0.000000in}{-0.048611in}}%
\pgfusepath{stroke,fill}%
}%
\begin{pgfscope}%
\pgfsys@transformshift{1.300000in}{0.880000in}%
\pgfsys@useobject{currentmarker}{}%
\end{pgfscope}%
\end{pgfscope}%
\begin{pgfscope}%
\definecolor{textcolor}{rgb}{0.000000,0.000000,0.000000}%
\pgfsetstrokecolor{textcolor}%
\pgfsetfillcolor{textcolor}%
\pgftext[x=1.300000in,y=0.782778in,,top]{\color{textcolor}\rmfamily\fontsize{10.400000}{12.480000}\selectfont \(\displaystyle {0}\)}%
\end{pgfscope}%
\begin{pgfscope}%
\pgfpathrectangle{\pgfqpoint{1.000000in}{0.880000in}}{\pgfqpoint{3.100000in}{3.080000in}}%
\pgfusepath{clip}%
\pgfsetbuttcap%
\pgfsetroundjoin%
\pgfsetlinewidth{0.501875pt}%
\definecolor{currentstroke}{rgb}{0.000000,0.000000,0.000000}%
\pgfsetstrokecolor{currentstroke}%
\pgfsetstrokeopacity{0.450000}%
\pgfsetdash{{2.500000pt}{5.000000pt}}{0.000000pt}%
\pgfpathmoveto{\pgfqpoint{1.928141in}{0.880000in}}%
\pgfpathlineto{\pgfqpoint{1.928141in}{3.960000in}}%
\pgfusepath{stroke}%
\end{pgfscope}%
\begin{pgfscope}%
\pgfsetbuttcap%
\pgfsetroundjoin%
\definecolor{currentfill}{rgb}{0.000000,0.000000,0.000000}%
\pgfsetfillcolor{currentfill}%
\pgfsetlinewidth{0.803000pt}%
\definecolor{currentstroke}{rgb}{0.000000,0.000000,0.000000}%
\pgfsetstrokecolor{currentstroke}%
\pgfsetdash{}{0pt}%
\pgfsys@defobject{currentmarker}{\pgfqpoint{0.000000in}{-0.048611in}}{\pgfqpoint{0.000000in}{0.000000in}}{%
\pgfpathmoveto{\pgfqpoint{0.000000in}{0.000000in}}%
\pgfpathlineto{\pgfqpoint{0.000000in}{-0.048611in}}%
\pgfusepath{stroke,fill}%
}%
\begin{pgfscope}%
\pgfsys@transformshift{1.928141in}{0.880000in}%
\pgfsys@useobject{currentmarker}{}%
\end{pgfscope}%
\end{pgfscope}%
\begin{pgfscope}%
\definecolor{textcolor}{rgb}{0.000000,0.000000,0.000000}%
\pgfsetstrokecolor{textcolor}%
\pgfsetfillcolor{textcolor}%
\pgftext[x=1.928141in,y=0.782778in,,top]{\color{textcolor}\rmfamily\fontsize{10.400000}{12.480000}\selectfont \(\displaystyle {50}\)}%
\end{pgfscope}%
\begin{pgfscope}%
\pgfpathrectangle{\pgfqpoint{1.000000in}{0.880000in}}{\pgfqpoint{3.100000in}{3.080000in}}%
\pgfusepath{clip}%
\pgfsetbuttcap%
\pgfsetroundjoin%
\pgfsetlinewidth{0.501875pt}%
\definecolor{currentstroke}{rgb}{0.000000,0.000000,0.000000}%
\pgfsetstrokecolor{currentstroke}%
\pgfsetstrokeopacity{0.450000}%
\pgfsetdash{{2.500000pt}{5.000000pt}}{0.000000pt}%
\pgfpathmoveto{\pgfqpoint{2.556281in}{0.880000in}}%
\pgfpathlineto{\pgfqpoint{2.556281in}{3.960000in}}%
\pgfusepath{stroke}%
\end{pgfscope}%
\begin{pgfscope}%
\pgfsetbuttcap%
\pgfsetroundjoin%
\definecolor{currentfill}{rgb}{0.000000,0.000000,0.000000}%
\pgfsetfillcolor{currentfill}%
\pgfsetlinewidth{0.803000pt}%
\definecolor{currentstroke}{rgb}{0.000000,0.000000,0.000000}%
\pgfsetstrokecolor{currentstroke}%
\pgfsetdash{}{0pt}%
\pgfsys@defobject{currentmarker}{\pgfqpoint{0.000000in}{-0.048611in}}{\pgfqpoint{0.000000in}{0.000000in}}{%
\pgfpathmoveto{\pgfqpoint{0.000000in}{0.000000in}}%
\pgfpathlineto{\pgfqpoint{0.000000in}{-0.048611in}}%
\pgfusepath{stroke,fill}%
}%
\begin{pgfscope}%
\pgfsys@transformshift{2.556281in}{0.880000in}%
\pgfsys@useobject{currentmarker}{}%
\end{pgfscope}%
\end{pgfscope}%
\begin{pgfscope}%
\definecolor{textcolor}{rgb}{0.000000,0.000000,0.000000}%
\pgfsetstrokecolor{textcolor}%
\pgfsetfillcolor{textcolor}%
\pgftext[x=2.556281in,y=0.782778in,,top]{\color{textcolor}\rmfamily\fontsize{10.400000}{12.480000}\selectfont \(\displaystyle {100}\)}%
\end{pgfscope}%
\begin{pgfscope}%
\pgfpathrectangle{\pgfqpoint{1.000000in}{0.880000in}}{\pgfqpoint{3.100000in}{3.080000in}}%
\pgfusepath{clip}%
\pgfsetbuttcap%
\pgfsetroundjoin%
\pgfsetlinewidth{0.501875pt}%
\definecolor{currentstroke}{rgb}{0.000000,0.000000,0.000000}%
\pgfsetstrokecolor{currentstroke}%
\pgfsetstrokeopacity{0.450000}%
\pgfsetdash{{2.500000pt}{5.000000pt}}{0.000000pt}%
\pgfpathmoveto{\pgfqpoint{3.184422in}{0.880000in}}%
\pgfpathlineto{\pgfqpoint{3.184422in}{3.960000in}}%
\pgfusepath{stroke}%
\end{pgfscope}%
\begin{pgfscope}%
\pgfsetbuttcap%
\pgfsetroundjoin%
\definecolor{currentfill}{rgb}{0.000000,0.000000,0.000000}%
\pgfsetfillcolor{currentfill}%
\pgfsetlinewidth{0.803000pt}%
\definecolor{currentstroke}{rgb}{0.000000,0.000000,0.000000}%
\pgfsetstrokecolor{currentstroke}%
\pgfsetdash{}{0pt}%
\pgfsys@defobject{currentmarker}{\pgfqpoint{0.000000in}{-0.048611in}}{\pgfqpoint{0.000000in}{0.000000in}}{%
\pgfpathmoveto{\pgfqpoint{0.000000in}{0.000000in}}%
\pgfpathlineto{\pgfqpoint{0.000000in}{-0.048611in}}%
\pgfusepath{stroke,fill}%
}%
\begin{pgfscope}%
\pgfsys@transformshift{3.184422in}{0.880000in}%
\pgfsys@useobject{currentmarker}{}%
\end{pgfscope}%
\end{pgfscope}%
\begin{pgfscope}%
\definecolor{textcolor}{rgb}{0.000000,0.000000,0.000000}%
\pgfsetstrokecolor{textcolor}%
\pgfsetfillcolor{textcolor}%
\pgftext[x=3.184422in,y=0.782778in,,top]{\color{textcolor}\rmfamily\fontsize{10.400000}{12.480000}\selectfont \(\displaystyle {150}\)}%
\end{pgfscope}%
\begin{pgfscope}%
\pgfpathrectangle{\pgfqpoint{1.000000in}{0.880000in}}{\pgfqpoint{3.100000in}{3.080000in}}%
\pgfusepath{clip}%
\pgfsetbuttcap%
\pgfsetroundjoin%
\pgfsetlinewidth{0.501875pt}%
\definecolor{currentstroke}{rgb}{0.000000,0.000000,0.000000}%
\pgfsetstrokecolor{currentstroke}%
\pgfsetstrokeopacity{0.450000}%
\pgfsetdash{{2.500000pt}{5.000000pt}}{0.000000pt}%
\pgfpathmoveto{\pgfqpoint{3.812563in}{0.880000in}}%
\pgfpathlineto{\pgfqpoint{3.812563in}{3.960000in}}%
\pgfusepath{stroke}%
\end{pgfscope}%
\begin{pgfscope}%
\pgfsetbuttcap%
\pgfsetroundjoin%
\definecolor{currentfill}{rgb}{0.000000,0.000000,0.000000}%
\pgfsetfillcolor{currentfill}%
\pgfsetlinewidth{0.803000pt}%
\definecolor{currentstroke}{rgb}{0.000000,0.000000,0.000000}%
\pgfsetstrokecolor{currentstroke}%
\pgfsetdash{}{0pt}%
\pgfsys@defobject{currentmarker}{\pgfqpoint{0.000000in}{-0.048611in}}{\pgfqpoint{0.000000in}{0.000000in}}{%
\pgfpathmoveto{\pgfqpoint{0.000000in}{0.000000in}}%
\pgfpathlineto{\pgfqpoint{0.000000in}{-0.048611in}}%
\pgfusepath{stroke,fill}%
}%
\begin{pgfscope}%
\pgfsys@transformshift{3.812563in}{0.880000in}%
\pgfsys@useobject{currentmarker}{}%
\end{pgfscope}%
\end{pgfscope}%
\begin{pgfscope}%
\definecolor{textcolor}{rgb}{0.000000,0.000000,0.000000}%
\pgfsetstrokecolor{textcolor}%
\pgfsetfillcolor{textcolor}%
\pgftext[x=3.812563in,y=0.782778in,,top]{\color{textcolor}\rmfamily\fontsize{10.400000}{12.480000}\selectfont \(\displaystyle {200}\)}%
\end{pgfscope}%
\begin{pgfscope}%
\definecolor{textcolor}{rgb}{0.000000,0.000000,0.000000}%
\pgfsetstrokecolor{textcolor}%
\pgfsetfillcolor{textcolor}%
\pgftext[x=2.550000in,y=0.603766in,,top]{\color{textcolor}\rmfamily\fontsize{12.800000}{15.360000}\selectfont epoch}%
\end{pgfscope}%
\begin{pgfscope}%
\pgfpathrectangle{\pgfqpoint{1.000000in}{0.880000in}}{\pgfqpoint{3.100000in}{3.080000in}}%
\pgfusepath{clip}%
\pgfsetbuttcap%
\pgfsetroundjoin%
\pgfsetlinewidth{0.501875pt}%
\definecolor{currentstroke}{rgb}{0.000000,0.000000,0.000000}%
\pgfsetstrokecolor{currentstroke}%
\pgfsetstrokeopacity{0.450000}%
\pgfsetdash{{2.500000pt}{5.000000pt}}{0.000000pt}%
\pgfpathmoveto{\pgfqpoint{1.000000in}{1.016314in}}%
\pgfpathlineto{\pgfqpoint{4.100000in}{1.016314in}}%
\pgfusepath{stroke}%
\end{pgfscope}%
\begin{pgfscope}%
\pgfsetbuttcap%
\pgfsetroundjoin%
\definecolor{currentfill}{rgb}{0.000000,0.000000,0.000000}%
\pgfsetfillcolor{currentfill}%
\pgfsetlinewidth{0.803000pt}%
\definecolor{currentstroke}{rgb}{0.000000,0.000000,0.000000}%
\pgfsetstrokecolor{currentstroke}%
\pgfsetdash{}{0pt}%
\pgfsys@defobject{currentmarker}{\pgfqpoint{-0.048611in}{0.000000in}}{\pgfqpoint{-0.000000in}{0.000000in}}{%
\pgfpathmoveto{\pgfqpoint{-0.000000in}{0.000000in}}%
\pgfpathlineto{\pgfqpoint{-0.048611in}{0.000000in}}%
\pgfusepath{stroke,fill}%
}%
\begin{pgfscope}%
\pgfsys@transformshift{1.000000in}{1.016314in}%
\pgfsys@useobject{currentmarker}{}%
\end{pgfscope}%
\end{pgfscope}%
\begin{pgfscope}%
\definecolor{textcolor}{rgb}{0.000000,0.000000,0.000000}%
\pgfsetstrokecolor{textcolor}%
\pgfsetfillcolor{textcolor}%
\pgftext[x=0.586419in, y=0.968088in, left, base]{\color{textcolor}\rmfamily\fontsize{10.400000}{12.480000}\selectfont \(\displaystyle {0.450}\)}%
\end{pgfscope}%
\begin{pgfscope}%
\pgfpathrectangle{\pgfqpoint{1.000000in}{0.880000in}}{\pgfqpoint{3.100000in}{3.080000in}}%
\pgfusepath{clip}%
\pgfsetbuttcap%
\pgfsetroundjoin%
\pgfsetlinewidth{0.501875pt}%
\definecolor{currentstroke}{rgb}{0.000000,0.000000,0.000000}%
\pgfsetstrokecolor{currentstroke}%
\pgfsetstrokeopacity{0.450000}%
\pgfsetdash{{2.500000pt}{5.000000pt}}{0.000000pt}%
\pgfpathmoveto{\pgfqpoint{1.000000in}{1.385608in}}%
\pgfpathlineto{\pgfqpoint{4.100000in}{1.385608in}}%
\pgfusepath{stroke}%
\end{pgfscope}%
\begin{pgfscope}%
\pgfsetbuttcap%
\pgfsetroundjoin%
\definecolor{currentfill}{rgb}{0.000000,0.000000,0.000000}%
\pgfsetfillcolor{currentfill}%
\pgfsetlinewidth{0.803000pt}%
\definecolor{currentstroke}{rgb}{0.000000,0.000000,0.000000}%
\pgfsetstrokecolor{currentstroke}%
\pgfsetdash{}{0pt}%
\pgfsys@defobject{currentmarker}{\pgfqpoint{-0.048611in}{0.000000in}}{\pgfqpoint{-0.000000in}{0.000000in}}{%
\pgfpathmoveto{\pgfqpoint{-0.000000in}{0.000000in}}%
\pgfpathlineto{\pgfqpoint{-0.048611in}{0.000000in}}%
\pgfusepath{stroke,fill}%
}%
\begin{pgfscope}%
\pgfsys@transformshift{1.000000in}{1.385608in}%
\pgfsys@useobject{currentmarker}{}%
\end{pgfscope}%
\end{pgfscope}%
\begin{pgfscope}%
\definecolor{textcolor}{rgb}{0.000000,0.000000,0.000000}%
\pgfsetstrokecolor{textcolor}%
\pgfsetfillcolor{textcolor}%
\pgftext[x=0.586419in, y=1.337382in, left, base]{\color{textcolor}\rmfamily\fontsize{10.400000}{12.480000}\selectfont \(\displaystyle {0.475}\)}%
\end{pgfscope}%
\begin{pgfscope}%
\pgfpathrectangle{\pgfqpoint{1.000000in}{0.880000in}}{\pgfqpoint{3.100000in}{3.080000in}}%
\pgfusepath{clip}%
\pgfsetbuttcap%
\pgfsetroundjoin%
\pgfsetlinewidth{0.501875pt}%
\definecolor{currentstroke}{rgb}{0.000000,0.000000,0.000000}%
\pgfsetstrokecolor{currentstroke}%
\pgfsetstrokeopacity{0.450000}%
\pgfsetdash{{2.500000pt}{5.000000pt}}{0.000000pt}%
\pgfpathmoveto{\pgfqpoint{1.000000in}{1.754902in}}%
\pgfpathlineto{\pgfqpoint{4.100000in}{1.754902in}}%
\pgfusepath{stroke}%
\end{pgfscope}%
\begin{pgfscope}%
\pgfsetbuttcap%
\pgfsetroundjoin%
\definecolor{currentfill}{rgb}{0.000000,0.000000,0.000000}%
\pgfsetfillcolor{currentfill}%
\pgfsetlinewidth{0.803000pt}%
\definecolor{currentstroke}{rgb}{0.000000,0.000000,0.000000}%
\pgfsetstrokecolor{currentstroke}%
\pgfsetdash{}{0pt}%
\pgfsys@defobject{currentmarker}{\pgfqpoint{-0.048611in}{0.000000in}}{\pgfqpoint{-0.000000in}{0.000000in}}{%
\pgfpathmoveto{\pgfqpoint{-0.000000in}{0.000000in}}%
\pgfpathlineto{\pgfqpoint{-0.048611in}{0.000000in}}%
\pgfusepath{stroke,fill}%
}%
\begin{pgfscope}%
\pgfsys@transformshift{1.000000in}{1.754902in}%
\pgfsys@useobject{currentmarker}{}%
\end{pgfscope}%
\end{pgfscope}%
\begin{pgfscope}%
\definecolor{textcolor}{rgb}{0.000000,0.000000,0.000000}%
\pgfsetstrokecolor{textcolor}%
\pgfsetfillcolor{textcolor}%
\pgftext[x=0.586419in, y=1.706676in, left, base]{\color{textcolor}\rmfamily\fontsize{10.400000}{12.480000}\selectfont \(\displaystyle {0.500}\)}%
\end{pgfscope}%
\begin{pgfscope}%
\pgfpathrectangle{\pgfqpoint{1.000000in}{0.880000in}}{\pgfqpoint{3.100000in}{3.080000in}}%
\pgfusepath{clip}%
\pgfsetbuttcap%
\pgfsetroundjoin%
\pgfsetlinewidth{0.501875pt}%
\definecolor{currentstroke}{rgb}{0.000000,0.000000,0.000000}%
\pgfsetstrokecolor{currentstroke}%
\pgfsetstrokeopacity{0.450000}%
\pgfsetdash{{2.500000pt}{5.000000pt}}{0.000000pt}%
\pgfpathmoveto{\pgfqpoint{1.000000in}{2.124196in}}%
\pgfpathlineto{\pgfqpoint{4.100000in}{2.124196in}}%
\pgfusepath{stroke}%
\end{pgfscope}%
\begin{pgfscope}%
\pgfsetbuttcap%
\pgfsetroundjoin%
\definecolor{currentfill}{rgb}{0.000000,0.000000,0.000000}%
\pgfsetfillcolor{currentfill}%
\pgfsetlinewidth{0.803000pt}%
\definecolor{currentstroke}{rgb}{0.000000,0.000000,0.000000}%
\pgfsetstrokecolor{currentstroke}%
\pgfsetdash{}{0pt}%
\pgfsys@defobject{currentmarker}{\pgfqpoint{-0.048611in}{0.000000in}}{\pgfqpoint{-0.000000in}{0.000000in}}{%
\pgfpathmoveto{\pgfqpoint{-0.000000in}{0.000000in}}%
\pgfpathlineto{\pgfqpoint{-0.048611in}{0.000000in}}%
\pgfusepath{stroke,fill}%
}%
\begin{pgfscope}%
\pgfsys@transformshift{1.000000in}{2.124196in}%
\pgfsys@useobject{currentmarker}{}%
\end{pgfscope}%
\end{pgfscope}%
\begin{pgfscope}%
\definecolor{textcolor}{rgb}{0.000000,0.000000,0.000000}%
\pgfsetstrokecolor{textcolor}%
\pgfsetfillcolor{textcolor}%
\pgftext[x=0.586419in, y=2.075970in, left, base]{\color{textcolor}\rmfamily\fontsize{10.400000}{12.480000}\selectfont \(\displaystyle {0.525}\)}%
\end{pgfscope}%
\begin{pgfscope}%
\pgfpathrectangle{\pgfqpoint{1.000000in}{0.880000in}}{\pgfqpoint{3.100000in}{3.080000in}}%
\pgfusepath{clip}%
\pgfsetbuttcap%
\pgfsetroundjoin%
\pgfsetlinewidth{0.501875pt}%
\definecolor{currentstroke}{rgb}{0.000000,0.000000,0.000000}%
\pgfsetstrokecolor{currentstroke}%
\pgfsetstrokeopacity{0.450000}%
\pgfsetdash{{2.500000pt}{5.000000pt}}{0.000000pt}%
\pgfpathmoveto{\pgfqpoint{1.000000in}{2.493489in}}%
\pgfpathlineto{\pgfqpoint{4.100000in}{2.493489in}}%
\pgfusepath{stroke}%
\end{pgfscope}%
\begin{pgfscope}%
\pgfsetbuttcap%
\pgfsetroundjoin%
\definecolor{currentfill}{rgb}{0.000000,0.000000,0.000000}%
\pgfsetfillcolor{currentfill}%
\pgfsetlinewidth{0.803000pt}%
\definecolor{currentstroke}{rgb}{0.000000,0.000000,0.000000}%
\pgfsetstrokecolor{currentstroke}%
\pgfsetdash{}{0pt}%
\pgfsys@defobject{currentmarker}{\pgfqpoint{-0.048611in}{0.000000in}}{\pgfqpoint{-0.000000in}{0.000000in}}{%
\pgfpathmoveto{\pgfqpoint{-0.000000in}{0.000000in}}%
\pgfpathlineto{\pgfqpoint{-0.048611in}{0.000000in}}%
\pgfusepath{stroke,fill}%
}%
\begin{pgfscope}%
\pgfsys@transformshift{1.000000in}{2.493489in}%
\pgfsys@useobject{currentmarker}{}%
\end{pgfscope}%
\end{pgfscope}%
\begin{pgfscope}%
\definecolor{textcolor}{rgb}{0.000000,0.000000,0.000000}%
\pgfsetstrokecolor{textcolor}%
\pgfsetfillcolor{textcolor}%
\pgftext[x=0.586419in, y=2.445264in, left, base]{\color{textcolor}\rmfamily\fontsize{10.400000}{12.480000}\selectfont \(\displaystyle {0.550}\)}%
\end{pgfscope}%
\begin{pgfscope}%
\pgfpathrectangle{\pgfqpoint{1.000000in}{0.880000in}}{\pgfqpoint{3.100000in}{3.080000in}}%
\pgfusepath{clip}%
\pgfsetbuttcap%
\pgfsetroundjoin%
\pgfsetlinewidth{0.501875pt}%
\definecolor{currentstroke}{rgb}{0.000000,0.000000,0.000000}%
\pgfsetstrokecolor{currentstroke}%
\pgfsetstrokeopacity{0.450000}%
\pgfsetdash{{2.500000pt}{5.000000pt}}{0.000000pt}%
\pgfpathmoveto{\pgfqpoint{1.000000in}{2.862783in}}%
\pgfpathlineto{\pgfqpoint{4.100000in}{2.862783in}}%
\pgfusepath{stroke}%
\end{pgfscope}%
\begin{pgfscope}%
\pgfsetbuttcap%
\pgfsetroundjoin%
\definecolor{currentfill}{rgb}{0.000000,0.000000,0.000000}%
\pgfsetfillcolor{currentfill}%
\pgfsetlinewidth{0.803000pt}%
\definecolor{currentstroke}{rgb}{0.000000,0.000000,0.000000}%
\pgfsetstrokecolor{currentstroke}%
\pgfsetdash{}{0pt}%
\pgfsys@defobject{currentmarker}{\pgfqpoint{-0.048611in}{0.000000in}}{\pgfqpoint{-0.000000in}{0.000000in}}{%
\pgfpathmoveto{\pgfqpoint{-0.000000in}{0.000000in}}%
\pgfpathlineto{\pgfqpoint{-0.048611in}{0.000000in}}%
\pgfusepath{stroke,fill}%
}%
\begin{pgfscope}%
\pgfsys@transformshift{1.000000in}{2.862783in}%
\pgfsys@useobject{currentmarker}{}%
\end{pgfscope}%
\end{pgfscope}%
\begin{pgfscope}%
\definecolor{textcolor}{rgb}{0.000000,0.000000,0.000000}%
\pgfsetstrokecolor{textcolor}%
\pgfsetfillcolor{textcolor}%
\pgftext[x=0.586419in, y=2.814558in, left, base]{\color{textcolor}\rmfamily\fontsize{10.400000}{12.480000}\selectfont \(\displaystyle {0.575}\)}%
\end{pgfscope}%
\begin{pgfscope}%
\pgfpathrectangle{\pgfqpoint{1.000000in}{0.880000in}}{\pgfqpoint{3.100000in}{3.080000in}}%
\pgfusepath{clip}%
\pgfsetbuttcap%
\pgfsetroundjoin%
\pgfsetlinewidth{0.501875pt}%
\definecolor{currentstroke}{rgb}{0.000000,0.000000,0.000000}%
\pgfsetstrokecolor{currentstroke}%
\pgfsetstrokeopacity{0.450000}%
\pgfsetdash{{2.500000pt}{5.000000pt}}{0.000000pt}%
\pgfpathmoveto{\pgfqpoint{1.000000in}{3.232077in}}%
\pgfpathlineto{\pgfqpoint{4.100000in}{3.232077in}}%
\pgfusepath{stroke}%
\end{pgfscope}%
\begin{pgfscope}%
\pgfsetbuttcap%
\pgfsetroundjoin%
\definecolor{currentfill}{rgb}{0.000000,0.000000,0.000000}%
\pgfsetfillcolor{currentfill}%
\pgfsetlinewidth{0.803000pt}%
\definecolor{currentstroke}{rgb}{0.000000,0.000000,0.000000}%
\pgfsetstrokecolor{currentstroke}%
\pgfsetdash{}{0pt}%
\pgfsys@defobject{currentmarker}{\pgfqpoint{-0.048611in}{0.000000in}}{\pgfqpoint{-0.000000in}{0.000000in}}{%
\pgfpathmoveto{\pgfqpoint{-0.000000in}{0.000000in}}%
\pgfpathlineto{\pgfqpoint{-0.048611in}{0.000000in}}%
\pgfusepath{stroke,fill}%
}%
\begin{pgfscope}%
\pgfsys@transformshift{1.000000in}{3.232077in}%
\pgfsys@useobject{currentmarker}{}%
\end{pgfscope}%
\end{pgfscope}%
\begin{pgfscope}%
\definecolor{textcolor}{rgb}{0.000000,0.000000,0.000000}%
\pgfsetstrokecolor{textcolor}%
\pgfsetfillcolor{textcolor}%
\pgftext[x=0.586419in, y=3.183852in, left, base]{\color{textcolor}\rmfamily\fontsize{10.400000}{12.480000}\selectfont \(\displaystyle {0.600}\)}%
\end{pgfscope}%
\begin{pgfscope}%
\pgfpathrectangle{\pgfqpoint{1.000000in}{0.880000in}}{\pgfqpoint{3.100000in}{3.080000in}}%
\pgfusepath{clip}%
\pgfsetbuttcap%
\pgfsetroundjoin%
\pgfsetlinewidth{0.501875pt}%
\definecolor{currentstroke}{rgb}{0.000000,0.000000,0.000000}%
\pgfsetstrokecolor{currentstroke}%
\pgfsetstrokeopacity{0.450000}%
\pgfsetdash{{2.500000pt}{5.000000pt}}{0.000000pt}%
\pgfpathmoveto{\pgfqpoint{1.000000in}{3.601371in}}%
\pgfpathlineto{\pgfqpoint{4.100000in}{3.601371in}}%
\pgfusepath{stroke}%
\end{pgfscope}%
\begin{pgfscope}%
\pgfsetbuttcap%
\pgfsetroundjoin%
\definecolor{currentfill}{rgb}{0.000000,0.000000,0.000000}%
\pgfsetfillcolor{currentfill}%
\pgfsetlinewidth{0.803000pt}%
\definecolor{currentstroke}{rgb}{0.000000,0.000000,0.000000}%
\pgfsetstrokecolor{currentstroke}%
\pgfsetdash{}{0pt}%
\pgfsys@defobject{currentmarker}{\pgfqpoint{-0.048611in}{0.000000in}}{\pgfqpoint{-0.000000in}{0.000000in}}{%
\pgfpathmoveto{\pgfqpoint{-0.000000in}{0.000000in}}%
\pgfpathlineto{\pgfqpoint{-0.048611in}{0.000000in}}%
\pgfusepath{stroke,fill}%
}%
\begin{pgfscope}%
\pgfsys@transformshift{1.000000in}{3.601371in}%
\pgfsys@useobject{currentmarker}{}%
\end{pgfscope}%
\end{pgfscope}%
\begin{pgfscope}%
\definecolor{textcolor}{rgb}{0.000000,0.000000,0.000000}%
\pgfsetstrokecolor{textcolor}%
\pgfsetfillcolor{textcolor}%
\pgftext[x=0.586419in, y=3.553146in, left, base]{\color{textcolor}\rmfamily\fontsize{10.400000}{12.480000}\selectfont \(\displaystyle {0.625}\)}%
\end{pgfscope}%
\begin{pgfscope}%
\definecolor{textcolor}{rgb}{0.000000,0.000000,0.000000}%
\pgfsetstrokecolor{textcolor}%
\pgfsetfillcolor{textcolor}%
\pgftext[x=0.530863in,y=2.420000in,,bottom,rotate=90.000000]{\color{textcolor}\rmfamily\fontsize{12.800000}{15.360000}\selectfont accuracy}%
\end{pgfscope}%
\begin{pgfscope}%
\pgfpathrectangle{\pgfqpoint{1.000000in}{0.880000in}}{\pgfqpoint{3.100000in}{3.080000in}}%
\pgfusepath{clip}%
\pgfsetroundcap%
\pgfsetroundjoin%
\pgfsetlinewidth{1.204500pt}%
\definecolor{currentstroke}{rgb}{0.121569,0.466667,0.705882}%
\pgfsetstrokecolor{currentstroke}%
\pgfsetstrokeopacity{0.850000}%
\pgfsetdash{}{0pt}%
\pgfpathmoveto{\pgfqpoint{1.300000in}{1.178065in}}%
\pgfpathlineto{\pgfqpoint{1.312563in}{1.409243in}}%
\pgfpathlineto{\pgfqpoint{1.325126in}{1.605707in}}%
\pgfpathlineto{\pgfqpoint{1.337688in}{1.841316in}}%
\pgfpathlineto{\pgfqpoint{1.350251in}{2.019316in}}%
\pgfpathlineto{\pgfqpoint{1.375377in}{2.271175in}}%
\pgfpathlineto{\pgfqpoint{1.387940in}{2.377531in}}%
\pgfpathlineto{\pgfqpoint{1.450754in}{2.745348in}}%
\pgfpathlineto{\pgfqpoint{1.463317in}{2.788186in}}%
\pgfpathlineto{\pgfqpoint{1.475879in}{2.838410in}}%
\pgfpathlineto{\pgfqpoint{1.488442in}{2.878294in}}%
\pgfpathlineto{\pgfqpoint{1.501005in}{2.922609in}}%
\pgfpathlineto{\pgfqpoint{1.551256in}{3.026750in}}%
\pgfpathlineto{\pgfqpoint{1.563819in}{3.032659in}}%
\pgfpathlineto{\pgfqpoint{1.576382in}{3.042999in}}%
\pgfpathlineto{\pgfqpoint{1.588945in}{3.054816in}}%
\pgfpathlineto{\pgfqpoint{1.601508in}{3.084360in}}%
\pgfpathlineto{\pgfqpoint{1.614070in}{3.098393in}}%
\pgfpathlineto{\pgfqpoint{1.626633in}{3.103563in}}%
\pgfpathlineto{\pgfqpoint{1.639196in}{3.127198in}}%
\pgfpathlineto{\pgfqpoint{1.651759in}{3.141970in}}%
\pgfpathlineto{\pgfqpoint{1.664322in}{3.151571in}}%
\pgfpathlineto{\pgfqpoint{1.676884in}{3.167082in}}%
\pgfpathlineto{\pgfqpoint{1.689447in}{3.167082in}}%
\pgfpathlineto{\pgfqpoint{1.702010in}{3.183331in}}%
\pgfpathlineto{\pgfqpoint{1.714573in}{3.187762in}}%
\pgfpathlineto{\pgfqpoint{1.727136in}{3.190716in}}%
\pgfpathlineto{\pgfqpoint{1.752261in}{3.214351in}}%
\pgfpathlineto{\pgfqpoint{1.764824in}{3.223953in}}%
\pgfpathlineto{\pgfqpoint{1.777387in}{3.232077in}}%
\pgfpathlineto{\pgfqpoint{1.789950in}{3.241679in}}%
\pgfpathlineto{\pgfqpoint{1.802513in}{3.247588in}}%
\pgfpathlineto{\pgfqpoint{1.815075in}{3.252019in}}%
\pgfpathlineto{\pgfqpoint{1.827638in}{3.267530in}}%
\pgfpathlineto{\pgfqpoint{1.852764in}{3.272700in}}%
\pgfpathlineto{\pgfqpoint{1.865327in}{3.277870in}}%
\pgfpathlineto{\pgfqpoint{1.877889in}{3.292642in}}%
\pgfpathlineto{\pgfqpoint{1.890452in}{3.296334in}}%
\pgfpathlineto{\pgfqpoint{1.928141in}{3.314061in}}%
\pgfpathlineto{\pgfqpoint{1.940704in}{3.313322in}}%
\pgfpathlineto{\pgfqpoint{1.953266in}{3.318492in}}%
\pgfpathlineto{\pgfqpoint{1.965829in}{3.329571in}}%
\pgfpathlineto{\pgfqpoint{1.978392in}{3.330310in}}%
\pgfpathlineto{\pgfqpoint{1.990955in}{3.334741in}}%
\pgfpathlineto{\pgfqpoint{2.003518in}{3.343604in}}%
\pgfpathlineto{\pgfqpoint{2.016080in}{3.342866in}}%
\pgfpathlineto{\pgfqpoint{2.041206in}{3.348036in}}%
\pgfpathlineto{\pgfqpoint{2.066332in}{3.351729in}}%
\pgfpathlineto{\pgfqpoint{2.078894in}{3.357637in}}%
\pgfpathlineto{\pgfqpoint{2.091457in}{3.362069in}}%
\pgfpathlineto{\pgfqpoint{2.104020in}{3.370932in}}%
\pgfpathlineto{\pgfqpoint{2.141709in}{3.384965in}}%
\pgfpathlineto{\pgfqpoint{2.154271in}{3.385704in}}%
\pgfpathlineto{\pgfqpoint{2.179397in}{3.395305in}}%
\pgfpathlineto{\pgfqpoint{2.191960in}{3.396782in}}%
\pgfpathlineto{\pgfqpoint{2.204523in}{3.404907in}}%
\pgfpathlineto{\pgfqpoint{2.217085in}{3.405645in}}%
\pgfpathlineto{\pgfqpoint{2.229648in}{3.407861in}}%
\pgfpathlineto{\pgfqpoint{2.242211in}{3.412293in}}%
\pgfpathlineto{\pgfqpoint{2.254774in}{3.418940in}}%
\pgfpathlineto{\pgfqpoint{2.267337in}{3.419679in}}%
\pgfpathlineto{\pgfqpoint{2.279899in}{3.427803in}}%
\pgfpathlineto{\pgfqpoint{2.292462in}{3.426326in}}%
\pgfpathlineto{\pgfqpoint{2.305025in}{3.431496in}}%
\pgfpathlineto{\pgfqpoint{2.330151in}{3.432973in}}%
\pgfpathlineto{\pgfqpoint{2.342714in}{3.438143in}}%
\pgfpathlineto{\pgfqpoint{2.355276in}{3.446268in}}%
\pgfpathlineto{\pgfqpoint{2.367839in}{3.451438in}}%
\pgfpathlineto{\pgfqpoint{2.392965in}{3.455131in}}%
\pgfpathlineto{\pgfqpoint{2.418090in}{3.464733in}}%
\pgfpathlineto{\pgfqpoint{2.443216in}{3.474334in}}%
\pgfpathlineto{\pgfqpoint{2.455779in}{3.481720in}}%
\pgfpathlineto{\pgfqpoint{2.468342in}{3.482459in}}%
\pgfpathlineto{\pgfqpoint{2.480905in}{3.484674in}}%
\pgfpathlineto{\pgfqpoint{2.493467in}{3.489845in}}%
\pgfpathlineto{\pgfqpoint{2.506030in}{3.489845in}}%
\pgfpathlineto{\pgfqpoint{2.518593in}{3.495753in}}%
\pgfpathlineto{\pgfqpoint{2.531156in}{3.495753in}}%
\pgfpathlineto{\pgfqpoint{2.543719in}{3.497230in}}%
\pgfpathlineto{\pgfqpoint{2.556281in}{3.500923in}}%
\pgfpathlineto{\pgfqpoint{2.568844in}{3.501662in}}%
\pgfpathlineto{\pgfqpoint{2.581407in}{3.503878in}}%
\pgfpathlineto{\pgfqpoint{2.593970in}{3.503139in}}%
\pgfpathlineto{\pgfqpoint{2.606533in}{3.509786in}}%
\pgfpathlineto{\pgfqpoint{2.619095in}{3.511264in}}%
\pgfpathlineto{\pgfqpoint{2.631658in}{3.515695in}}%
\pgfpathlineto{\pgfqpoint{2.644221in}{3.516434in}}%
\pgfpathlineto{\pgfqpoint{2.656784in}{3.518649in}}%
\pgfpathlineto{\pgfqpoint{2.669347in}{3.519388in}}%
\pgfpathlineto{\pgfqpoint{2.694472in}{3.524558in}}%
\pgfpathlineto{\pgfqpoint{2.707035in}{3.523820in}}%
\pgfpathlineto{\pgfqpoint{2.719598in}{3.524558in}}%
\pgfpathlineto{\pgfqpoint{2.732161in}{3.523820in}}%
\pgfpathlineto{\pgfqpoint{2.794975in}{3.535637in}}%
\pgfpathlineto{\pgfqpoint{2.807538in}{3.538591in}}%
\pgfpathlineto{\pgfqpoint{2.820101in}{3.539330in}}%
\pgfpathlineto{\pgfqpoint{2.832663in}{3.543023in}}%
\pgfpathlineto{\pgfqpoint{2.845226in}{3.544500in}}%
\pgfpathlineto{\pgfqpoint{2.857789in}{3.551147in}}%
\pgfpathlineto{\pgfqpoint{2.870352in}{3.553363in}}%
\pgfpathlineto{\pgfqpoint{2.882915in}{3.559272in}}%
\pgfpathlineto{\pgfqpoint{2.895477in}{3.563703in}}%
\pgfpathlineto{\pgfqpoint{2.908040in}{3.566658in}}%
\pgfpathlineto{\pgfqpoint{2.920603in}{3.572566in}}%
\pgfpathlineto{\pgfqpoint{2.945729in}{3.579952in}}%
\pgfpathlineto{\pgfqpoint{2.958291in}{3.579214in}}%
\pgfpathlineto{\pgfqpoint{2.970854in}{3.581429in}}%
\pgfpathlineto{\pgfqpoint{2.995980in}{3.589554in}}%
\pgfpathlineto{\pgfqpoint{3.008543in}{3.593985in}}%
\pgfpathlineto{\pgfqpoint{3.021106in}{3.595463in}}%
\pgfpathlineto{\pgfqpoint{3.058794in}{3.592508in}}%
\pgfpathlineto{\pgfqpoint{3.071357in}{3.594724in}}%
\pgfpathlineto{\pgfqpoint{3.083920in}{3.593985in}}%
\pgfpathlineto{\pgfqpoint{3.109045in}{3.597678in}}%
\pgfpathlineto{\pgfqpoint{3.134171in}{3.602848in}}%
\pgfpathlineto{\pgfqpoint{3.146734in}{3.606541in}}%
\pgfpathlineto{\pgfqpoint{3.184422in}{3.613189in}}%
\pgfpathlineto{\pgfqpoint{3.209548in}{3.613927in}}%
\pgfpathlineto{\pgfqpoint{3.222111in}{3.619836in}}%
\pgfpathlineto{\pgfqpoint{3.247236in}{3.626483in}}%
\pgfpathlineto{\pgfqpoint{3.259799in}{3.626483in}}%
\pgfpathlineto{\pgfqpoint{3.272362in}{3.625006in}}%
\pgfpathlineto{\pgfqpoint{3.297487in}{3.625745in}}%
\pgfpathlineto{\pgfqpoint{3.322613in}{3.627222in}}%
\pgfpathlineto{\pgfqpoint{3.347739in}{3.633131in}}%
\pgfpathlineto{\pgfqpoint{3.360302in}{3.632392in}}%
\pgfpathlineto{\pgfqpoint{3.372864in}{3.633131in}}%
\pgfpathlineto{\pgfqpoint{3.385427in}{3.630176in}}%
\pgfpathlineto{\pgfqpoint{3.397990in}{3.634608in}}%
\pgfpathlineto{\pgfqpoint{3.410553in}{3.637562in}}%
\pgfpathlineto{\pgfqpoint{3.423116in}{3.638301in}}%
\pgfpathlineto{\pgfqpoint{3.435678in}{3.636085in}}%
\pgfpathlineto{\pgfqpoint{3.460804in}{3.638301in}}%
\pgfpathlineto{\pgfqpoint{3.473367in}{3.643471in}}%
\pgfpathlineto{\pgfqpoint{3.485930in}{3.641255in}}%
\pgfpathlineto{\pgfqpoint{3.523618in}{3.644209in}}%
\pgfpathlineto{\pgfqpoint{3.573869in}{3.649379in}}%
\pgfpathlineto{\pgfqpoint{3.586432in}{3.648641in}}%
\pgfpathlineto{\pgfqpoint{3.636683in}{3.650857in}}%
\pgfpathlineto{\pgfqpoint{3.649246in}{3.650857in}}%
\pgfpathlineto{\pgfqpoint{3.661809in}{3.656027in}}%
\pgfpathlineto{\pgfqpoint{3.674372in}{3.655288in}}%
\pgfpathlineto{\pgfqpoint{3.686935in}{3.658981in}}%
\pgfpathlineto{\pgfqpoint{3.699497in}{3.661197in}}%
\pgfpathlineto{\pgfqpoint{3.724623in}{3.658243in}}%
\pgfpathlineto{\pgfqpoint{3.737186in}{3.659720in}}%
\pgfpathlineto{\pgfqpoint{3.749749in}{3.658243in}}%
\pgfpathlineto{\pgfqpoint{3.762312in}{3.658981in}}%
\pgfpathlineto{\pgfqpoint{3.774874in}{3.661935in}}%
\pgfpathlineto{\pgfqpoint{3.787437in}{3.659720in}}%
\pgfpathlineto{\pgfqpoint{3.800000in}{3.660458in}}%
\pgfpathlineto{\pgfqpoint{3.800000in}{3.660458in}}%
\pgfusepath{stroke}%
\end{pgfscope}%
\begin{pgfscope}%
\pgfpathrectangle{\pgfqpoint{1.000000in}{0.880000in}}{\pgfqpoint{3.100000in}{3.080000in}}%
\pgfusepath{clip}%
\pgfsetroundcap%
\pgfsetroundjoin%
\pgfsetlinewidth{1.204500pt}%
\definecolor{currentstroke}{rgb}{1.000000,0.498039,0.054902}%
\pgfsetstrokecolor{currentstroke}%
\pgfsetstrokeopacity{0.850000}%
\pgfsetdash{}{0pt}%
\pgfpathmoveto{\pgfqpoint{1.300000in}{1.178065in}}%
\pgfpathlineto{\pgfqpoint{1.312563in}{1.408504in}}%
\pgfpathlineto{\pgfqpoint{1.325126in}{1.605707in}}%
\pgfpathlineto{\pgfqpoint{1.337688in}{1.840578in}}%
\pgfpathlineto{\pgfqpoint{1.350251in}{2.017100in}}%
\pgfpathlineto{\pgfqpoint{1.375377in}{2.267482in}}%
\pgfpathlineto{\pgfqpoint{1.387940in}{2.367191in}}%
\pgfpathlineto{\pgfqpoint{1.438191in}{2.664842in}}%
\pgfpathlineto{\pgfqpoint{1.450754in}{2.729099in}}%
\pgfpathlineto{\pgfqpoint{1.463317in}{2.779323in}}%
\pgfpathlineto{\pgfqpoint{1.488442in}{2.867954in}}%
\pgfpathlineto{\pgfqpoint{1.501005in}{2.898236in}}%
\pgfpathlineto{\pgfqpoint{1.513568in}{2.932211in}}%
\pgfpathlineto{\pgfqpoint{1.538693in}{2.981696in}}%
\pgfpathlineto{\pgfqpoint{1.551256in}{3.017148in}}%
\pgfpathlineto{\pgfqpoint{1.563819in}{3.026750in}}%
\pgfpathlineto{\pgfqpoint{1.576382in}{3.030443in}}%
\pgfpathlineto{\pgfqpoint{1.588945in}{3.045215in}}%
\pgfpathlineto{\pgfqpoint{1.601508in}{3.057771in}}%
\pgfpathlineto{\pgfqpoint{1.614070in}{3.082144in}}%
\pgfpathlineto{\pgfqpoint{1.626633in}{3.096916in}}%
\pgfpathlineto{\pgfqpoint{1.639196in}{3.105040in}}%
\pgfpathlineto{\pgfqpoint{1.664322in}{3.133845in}}%
\pgfpathlineto{\pgfqpoint{1.676884in}{3.144924in}}%
\pgfpathlineto{\pgfqpoint{1.714573in}{3.168559in}}%
\pgfpathlineto{\pgfqpoint{1.739698in}{3.193671in}}%
\pgfpathlineto{\pgfqpoint{1.752261in}{3.202534in}}%
\pgfpathlineto{\pgfqpoint{1.764824in}{3.206965in}}%
\pgfpathlineto{\pgfqpoint{1.777387in}{3.218783in}}%
\pgfpathlineto{\pgfqpoint{1.802513in}{3.226907in}}%
\pgfpathlineto{\pgfqpoint{1.815075in}{3.233555in}}%
\pgfpathlineto{\pgfqpoint{1.827638in}{3.238725in}}%
\pgfpathlineto{\pgfqpoint{1.840201in}{3.252758in}}%
\pgfpathlineto{\pgfqpoint{1.852764in}{3.259405in}}%
\pgfpathlineto{\pgfqpoint{1.865327in}{3.263837in}}%
\pgfpathlineto{\pgfqpoint{1.877889in}{3.266791in}}%
\pgfpathlineto{\pgfqpoint{1.890452in}{3.272700in}}%
\pgfpathlineto{\pgfqpoint{1.903015in}{3.275654in}}%
\pgfpathlineto{\pgfqpoint{1.928141in}{3.285994in}}%
\pgfpathlineto{\pgfqpoint{1.953266in}{3.292642in}}%
\pgfpathlineto{\pgfqpoint{1.965829in}{3.300766in}}%
\pgfpathlineto{\pgfqpoint{1.978392in}{3.302243in}}%
\pgfpathlineto{\pgfqpoint{1.990955in}{3.309629in}}%
\pgfpathlineto{\pgfqpoint{2.016080in}{3.316276in}}%
\pgfpathlineto{\pgfqpoint{2.028643in}{3.322185in}}%
\pgfpathlineto{\pgfqpoint{2.041206in}{3.325139in}}%
\pgfpathlineto{\pgfqpoint{2.066332in}{3.324401in}}%
\pgfpathlineto{\pgfqpoint{2.078894in}{3.328832in}}%
\pgfpathlineto{\pgfqpoint{2.091457in}{3.336218in}}%
\pgfpathlineto{\pgfqpoint{2.116583in}{3.343604in}}%
\pgfpathlineto{\pgfqpoint{2.154271in}{3.348774in}}%
\pgfpathlineto{\pgfqpoint{2.191960in}{3.362807in}}%
\pgfpathlineto{\pgfqpoint{2.204523in}{3.367978in}}%
\pgfpathlineto{\pgfqpoint{2.242211in}{3.374625in}}%
\pgfpathlineto{\pgfqpoint{2.267337in}{3.382749in}}%
\pgfpathlineto{\pgfqpoint{2.317588in}{3.386442in}}%
\pgfpathlineto{\pgfqpoint{2.342714in}{3.390135in}}%
\pgfpathlineto{\pgfqpoint{2.355276in}{3.389397in}}%
\pgfpathlineto{\pgfqpoint{2.380402in}{3.392351in}}%
\pgfpathlineto{\pgfqpoint{2.405528in}{3.401953in}}%
\pgfpathlineto{\pgfqpoint{2.430653in}{3.407861in}}%
\pgfpathlineto{\pgfqpoint{2.443216in}{3.416724in}}%
\pgfpathlineto{\pgfqpoint{2.455779in}{3.418940in}}%
\pgfpathlineto{\pgfqpoint{2.468342in}{3.423372in}}%
\pgfpathlineto{\pgfqpoint{2.493467in}{3.426326in}}%
\pgfpathlineto{\pgfqpoint{2.506030in}{3.430019in}}%
\pgfpathlineto{\pgfqpoint{2.518593in}{3.428542in}}%
\pgfpathlineto{\pgfqpoint{2.543719in}{3.432235in}}%
\pgfpathlineto{\pgfqpoint{2.568844in}{3.442575in}}%
\pgfpathlineto{\pgfqpoint{2.581407in}{3.444052in}}%
\pgfpathlineto{\pgfqpoint{2.593970in}{3.448484in}}%
\pgfpathlineto{\pgfqpoint{2.606533in}{3.455869in}}%
\pgfpathlineto{\pgfqpoint{2.631658in}{3.461040in}}%
\pgfpathlineto{\pgfqpoint{2.656784in}{3.463255in}}%
\pgfpathlineto{\pgfqpoint{2.707035in}{3.472857in}}%
\pgfpathlineto{\pgfqpoint{2.744724in}{3.482459in}}%
\pgfpathlineto{\pgfqpoint{2.757286in}{3.480981in}}%
\pgfpathlineto{\pgfqpoint{2.782412in}{3.485413in}}%
\pgfpathlineto{\pgfqpoint{2.794975in}{3.484674in}}%
\pgfpathlineto{\pgfqpoint{2.807538in}{3.485413in}}%
\pgfpathlineto{\pgfqpoint{2.820101in}{3.483936in}}%
\pgfpathlineto{\pgfqpoint{2.832663in}{3.484674in}}%
\pgfpathlineto{\pgfqpoint{2.845226in}{3.487629in}}%
\pgfpathlineto{\pgfqpoint{2.857789in}{3.487629in}}%
\pgfpathlineto{\pgfqpoint{2.870352in}{3.485413in}}%
\pgfpathlineto{\pgfqpoint{2.882915in}{3.485413in}}%
\pgfpathlineto{\pgfqpoint{2.908040in}{3.493537in}}%
\pgfpathlineto{\pgfqpoint{2.945729in}{3.493537in}}%
\pgfpathlineto{\pgfqpoint{2.970854in}{3.499446in}}%
\pgfpathlineto{\pgfqpoint{2.983417in}{3.500923in}}%
\pgfpathlineto{\pgfqpoint{2.995980in}{3.503878in}}%
\pgfpathlineto{\pgfqpoint{3.008543in}{3.503878in}}%
\pgfpathlineto{\pgfqpoint{3.033668in}{3.509786in}}%
\pgfpathlineto{\pgfqpoint{3.046231in}{3.510525in}}%
\pgfpathlineto{\pgfqpoint{3.058794in}{3.509786in}}%
\pgfpathlineto{\pgfqpoint{3.083920in}{3.517911in}}%
\pgfpathlineto{\pgfqpoint{3.109045in}{3.517911in}}%
\pgfpathlineto{\pgfqpoint{3.121608in}{3.520865in}}%
\pgfpathlineto{\pgfqpoint{3.146734in}{3.520127in}}%
\pgfpathlineto{\pgfqpoint{3.171859in}{3.521604in}}%
\pgfpathlineto{\pgfqpoint{3.184422in}{3.525297in}}%
\pgfpathlineto{\pgfqpoint{3.196985in}{3.526774in}}%
\pgfpathlineto{\pgfqpoint{3.209548in}{3.526774in}}%
\pgfpathlineto{\pgfqpoint{3.222111in}{3.531205in}}%
\pgfpathlineto{\pgfqpoint{3.234673in}{3.533421in}}%
\pgfpathlineto{\pgfqpoint{3.247236in}{3.531944in}}%
\pgfpathlineto{\pgfqpoint{3.272362in}{3.532683in}}%
\pgfpathlineto{\pgfqpoint{3.284925in}{3.536376in}}%
\pgfpathlineto{\pgfqpoint{3.297487in}{3.534898in}}%
\pgfpathlineto{\pgfqpoint{3.322613in}{3.535637in}}%
\pgfpathlineto{\pgfqpoint{3.335176in}{3.534160in}}%
\pgfpathlineto{\pgfqpoint{3.347739in}{3.535637in}}%
\pgfpathlineto{\pgfqpoint{3.360302in}{3.535637in}}%
\pgfpathlineto{\pgfqpoint{3.372864in}{3.538591in}}%
\pgfpathlineto{\pgfqpoint{3.385427in}{3.538591in}}%
\pgfpathlineto{\pgfqpoint{3.397990in}{3.540068in}}%
\pgfpathlineto{\pgfqpoint{3.423116in}{3.539330in}}%
\pgfpathlineto{\pgfqpoint{3.435678in}{3.539330in}}%
\pgfpathlineto{\pgfqpoint{3.448241in}{3.543761in}}%
\pgfpathlineto{\pgfqpoint{3.485930in}{3.546716in}}%
\pgfpathlineto{\pgfqpoint{3.498492in}{3.551147in}}%
\pgfpathlineto{\pgfqpoint{3.523618in}{3.553363in}}%
\pgfpathlineto{\pgfqpoint{3.536181in}{3.552624in}}%
\pgfpathlineto{\pgfqpoint{3.548744in}{3.555579in}}%
\pgfpathlineto{\pgfqpoint{3.611558in}{3.556317in}}%
\pgfpathlineto{\pgfqpoint{3.624121in}{3.558533in}}%
\pgfpathlineto{\pgfqpoint{3.674372in}{3.561488in}}%
\pgfpathlineto{\pgfqpoint{3.699497in}{3.565180in}}%
\pgfpathlineto{\pgfqpoint{3.712060in}{3.566658in}}%
\pgfpathlineto{\pgfqpoint{3.724623in}{3.565180in}}%
\pgfpathlineto{\pgfqpoint{3.762312in}{3.568135in}}%
\pgfpathlineto{\pgfqpoint{3.774874in}{3.570351in}}%
\pgfpathlineto{\pgfqpoint{3.787437in}{3.570351in}}%
\pgfpathlineto{\pgfqpoint{3.800000in}{3.573305in}}%
\pgfpathlineto{\pgfqpoint{3.800000in}{3.573305in}}%
\pgfusepath{stroke}%
\end{pgfscope}%
\begin{pgfscope}%
\pgfpathrectangle{\pgfqpoint{1.000000in}{0.880000in}}{\pgfqpoint{3.100000in}{3.080000in}}%
\pgfusepath{clip}%
\pgfsetroundcap%
\pgfsetroundjoin%
\pgfsetlinewidth{1.204500pt}%
\definecolor{currentstroke}{rgb}{0.172549,0.627451,0.172549}%
\pgfsetstrokecolor{currentstroke}%
\pgfsetstrokeopacity{0.850000}%
\pgfsetdash{}{0pt}%
\pgfpathmoveto{\pgfqpoint{1.300000in}{1.178065in}}%
\pgfpathlineto{\pgfqpoint{1.312563in}{1.408504in}}%
\pgfpathlineto{\pgfqpoint{1.325126in}{1.602014in}}%
\pgfpathlineto{\pgfqpoint{1.337688in}{1.833931in}}%
\pgfpathlineto{\pgfqpoint{1.350251in}{2.007499in}}%
\pgfpathlineto{\pgfqpoint{1.375377in}{2.260834in}}%
\pgfpathlineto{\pgfqpoint{1.387940in}{2.353158in}}%
\pgfpathlineto{\pgfqpoint{1.425628in}{2.577689in}}%
\pgfpathlineto{\pgfqpoint{1.438191in}{2.631605in}}%
\pgfpathlineto{\pgfqpoint{1.450754in}{2.695124in}}%
\pgfpathlineto{\pgfqpoint{1.475879in}{2.792618in}}%
\pgfpathlineto{\pgfqpoint{1.488442in}{2.825115in}}%
\pgfpathlineto{\pgfqpoint{1.501005in}{2.869431in}}%
\pgfpathlineto{\pgfqpoint{1.513568in}{2.899713in}}%
\pgfpathlineto{\pgfqpoint{1.526131in}{2.924086in}}%
\pgfpathlineto{\pgfqpoint{1.538693in}{2.957323in}}%
\pgfpathlineto{\pgfqpoint{1.563819in}{3.000161in}}%
\pgfpathlineto{\pgfqpoint{1.576382in}{3.020103in}}%
\pgfpathlineto{\pgfqpoint{1.588945in}{3.034136in}}%
\pgfpathlineto{\pgfqpoint{1.601508in}{3.034874in}}%
\pgfpathlineto{\pgfqpoint{1.614070in}{3.038567in}}%
\pgfpathlineto{\pgfqpoint{1.626633in}{3.044476in}}%
\pgfpathlineto{\pgfqpoint{1.639196in}{3.053339in}}%
\pgfpathlineto{\pgfqpoint{1.664322in}{3.081405in}}%
\pgfpathlineto{\pgfqpoint{1.702010in}{3.113903in}}%
\pgfpathlineto{\pgfqpoint{1.714573in}{3.123505in}}%
\pgfpathlineto{\pgfqpoint{1.727136in}{3.131629in}}%
\pgfpathlineto{\pgfqpoint{1.739698in}{3.144185in}}%
\pgfpathlineto{\pgfqpoint{1.764824in}{3.157480in}}%
\pgfpathlineto{\pgfqpoint{1.777387in}{3.161173in}}%
\pgfpathlineto{\pgfqpoint{1.789950in}{3.167820in}}%
\pgfpathlineto{\pgfqpoint{1.802513in}{3.178899in}}%
\pgfpathlineto{\pgfqpoint{1.815075in}{3.187762in}}%
\pgfpathlineto{\pgfqpoint{1.840201in}{3.198841in}}%
\pgfpathlineto{\pgfqpoint{1.852764in}{3.201795in}}%
\pgfpathlineto{\pgfqpoint{1.865327in}{3.203272in}}%
\pgfpathlineto{\pgfqpoint{1.877889in}{3.212874in}}%
\pgfpathlineto{\pgfqpoint{1.890452in}{3.217306in}}%
\pgfpathlineto{\pgfqpoint{1.903015in}{3.223953in}}%
\pgfpathlineto{\pgfqpoint{1.915578in}{3.228384in}}%
\pgfpathlineto{\pgfqpoint{1.928141in}{3.225430in}}%
\pgfpathlineto{\pgfqpoint{1.940704in}{3.231339in}}%
\pgfpathlineto{\pgfqpoint{1.965829in}{3.240940in}}%
\pgfpathlineto{\pgfqpoint{1.978392in}{3.246849in}}%
\pgfpathlineto{\pgfqpoint{1.990955in}{3.250542in}}%
\pgfpathlineto{\pgfqpoint{2.003518in}{3.255712in}}%
\pgfpathlineto{\pgfqpoint{2.016080in}{3.258667in}}%
\pgfpathlineto{\pgfqpoint{2.028643in}{3.254974in}}%
\pgfpathlineto{\pgfqpoint{2.041206in}{3.253496in}}%
\pgfpathlineto{\pgfqpoint{2.066332in}{3.253496in}}%
\pgfpathlineto{\pgfqpoint{2.078894in}{3.257189in}}%
\pgfpathlineto{\pgfqpoint{2.091457in}{3.264575in}}%
\pgfpathlineto{\pgfqpoint{2.104020in}{3.268268in}}%
\pgfpathlineto{\pgfqpoint{2.116583in}{3.269007in}}%
\pgfpathlineto{\pgfqpoint{2.154271in}{3.279347in}}%
\pgfpathlineto{\pgfqpoint{2.166834in}{3.282301in}}%
\pgfpathlineto{\pgfqpoint{2.191960in}{3.285256in}}%
\pgfpathlineto{\pgfqpoint{2.204523in}{3.288949in}}%
\pgfpathlineto{\pgfqpoint{2.217085in}{3.287471in}}%
\pgfpathlineto{\pgfqpoint{2.242211in}{3.291164in}}%
\pgfpathlineto{\pgfqpoint{2.254774in}{3.293380in}}%
\pgfpathlineto{\pgfqpoint{2.267337in}{3.292642in}}%
\pgfpathlineto{\pgfqpoint{2.292462in}{3.294857in}}%
\pgfpathlineto{\pgfqpoint{2.305025in}{3.299289in}}%
\pgfpathlineto{\pgfqpoint{2.330151in}{3.304459in}}%
\pgfpathlineto{\pgfqpoint{2.355276in}{3.308890in}}%
\pgfpathlineto{\pgfqpoint{2.380402in}{3.308890in}}%
\pgfpathlineto{\pgfqpoint{2.392965in}{3.311845in}}%
\pgfpathlineto{\pgfqpoint{2.405528in}{3.313322in}}%
\pgfpathlineto{\pgfqpoint{2.418090in}{3.311845in}}%
\pgfpathlineto{\pgfqpoint{2.430653in}{3.313322in}}%
\pgfpathlineto{\pgfqpoint{2.443216in}{3.316276in}}%
\pgfpathlineto{\pgfqpoint{2.468342in}{3.316276in}}%
\pgfpathlineto{\pgfqpoint{2.506030in}{3.320708in}}%
\pgfpathlineto{\pgfqpoint{2.518593in}{3.325139in}}%
\pgfpathlineto{\pgfqpoint{2.531156in}{3.325878in}}%
\pgfpathlineto{\pgfqpoint{2.543719in}{3.325139in}}%
\pgfpathlineto{\pgfqpoint{2.556281in}{3.326617in}}%
\pgfpathlineto{\pgfqpoint{2.568844in}{3.325878in}}%
\pgfpathlineto{\pgfqpoint{2.593970in}{3.328094in}}%
\pgfpathlineto{\pgfqpoint{2.631658in}{3.331787in}}%
\pgfpathlineto{\pgfqpoint{2.669347in}{3.332525in}}%
\pgfpathlineto{\pgfqpoint{2.681910in}{3.335480in}}%
\pgfpathlineto{\pgfqpoint{2.694472in}{3.339911in}}%
\pgfpathlineto{\pgfqpoint{2.719598in}{3.343604in}}%
\pgfpathlineto{\pgfqpoint{2.732161in}{3.345081in}}%
\pgfpathlineto{\pgfqpoint{2.744724in}{3.343604in}}%
\pgfpathlineto{\pgfqpoint{2.757286in}{3.343604in}}%
\pgfpathlineto{\pgfqpoint{2.769849in}{3.345820in}}%
\pgfpathlineto{\pgfqpoint{2.794975in}{3.345820in}}%
\pgfpathlineto{\pgfqpoint{2.820101in}{3.348036in}}%
\pgfpathlineto{\pgfqpoint{2.845226in}{3.347297in}}%
\pgfpathlineto{\pgfqpoint{2.857789in}{3.348774in}}%
\pgfpathlineto{\pgfqpoint{2.870352in}{3.348036in}}%
\pgfpathlineto{\pgfqpoint{2.882915in}{3.349513in}}%
\pgfpathlineto{\pgfqpoint{2.933166in}{3.349513in}}%
\pgfpathlineto{\pgfqpoint{2.958291in}{3.351729in}}%
\pgfpathlineto{\pgfqpoint{2.995980in}{3.352467in}}%
\pgfpathlineto{\pgfqpoint{3.008543in}{3.353944in}}%
\pgfpathlineto{\pgfqpoint{3.021106in}{3.353206in}}%
\pgfpathlineto{\pgfqpoint{3.058794in}{3.356160in}}%
\pgfpathlineto{\pgfqpoint{3.096482in}{3.360592in}}%
\pgfpathlineto{\pgfqpoint{3.121608in}{3.359853in}}%
\pgfpathlineto{\pgfqpoint{3.134171in}{3.359853in}}%
\pgfpathlineto{\pgfqpoint{3.146734in}{3.361330in}}%
\pgfpathlineto{\pgfqpoint{3.171859in}{3.362069in}}%
\pgfpathlineto{\pgfqpoint{3.196985in}{3.364285in}}%
\pgfpathlineto{\pgfqpoint{3.209548in}{3.364285in}}%
\pgfpathlineto{\pgfqpoint{3.234673in}{3.367978in}}%
\pgfpathlineto{\pgfqpoint{3.272362in}{3.370932in}}%
\pgfpathlineto{\pgfqpoint{3.297487in}{3.373148in}}%
\pgfpathlineto{\pgfqpoint{3.347739in}{3.372409in}}%
\pgfpathlineto{\pgfqpoint{3.360302in}{3.374625in}}%
\pgfpathlineto{\pgfqpoint{3.385427in}{3.375363in}}%
\pgfpathlineto{\pgfqpoint{3.448241in}{3.374625in}}%
\pgfpathlineto{\pgfqpoint{3.473367in}{3.373148in}}%
\pgfpathlineto{\pgfqpoint{3.523618in}{3.375363in}}%
\pgfpathlineto{\pgfqpoint{3.548744in}{3.375363in}}%
\pgfpathlineto{\pgfqpoint{3.573869in}{3.377579in}}%
\pgfpathlineto{\pgfqpoint{3.674372in}{3.378318in}}%
\pgfpathlineto{\pgfqpoint{3.686935in}{3.379056in}}%
\pgfpathlineto{\pgfqpoint{3.699497in}{3.378318in}}%
\pgfpathlineto{\pgfqpoint{3.737186in}{3.379795in}}%
\pgfpathlineto{\pgfqpoint{3.762312in}{3.379056in}}%
\pgfpathlineto{\pgfqpoint{3.774874in}{3.378318in}}%
\pgfpathlineto{\pgfqpoint{3.787437in}{3.379795in}}%
\pgfpathlineto{\pgfqpoint{3.800000in}{3.379795in}}%
\pgfpathlineto{\pgfqpoint{3.800000in}{3.379795in}}%
\pgfusepath{stroke}%
\end{pgfscope}%
\begin{pgfscope}%
\pgfsetrectcap%
\pgfsetmiterjoin%
\pgfsetlinewidth{0.803000pt}%
\definecolor{currentstroke}{rgb}{0.000000,0.000000,0.000000}%
\pgfsetstrokecolor{currentstroke}%
\pgfsetdash{}{0pt}%
\pgfpathmoveto{\pgfqpoint{1.000000in}{0.880000in}}%
\pgfpathlineto{\pgfqpoint{1.000000in}{3.960000in}}%
\pgfusepath{stroke}%
\end{pgfscope}%
\begin{pgfscope}%
\pgfsetrectcap%
\pgfsetmiterjoin%
\pgfsetlinewidth{0.803000pt}%
\definecolor{currentstroke}{rgb}{0.000000,0.000000,0.000000}%
\pgfsetstrokecolor{currentstroke}%
\pgfsetdash{}{0pt}%
\pgfpathmoveto{\pgfqpoint{4.100000in}{0.880000in}}%
\pgfpathlineto{\pgfqpoint{4.100000in}{3.960000in}}%
\pgfusepath{stroke}%
\end{pgfscope}%
\begin{pgfscope}%
\pgfsetrectcap%
\pgfsetmiterjoin%
\pgfsetlinewidth{0.803000pt}%
\definecolor{currentstroke}{rgb}{0.000000,0.000000,0.000000}%
\pgfsetstrokecolor{currentstroke}%
\pgfsetdash{}{0pt}%
\pgfpathmoveto{\pgfqpoint{1.000000in}{0.880000in}}%
\pgfpathlineto{\pgfqpoint{4.100000in}{0.880000in}}%
\pgfusepath{stroke}%
\end{pgfscope}%
\begin{pgfscope}%
\pgfsetrectcap%
\pgfsetmiterjoin%
\pgfsetlinewidth{0.803000pt}%
\definecolor{currentstroke}{rgb}{0.000000,0.000000,0.000000}%
\pgfsetstrokecolor{currentstroke}%
\pgfsetdash{}{0pt}%
\pgfpathmoveto{\pgfqpoint{1.000000in}{3.960000in}}%
\pgfpathlineto{\pgfqpoint{4.100000in}{3.960000in}}%
\pgfusepath{stroke}%
\end{pgfscope}%
\begin{pgfscope}%
\pgfsetbuttcap%
\pgfsetmiterjoin%
\definecolor{currentfill}{rgb}{1.000000,1.000000,1.000000}%
\pgfsetfillcolor{currentfill}%
\pgfsetlinewidth{0.000000pt}%
\definecolor{currentstroke}{rgb}{0.000000,0.000000,0.000000}%
\pgfsetstrokecolor{currentstroke}%
\pgfsetstrokeopacity{0.000000}%
\pgfsetdash{}{0pt}%
\pgfpathmoveto{\pgfqpoint{4.100000in}{0.880000in}}%
\pgfpathlineto{\pgfqpoint{7.200000in}{0.880000in}}%
\pgfpathlineto{\pgfqpoint{7.200000in}{3.960000in}}%
\pgfpathlineto{\pgfqpoint{4.100000in}{3.960000in}}%
\pgfpathclose%
\pgfusepath{fill}%
\end{pgfscope}%
\begin{pgfscope}%
\pgfpathrectangle{\pgfqpoint{4.100000in}{0.880000in}}{\pgfqpoint{3.100000in}{3.080000in}}%
\pgfusepath{clip}%
\pgfsetbuttcap%
\pgfsetroundjoin%
\pgfsetlinewidth{0.501875pt}%
\definecolor{currentstroke}{rgb}{0.000000,0.000000,0.000000}%
\pgfsetstrokecolor{currentstroke}%
\pgfsetstrokeopacity{0.450000}%
\pgfsetdash{{2.500000pt}{5.000000pt}}{0.000000pt}%
\pgfpathmoveto{\pgfqpoint{4.400000in}{0.880000in}}%
\pgfpathlineto{\pgfqpoint{4.400000in}{3.960000in}}%
\pgfusepath{stroke}%
\end{pgfscope}%
\begin{pgfscope}%
\pgfsetbuttcap%
\pgfsetroundjoin%
\definecolor{currentfill}{rgb}{0.000000,0.000000,0.000000}%
\pgfsetfillcolor{currentfill}%
\pgfsetlinewidth{0.803000pt}%
\definecolor{currentstroke}{rgb}{0.000000,0.000000,0.000000}%
\pgfsetstrokecolor{currentstroke}%
\pgfsetdash{}{0pt}%
\pgfsys@defobject{currentmarker}{\pgfqpoint{0.000000in}{-0.048611in}}{\pgfqpoint{0.000000in}{0.000000in}}{%
\pgfpathmoveto{\pgfqpoint{0.000000in}{0.000000in}}%
\pgfpathlineto{\pgfqpoint{0.000000in}{-0.048611in}}%
\pgfusepath{stroke,fill}%
}%
\begin{pgfscope}%
\pgfsys@transformshift{4.400000in}{0.880000in}%
\pgfsys@useobject{currentmarker}{}%
\end{pgfscope}%
\end{pgfscope}%
\begin{pgfscope}%
\definecolor{textcolor}{rgb}{0.000000,0.000000,0.000000}%
\pgfsetstrokecolor{textcolor}%
\pgfsetfillcolor{textcolor}%
\pgftext[x=4.400000in,y=0.782778in,,top]{\color{textcolor}\rmfamily\fontsize{10.400000}{12.480000}\selectfont \(\displaystyle {0}\)}%
\end{pgfscope}%
\begin{pgfscope}%
\pgfpathrectangle{\pgfqpoint{4.100000in}{0.880000in}}{\pgfqpoint{3.100000in}{3.080000in}}%
\pgfusepath{clip}%
\pgfsetbuttcap%
\pgfsetroundjoin%
\pgfsetlinewidth{0.501875pt}%
\definecolor{currentstroke}{rgb}{0.000000,0.000000,0.000000}%
\pgfsetstrokecolor{currentstroke}%
\pgfsetstrokeopacity{0.450000}%
\pgfsetdash{{2.500000pt}{5.000000pt}}{0.000000pt}%
\pgfpathmoveto{\pgfqpoint{5.028141in}{0.880000in}}%
\pgfpathlineto{\pgfqpoint{5.028141in}{3.960000in}}%
\pgfusepath{stroke}%
\end{pgfscope}%
\begin{pgfscope}%
\pgfsetbuttcap%
\pgfsetroundjoin%
\definecolor{currentfill}{rgb}{0.000000,0.000000,0.000000}%
\pgfsetfillcolor{currentfill}%
\pgfsetlinewidth{0.803000pt}%
\definecolor{currentstroke}{rgb}{0.000000,0.000000,0.000000}%
\pgfsetstrokecolor{currentstroke}%
\pgfsetdash{}{0pt}%
\pgfsys@defobject{currentmarker}{\pgfqpoint{0.000000in}{-0.048611in}}{\pgfqpoint{0.000000in}{0.000000in}}{%
\pgfpathmoveto{\pgfqpoint{0.000000in}{0.000000in}}%
\pgfpathlineto{\pgfqpoint{0.000000in}{-0.048611in}}%
\pgfusepath{stroke,fill}%
}%
\begin{pgfscope}%
\pgfsys@transformshift{5.028141in}{0.880000in}%
\pgfsys@useobject{currentmarker}{}%
\end{pgfscope}%
\end{pgfscope}%
\begin{pgfscope}%
\definecolor{textcolor}{rgb}{0.000000,0.000000,0.000000}%
\pgfsetstrokecolor{textcolor}%
\pgfsetfillcolor{textcolor}%
\pgftext[x=5.028141in,y=0.782778in,,top]{\color{textcolor}\rmfamily\fontsize{10.400000}{12.480000}\selectfont \(\displaystyle {50}\)}%
\end{pgfscope}%
\begin{pgfscope}%
\pgfpathrectangle{\pgfqpoint{4.100000in}{0.880000in}}{\pgfqpoint{3.100000in}{3.080000in}}%
\pgfusepath{clip}%
\pgfsetbuttcap%
\pgfsetroundjoin%
\pgfsetlinewidth{0.501875pt}%
\definecolor{currentstroke}{rgb}{0.000000,0.000000,0.000000}%
\pgfsetstrokecolor{currentstroke}%
\pgfsetstrokeopacity{0.450000}%
\pgfsetdash{{2.500000pt}{5.000000pt}}{0.000000pt}%
\pgfpathmoveto{\pgfqpoint{5.656281in}{0.880000in}}%
\pgfpathlineto{\pgfqpoint{5.656281in}{3.960000in}}%
\pgfusepath{stroke}%
\end{pgfscope}%
\begin{pgfscope}%
\pgfsetbuttcap%
\pgfsetroundjoin%
\definecolor{currentfill}{rgb}{0.000000,0.000000,0.000000}%
\pgfsetfillcolor{currentfill}%
\pgfsetlinewidth{0.803000pt}%
\definecolor{currentstroke}{rgb}{0.000000,0.000000,0.000000}%
\pgfsetstrokecolor{currentstroke}%
\pgfsetdash{}{0pt}%
\pgfsys@defobject{currentmarker}{\pgfqpoint{0.000000in}{-0.048611in}}{\pgfqpoint{0.000000in}{0.000000in}}{%
\pgfpathmoveto{\pgfqpoint{0.000000in}{0.000000in}}%
\pgfpathlineto{\pgfqpoint{0.000000in}{-0.048611in}}%
\pgfusepath{stroke,fill}%
}%
\begin{pgfscope}%
\pgfsys@transformshift{5.656281in}{0.880000in}%
\pgfsys@useobject{currentmarker}{}%
\end{pgfscope}%
\end{pgfscope}%
\begin{pgfscope}%
\definecolor{textcolor}{rgb}{0.000000,0.000000,0.000000}%
\pgfsetstrokecolor{textcolor}%
\pgfsetfillcolor{textcolor}%
\pgftext[x=5.656281in,y=0.782778in,,top]{\color{textcolor}\rmfamily\fontsize{10.400000}{12.480000}\selectfont \(\displaystyle {100}\)}%
\end{pgfscope}%
\begin{pgfscope}%
\pgfpathrectangle{\pgfqpoint{4.100000in}{0.880000in}}{\pgfqpoint{3.100000in}{3.080000in}}%
\pgfusepath{clip}%
\pgfsetbuttcap%
\pgfsetroundjoin%
\pgfsetlinewidth{0.501875pt}%
\definecolor{currentstroke}{rgb}{0.000000,0.000000,0.000000}%
\pgfsetstrokecolor{currentstroke}%
\pgfsetstrokeopacity{0.450000}%
\pgfsetdash{{2.500000pt}{5.000000pt}}{0.000000pt}%
\pgfpathmoveto{\pgfqpoint{6.284422in}{0.880000in}}%
\pgfpathlineto{\pgfqpoint{6.284422in}{3.960000in}}%
\pgfusepath{stroke}%
\end{pgfscope}%
\begin{pgfscope}%
\pgfsetbuttcap%
\pgfsetroundjoin%
\definecolor{currentfill}{rgb}{0.000000,0.000000,0.000000}%
\pgfsetfillcolor{currentfill}%
\pgfsetlinewidth{0.803000pt}%
\definecolor{currentstroke}{rgb}{0.000000,0.000000,0.000000}%
\pgfsetstrokecolor{currentstroke}%
\pgfsetdash{}{0pt}%
\pgfsys@defobject{currentmarker}{\pgfqpoint{0.000000in}{-0.048611in}}{\pgfqpoint{0.000000in}{0.000000in}}{%
\pgfpathmoveto{\pgfqpoint{0.000000in}{0.000000in}}%
\pgfpathlineto{\pgfqpoint{0.000000in}{-0.048611in}}%
\pgfusepath{stroke,fill}%
}%
\begin{pgfscope}%
\pgfsys@transformshift{6.284422in}{0.880000in}%
\pgfsys@useobject{currentmarker}{}%
\end{pgfscope}%
\end{pgfscope}%
\begin{pgfscope}%
\definecolor{textcolor}{rgb}{0.000000,0.000000,0.000000}%
\pgfsetstrokecolor{textcolor}%
\pgfsetfillcolor{textcolor}%
\pgftext[x=6.284422in,y=0.782778in,,top]{\color{textcolor}\rmfamily\fontsize{10.400000}{12.480000}\selectfont \(\displaystyle {150}\)}%
\end{pgfscope}%
\begin{pgfscope}%
\pgfpathrectangle{\pgfqpoint{4.100000in}{0.880000in}}{\pgfqpoint{3.100000in}{3.080000in}}%
\pgfusepath{clip}%
\pgfsetbuttcap%
\pgfsetroundjoin%
\pgfsetlinewidth{0.501875pt}%
\definecolor{currentstroke}{rgb}{0.000000,0.000000,0.000000}%
\pgfsetstrokecolor{currentstroke}%
\pgfsetstrokeopacity{0.450000}%
\pgfsetdash{{2.500000pt}{5.000000pt}}{0.000000pt}%
\pgfpathmoveto{\pgfqpoint{6.912563in}{0.880000in}}%
\pgfpathlineto{\pgfqpoint{6.912563in}{3.960000in}}%
\pgfusepath{stroke}%
\end{pgfscope}%
\begin{pgfscope}%
\pgfsetbuttcap%
\pgfsetroundjoin%
\definecolor{currentfill}{rgb}{0.000000,0.000000,0.000000}%
\pgfsetfillcolor{currentfill}%
\pgfsetlinewidth{0.803000pt}%
\definecolor{currentstroke}{rgb}{0.000000,0.000000,0.000000}%
\pgfsetstrokecolor{currentstroke}%
\pgfsetdash{}{0pt}%
\pgfsys@defobject{currentmarker}{\pgfqpoint{0.000000in}{-0.048611in}}{\pgfqpoint{0.000000in}{0.000000in}}{%
\pgfpathmoveto{\pgfqpoint{0.000000in}{0.000000in}}%
\pgfpathlineto{\pgfqpoint{0.000000in}{-0.048611in}}%
\pgfusepath{stroke,fill}%
}%
\begin{pgfscope}%
\pgfsys@transformshift{6.912563in}{0.880000in}%
\pgfsys@useobject{currentmarker}{}%
\end{pgfscope}%
\end{pgfscope}%
\begin{pgfscope}%
\definecolor{textcolor}{rgb}{0.000000,0.000000,0.000000}%
\pgfsetstrokecolor{textcolor}%
\pgfsetfillcolor{textcolor}%
\pgftext[x=6.912563in,y=0.782778in,,top]{\color{textcolor}\rmfamily\fontsize{10.400000}{12.480000}\selectfont \(\displaystyle {200}\)}%
\end{pgfscope}%
\begin{pgfscope}%
\definecolor{textcolor}{rgb}{0.000000,0.000000,0.000000}%
\pgfsetstrokecolor{textcolor}%
\pgfsetfillcolor{textcolor}%
\pgftext[x=5.650000in,y=0.603766in,,top]{\color{textcolor}\rmfamily\fontsize{12.800000}{15.360000}\selectfont epoch}%
\end{pgfscope}%
\begin{pgfscope}%
\pgfpathrectangle{\pgfqpoint{4.100000in}{0.880000in}}{\pgfqpoint{3.100000in}{3.080000in}}%
\pgfusepath{clip}%
\pgfsetbuttcap%
\pgfsetroundjoin%
\pgfsetlinewidth{0.501875pt}%
\definecolor{currentstroke}{rgb}{0.000000,0.000000,0.000000}%
\pgfsetstrokecolor{currentstroke}%
\pgfsetstrokeopacity{0.450000}%
\pgfsetdash{{2.500000pt}{5.000000pt}}{0.000000pt}%
\pgfpathmoveto{\pgfqpoint{4.100000in}{1.016314in}}%
\pgfpathlineto{\pgfqpoint{7.200000in}{1.016314in}}%
\pgfusepath{stroke}%
\end{pgfscope}%
\begin{pgfscope}%
\pgfsetbuttcap%
\pgfsetroundjoin%
\definecolor{currentfill}{rgb}{0.000000,0.000000,0.000000}%
\pgfsetfillcolor{currentfill}%
\pgfsetlinewidth{0.803000pt}%
\definecolor{currentstroke}{rgb}{0.000000,0.000000,0.000000}%
\pgfsetstrokecolor{currentstroke}%
\pgfsetdash{}{0pt}%
\pgfsys@defobject{currentmarker}{\pgfqpoint{-0.024306in}{-0.000000in}}{\pgfqpoint{0.024306in}{0.000000in}}{%
\pgfpathmoveto{\pgfqpoint{0.024306in}{-0.000000in}}%
\pgfpathlineto{\pgfqpoint{-0.024306in}{0.000000in}}%
\pgfusepath{stroke,fill}%
}%
\begin{pgfscope}%
\pgfsys@transformshift{4.100000in}{1.016314in}%
\pgfsys@useobject{currentmarker}{}%
\end{pgfscope}%
\end{pgfscope}%
\begin{pgfscope}%
\pgfpathrectangle{\pgfqpoint{4.100000in}{0.880000in}}{\pgfqpoint{3.100000in}{3.080000in}}%
\pgfusepath{clip}%
\pgfsetbuttcap%
\pgfsetroundjoin%
\pgfsetlinewidth{0.501875pt}%
\definecolor{currentstroke}{rgb}{0.000000,0.000000,0.000000}%
\pgfsetstrokecolor{currentstroke}%
\pgfsetstrokeopacity{0.450000}%
\pgfsetdash{{2.500000pt}{5.000000pt}}{0.000000pt}%
\pgfpathmoveto{\pgfqpoint{4.100000in}{1.385608in}}%
\pgfpathlineto{\pgfqpoint{7.200000in}{1.385608in}}%
\pgfusepath{stroke}%
\end{pgfscope}%
\begin{pgfscope}%
\pgfsetbuttcap%
\pgfsetroundjoin%
\definecolor{currentfill}{rgb}{0.000000,0.000000,0.000000}%
\pgfsetfillcolor{currentfill}%
\pgfsetlinewidth{0.803000pt}%
\definecolor{currentstroke}{rgb}{0.000000,0.000000,0.000000}%
\pgfsetstrokecolor{currentstroke}%
\pgfsetdash{}{0pt}%
\pgfsys@defobject{currentmarker}{\pgfqpoint{-0.024306in}{-0.000000in}}{\pgfqpoint{0.024306in}{0.000000in}}{%
\pgfpathmoveto{\pgfqpoint{0.024306in}{-0.000000in}}%
\pgfpathlineto{\pgfqpoint{-0.024306in}{0.000000in}}%
\pgfusepath{stroke,fill}%
}%
\begin{pgfscope}%
\pgfsys@transformshift{4.100000in}{1.385608in}%
\pgfsys@useobject{currentmarker}{}%
\end{pgfscope}%
\end{pgfscope}%
\begin{pgfscope}%
\pgfpathrectangle{\pgfqpoint{4.100000in}{0.880000in}}{\pgfqpoint{3.100000in}{3.080000in}}%
\pgfusepath{clip}%
\pgfsetbuttcap%
\pgfsetroundjoin%
\pgfsetlinewidth{0.501875pt}%
\definecolor{currentstroke}{rgb}{0.000000,0.000000,0.000000}%
\pgfsetstrokecolor{currentstroke}%
\pgfsetstrokeopacity{0.450000}%
\pgfsetdash{{2.500000pt}{5.000000pt}}{0.000000pt}%
\pgfpathmoveto{\pgfqpoint{4.100000in}{1.754902in}}%
\pgfpathlineto{\pgfqpoint{7.200000in}{1.754902in}}%
\pgfusepath{stroke}%
\end{pgfscope}%
\begin{pgfscope}%
\pgfsetbuttcap%
\pgfsetroundjoin%
\definecolor{currentfill}{rgb}{0.000000,0.000000,0.000000}%
\pgfsetfillcolor{currentfill}%
\pgfsetlinewidth{0.803000pt}%
\definecolor{currentstroke}{rgb}{0.000000,0.000000,0.000000}%
\pgfsetstrokecolor{currentstroke}%
\pgfsetdash{}{0pt}%
\pgfsys@defobject{currentmarker}{\pgfqpoint{-0.024306in}{-0.000000in}}{\pgfqpoint{0.024306in}{0.000000in}}{%
\pgfpathmoveto{\pgfqpoint{0.024306in}{-0.000000in}}%
\pgfpathlineto{\pgfqpoint{-0.024306in}{0.000000in}}%
\pgfusepath{stroke,fill}%
}%
\begin{pgfscope}%
\pgfsys@transformshift{4.100000in}{1.754902in}%
\pgfsys@useobject{currentmarker}{}%
\end{pgfscope}%
\end{pgfscope}%
\begin{pgfscope}%
\pgfpathrectangle{\pgfqpoint{4.100000in}{0.880000in}}{\pgfqpoint{3.100000in}{3.080000in}}%
\pgfusepath{clip}%
\pgfsetbuttcap%
\pgfsetroundjoin%
\pgfsetlinewidth{0.501875pt}%
\definecolor{currentstroke}{rgb}{0.000000,0.000000,0.000000}%
\pgfsetstrokecolor{currentstroke}%
\pgfsetstrokeopacity{0.450000}%
\pgfsetdash{{2.500000pt}{5.000000pt}}{0.000000pt}%
\pgfpathmoveto{\pgfqpoint{4.100000in}{2.124196in}}%
\pgfpathlineto{\pgfqpoint{7.200000in}{2.124196in}}%
\pgfusepath{stroke}%
\end{pgfscope}%
\begin{pgfscope}%
\pgfsetbuttcap%
\pgfsetroundjoin%
\definecolor{currentfill}{rgb}{0.000000,0.000000,0.000000}%
\pgfsetfillcolor{currentfill}%
\pgfsetlinewidth{0.803000pt}%
\definecolor{currentstroke}{rgb}{0.000000,0.000000,0.000000}%
\pgfsetstrokecolor{currentstroke}%
\pgfsetdash{}{0pt}%
\pgfsys@defobject{currentmarker}{\pgfqpoint{-0.024306in}{-0.000000in}}{\pgfqpoint{0.024306in}{0.000000in}}{%
\pgfpathmoveto{\pgfqpoint{0.024306in}{-0.000000in}}%
\pgfpathlineto{\pgfqpoint{-0.024306in}{0.000000in}}%
\pgfusepath{stroke,fill}%
}%
\begin{pgfscope}%
\pgfsys@transformshift{4.100000in}{2.124196in}%
\pgfsys@useobject{currentmarker}{}%
\end{pgfscope}%
\end{pgfscope}%
\begin{pgfscope}%
\pgfpathrectangle{\pgfqpoint{4.100000in}{0.880000in}}{\pgfqpoint{3.100000in}{3.080000in}}%
\pgfusepath{clip}%
\pgfsetbuttcap%
\pgfsetroundjoin%
\pgfsetlinewidth{0.501875pt}%
\definecolor{currentstroke}{rgb}{0.000000,0.000000,0.000000}%
\pgfsetstrokecolor{currentstroke}%
\pgfsetstrokeopacity{0.450000}%
\pgfsetdash{{2.500000pt}{5.000000pt}}{0.000000pt}%
\pgfpathmoveto{\pgfqpoint{4.100000in}{2.493489in}}%
\pgfpathlineto{\pgfqpoint{7.200000in}{2.493489in}}%
\pgfusepath{stroke}%
\end{pgfscope}%
\begin{pgfscope}%
\pgfsetbuttcap%
\pgfsetroundjoin%
\definecolor{currentfill}{rgb}{0.000000,0.000000,0.000000}%
\pgfsetfillcolor{currentfill}%
\pgfsetlinewidth{0.803000pt}%
\definecolor{currentstroke}{rgb}{0.000000,0.000000,0.000000}%
\pgfsetstrokecolor{currentstroke}%
\pgfsetdash{}{0pt}%
\pgfsys@defobject{currentmarker}{\pgfqpoint{-0.024306in}{-0.000000in}}{\pgfqpoint{0.024306in}{0.000000in}}{%
\pgfpathmoveto{\pgfqpoint{0.024306in}{-0.000000in}}%
\pgfpathlineto{\pgfqpoint{-0.024306in}{0.000000in}}%
\pgfusepath{stroke,fill}%
}%
\begin{pgfscope}%
\pgfsys@transformshift{4.100000in}{2.493489in}%
\pgfsys@useobject{currentmarker}{}%
\end{pgfscope}%
\end{pgfscope}%
\begin{pgfscope}%
\pgfpathrectangle{\pgfqpoint{4.100000in}{0.880000in}}{\pgfqpoint{3.100000in}{3.080000in}}%
\pgfusepath{clip}%
\pgfsetbuttcap%
\pgfsetroundjoin%
\pgfsetlinewidth{0.501875pt}%
\definecolor{currentstroke}{rgb}{0.000000,0.000000,0.000000}%
\pgfsetstrokecolor{currentstroke}%
\pgfsetstrokeopacity{0.450000}%
\pgfsetdash{{2.500000pt}{5.000000pt}}{0.000000pt}%
\pgfpathmoveto{\pgfqpoint{4.100000in}{2.862783in}}%
\pgfpathlineto{\pgfqpoint{7.200000in}{2.862783in}}%
\pgfusepath{stroke}%
\end{pgfscope}%
\begin{pgfscope}%
\pgfsetbuttcap%
\pgfsetroundjoin%
\definecolor{currentfill}{rgb}{0.000000,0.000000,0.000000}%
\pgfsetfillcolor{currentfill}%
\pgfsetlinewidth{0.803000pt}%
\definecolor{currentstroke}{rgb}{0.000000,0.000000,0.000000}%
\pgfsetstrokecolor{currentstroke}%
\pgfsetdash{}{0pt}%
\pgfsys@defobject{currentmarker}{\pgfqpoint{-0.024306in}{-0.000000in}}{\pgfqpoint{0.024306in}{0.000000in}}{%
\pgfpathmoveto{\pgfqpoint{0.024306in}{-0.000000in}}%
\pgfpathlineto{\pgfqpoint{-0.024306in}{0.000000in}}%
\pgfusepath{stroke,fill}%
}%
\begin{pgfscope}%
\pgfsys@transformshift{4.100000in}{2.862783in}%
\pgfsys@useobject{currentmarker}{}%
\end{pgfscope}%
\end{pgfscope}%
\begin{pgfscope}%
\pgfpathrectangle{\pgfqpoint{4.100000in}{0.880000in}}{\pgfqpoint{3.100000in}{3.080000in}}%
\pgfusepath{clip}%
\pgfsetbuttcap%
\pgfsetroundjoin%
\pgfsetlinewidth{0.501875pt}%
\definecolor{currentstroke}{rgb}{0.000000,0.000000,0.000000}%
\pgfsetstrokecolor{currentstroke}%
\pgfsetstrokeopacity{0.450000}%
\pgfsetdash{{2.500000pt}{5.000000pt}}{0.000000pt}%
\pgfpathmoveto{\pgfqpoint{4.100000in}{3.232077in}}%
\pgfpathlineto{\pgfqpoint{7.200000in}{3.232077in}}%
\pgfusepath{stroke}%
\end{pgfscope}%
\begin{pgfscope}%
\pgfsetbuttcap%
\pgfsetroundjoin%
\definecolor{currentfill}{rgb}{0.000000,0.000000,0.000000}%
\pgfsetfillcolor{currentfill}%
\pgfsetlinewidth{0.803000pt}%
\definecolor{currentstroke}{rgb}{0.000000,0.000000,0.000000}%
\pgfsetstrokecolor{currentstroke}%
\pgfsetdash{}{0pt}%
\pgfsys@defobject{currentmarker}{\pgfqpoint{-0.024306in}{-0.000000in}}{\pgfqpoint{0.024306in}{0.000000in}}{%
\pgfpathmoveto{\pgfqpoint{0.024306in}{-0.000000in}}%
\pgfpathlineto{\pgfqpoint{-0.024306in}{0.000000in}}%
\pgfusepath{stroke,fill}%
}%
\begin{pgfscope}%
\pgfsys@transformshift{4.100000in}{3.232077in}%
\pgfsys@useobject{currentmarker}{}%
\end{pgfscope}%
\end{pgfscope}%
\begin{pgfscope}%
\pgfpathrectangle{\pgfqpoint{4.100000in}{0.880000in}}{\pgfqpoint{3.100000in}{3.080000in}}%
\pgfusepath{clip}%
\pgfsetbuttcap%
\pgfsetroundjoin%
\pgfsetlinewidth{0.501875pt}%
\definecolor{currentstroke}{rgb}{0.000000,0.000000,0.000000}%
\pgfsetstrokecolor{currentstroke}%
\pgfsetstrokeopacity{0.450000}%
\pgfsetdash{{2.500000pt}{5.000000pt}}{0.000000pt}%
\pgfpathmoveto{\pgfqpoint{4.100000in}{3.601371in}}%
\pgfpathlineto{\pgfqpoint{7.200000in}{3.601371in}}%
\pgfusepath{stroke}%
\end{pgfscope}%
\begin{pgfscope}%
\pgfsetbuttcap%
\pgfsetroundjoin%
\definecolor{currentfill}{rgb}{0.000000,0.000000,0.000000}%
\pgfsetfillcolor{currentfill}%
\pgfsetlinewidth{0.803000pt}%
\definecolor{currentstroke}{rgb}{0.000000,0.000000,0.000000}%
\pgfsetstrokecolor{currentstroke}%
\pgfsetdash{}{0pt}%
\pgfsys@defobject{currentmarker}{\pgfqpoint{-0.024306in}{-0.000000in}}{\pgfqpoint{0.024306in}{0.000000in}}{%
\pgfpathmoveto{\pgfqpoint{0.024306in}{-0.000000in}}%
\pgfpathlineto{\pgfqpoint{-0.024306in}{0.000000in}}%
\pgfusepath{stroke,fill}%
}%
\begin{pgfscope}%
\pgfsys@transformshift{4.100000in}{3.601371in}%
\pgfsys@useobject{currentmarker}{}%
\end{pgfscope}%
\end{pgfscope}%
\begin{pgfscope}%
\pgfpathrectangle{\pgfqpoint{4.100000in}{0.880000in}}{\pgfqpoint{3.100000in}{3.080000in}}%
\pgfusepath{clip}%
\pgfsetroundcap%
\pgfsetroundjoin%
\pgfsetlinewidth{1.204500pt}%
\definecolor{currentstroke}{rgb}{0.121569,0.466667,0.705882}%
\pgfsetstrokecolor{currentstroke}%
\pgfsetstrokeopacity{0.850000}%
\pgfsetdash{}{0pt}%
\pgfpathmoveto{\pgfqpoint{4.400000in}{1.226811in}}%
\pgfpathlineto{\pgfqpoint{4.425126in}{1.655192in}}%
\pgfpathlineto{\pgfqpoint{4.437688in}{1.802910in}}%
\pgfpathlineto{\pgfqpoint{4.450251in}{1.972785in}}%
\pgfpathlineto{\pgfqpoint{4.462814in}{2.098345in}}%
\pgfpathlineto{\pgfqpoint{4.475377in}{2.179590in}}%
\pgfpathlineto{\pgfqpoint{4.487940in}{2.290378in}}%
\pgfpathlineto{\pgfqpoint{4.513065in}{2.475025in}}%
\pgfpathlineto{\pgfqpoint{4.525628in}{2.486104in}}%
\pgfpathlineto{\pgfqpoint{4.538191in}{2.530419in}}%
\pgfpathlineto{\pgfqpoint{4.550754in}{2.582120in}}%
\pgfpathlineto{\pgfqpoint{4.588442in}{2.707680in}}%
\pgfpathlineto{\pgfqpoint{4.601005in}{2.737223in}}%
\pgfpathlineto{\pgfqpoint{4.638693in}{2.770460in}}%
\pgfpathlineto{\pgfqpoint{4.651256in}{2.792618in}}%
\pgfpathlineto{\pgfqpoint{4.663819in}{2.848012in}}%
\pgfpathlineto{\pgfqpoint{4.676382in}{2.888634in}}%
\pgfpathlineto{\pgfqpoint{4.688945in}{2.892327in}}%
\pgfpathlineto{\pgfqpoint{4.701508in}{2.918178in}}%
\pgfpathlineto{\pgfqpoint{4.714070in}{2.940335in}}%
\pgfpathlineto{\pgfqpoint{4.726633in}{2.947721in}}%
\pgfpathlineto{\pgfqpoint{4.739196in}{2.966186in}}%
\pgfpathlineto{\pgfqpoint{4.764322in}{2.988343in}}%
\pgfpathlineto{\pgfqpoint{4.776884in}{3.003115in}}%
\pgfpathlineto{\pgfqpoint{4.789447in}{3.010501in}}%
\pgfpathlineto{\pgfqpoint{4.802010in}{3.032659in}}%
\pgfpathlineto{\pgfqpoint{4.814573in}{3.028966in}}%
\pgfpathlineto{\pgfqpoint{4.827136in}{3.032659in}}%
\pgfpathlineto{\pgfqpoint{4.839698in}{3.043737in}}%
\pgfpathlineto{\pgfqpoint{4.852261in}{3.043737in}}%
\pgfpathlineto{\pgfqpoint{4.864824in}{3.047430in}}%
\pgfpathlineto{\pgfqpoint{4.877387in}{3.047430in}}%
\pgfpathlineto{\pgfqpoint{4.889950in}{3.051123in}}%
\pgfpathlineto{\pgfqpoint{4.902513in}{3.073281in}}%
\pgfpathlineto{\pgfqpoint{4.915075in}{3.073281in}}%
\pgfpathlineto{\pgfqpoint{4.927638in}{3.091746in}}%
\pgfpathlineto{\pgfqpoint{4.952764in}{3.106517in}}%
\pgfpathlineto{\pgfqpoint{4.965327in}{3.095439in}}%
\pgfpathlineto{\pgfqpoint{4.977889in}{3.110210in}}%
\pgfpathlineto{\pgfqpoint{4.990452in}{3.106517in}}%
\pgfpathlineto{\pgfqpoint{5.003015in}{3.113903in}}%
\pgfpathlineto{\pgfqpoint{5.015578in}{3.117596in}}%
\pgfpathlineto{\pgfqpoint{5.028141in}{3.136061in}}%
\pgfpathlineto{\pgfqpoint{5.053266in}{3.158219in}}%
\pgfpathlineto{\pgfqpoint{5.065829in}{3.165604in}}%
\pgfpathlineto{\pgfqpoint{5.078392in}{3.180376in}}%
\pgfpathlineto{\pgfqpoint{5.090955in}{3.184069in}}%
\pgfpathlineto{\pgfqpoint{5.103518in}{3.195148in}}%
\pgfpathlineto{\pgfqpoint{5.116080in}{3.195148in}}%
\pgfpathlineto{\pgfqpoint{5.128643in}{3.198841in}}%
\pgfpathlineto{\pgfqpoint{5.141206in}{3.213613in}}%
\pgfpathlineto{\pgfqpoint{5.153769in}{3.235770in}}%
\pgfpathlineto{\pgfqpoint{5.178894in}{3.243156in}}%
\pgfpathlineto{\pgfqpoint{5.191457in}{3.243156in}}%
\pgfpathlineto{\pgfqpoint{5.204020in}{3.239463in}}%
\pgfpathlineto{\pgfqpoint{5.216583in}{3.246849in}}%
\pgfpathlineto{\pgfqpoint{5.229146in}{3.257928in}}%
\pgfpathlineto{\pgfqpoint{5.241709in}{3.257928in}}%
\pgfpathlineto{\pgfqpoint{5.254271in}{3.261621in}}%
\pgfpathlineto{\pgfqpoint{5.266834in}{3.254235in}}%
\pgfpathlineto{\pgfqpoint{5.279397in}{3.254235in}}%
\pgfpathlineto{\pgfqpoint{5.291960in}{3.257928in}}%
\pgfpathlineto{\pgfqpoint{5.304523in}{3.254235in}}%
\pgfpathlineto{\pgfqpoint{5.317085in}{3.269007in}}%
\pgfpathlineto{\pgfqpoint{5.342211in}{3.276393in}}%
\pgfpathlineto{\pgfqpoint{5.354774in}{3.287471in}}%
\pgfpathlineto{\pgfqpoint{5.367337in}{3.291164in}}%
\pgfpathlineto{\pgfqpoint{5.379899in}{3.291164in}}%
\pgfpathlineto{\pgfqpoint{5.392462in}{3.294857in}}%
\pgfpathlineto{\pgfqpoint{5.405025in}{3.302243in}}%
\pgfpathlineto{\pgfqpoint{5.417588in}{3.313322in}}%
\pgfpathlineto{\pgfqpoint{5.430151in}{3.313322in}}%
\pgfpathlineto{\pgfqpoint{5.455276in}{3.320708in}}%
\pgfpathlineto{\pgfqpoint{5.467839in}{3.331787in}}%
\pgfpathlineto{\pgfqpoint{5.480402in}{3.328094in}}%
\pgfpathlineto{\pgfqpoint{5.492965in}{3.331787in}}%
\pgfpathlineto{\pgfqpoint{5.505528in}{3.339173in}}%
\pgfpathlineto{\pgfqpoint{5.518090in}{3.350251in}}%
\pgfpathlineto{\pgfqpoint{5.543216in}{3.357637in}}%
\pgfpathlineto{\pgfqpoint{5.555779in}{3.365023in}}%
\pgfpathlineto{\pgfqpoint{5.568342in}{3.361330in}}%
\pgfpathlineto{\pgfqpoint{5.580905in}{3.365023in}}%
\pgfpathlineto{\pgfqpoint{5.593467in}{3.365023in}}%
\pgfpathlineto{\pgfqpoint{5.606030in}{3.353944in}}%
\pgfpathlineto{\pgfqpoint{5.631156in}{3.353944in}}%
\pgfpathlineto{\pgfqpoint{5.643719in}{3.372409in}}%
\pgfpathlineto{\pgfqpoint{5.656281in}{3.368716in}}%
\pgfpathlineto{\pgfqpoint{5.668844in}{3.368716in}}%
\pgfpathlineto{\pgfqpoint{5.681407in}{3.376102in}}%
\pgfpathlineto{\pgfqpoint{5.706533in}{3.383488in}}%
\pgfpathlineto{\pgfqpoint{5.719095in}{3.394567in}}%
\pgfpathlineto{\pgfqpoint{5.731658in}{3.394567in}}%
\pgfpathlineto{\pgfqpoint{5.744221in}{3.398260in}}%
\pgfpathlineto{\pgfqpoint{5.756784in}{3.398260in}}%
\pgfpathlineto{\pgfqpoint{5.769347in}{3.409338in}}%
\pgfpathlineto{\pgfqpoint{5.781910in}{3.416724in}}%
\pgfpathlineto{\pgfqpoint{5.794472in}{3.420417in}}%
\pgfpathlineto{\pgfqpoint{5.807035in}{3.431496in}}%
\pgfpathlineto{\pgfqpoint{5.819598in}{3.438882in}}%
\pgfpathlineto{\pgfqpoint{5.832161in}{3.435189in}}%
\pgfpathlineto{\pgfqpoint{5.844724in}{3.438882in}}%
\pgfpathlineto{\pgfqpoint{5.857286in}{3.438882in}}%
\pgfpathlineto{\pgfqpoint{5.869849in}{3.442575in}}%
\pgfpathlineto{\pgfqpoint{5.882412in}{3.449961in}}%
\pgfpathlineto{\pgfqpoint{5.894975in}{3.449961in}}%
\pgfpathlineto{\pgfqpoint{5.907538in}{3.453654in}}%
\pgfpathlineto{\pgfqpoint{5.932663in}{3.475811in}}%
\pgfpathlineto{\pgfqpoint{5.945226in}{3.475811in}}%
\pgfpathlineto{\pgfqpoint{5.982915in}{3.486890in}}%
\pgfpathlineto{\pgfqpoint{5.995477in}{3.479504in}}%
\pgfpathlineto{\pgfqpoint{6.008040in}{3.486890in}}%
\pgfpathlineto{\pgfqpoint{6.033166in}{3.494276in}}%
\pgfpathlineto{\pgfqpoint{6.045729in}{3.505355in}}%
\pgfpathlineto{\pgfqpoint{6.058291in}{3.512741in}}%
\pgfpathlineto{\pgfqpoint{6.070854in}{3.509048in}}%
\pgfpathlineto{\pgfqpoint{6.083417in}{3.512741in}}%
\pgfpathlineto{\pgfqpoint{6.095980in}{3.520127in}}%
\pgfpathlineto{\pgfqpoint{6.108543in}{3.523820in}}%
\pgfpathlineto{\pgfqpoint{6.121106in}{3.523820in}}%
\pgfpathlineto{\pgfqpoint{6.133668in}{3.527512in}}%
\pgfpathlineto{\pgfqpoint{6.158794in}{3.527512in}}%
\pgfpathlineto{\pgfqpoint{6.171357in}{3.531205in}}%
\pgfpathlineto{\pgfqpoint{6.183920in}{3.545977in}}%
\pgfpathlineto{\pgfqpoint{6.196482in}{3.545977in}}%
\pgfpathlineto{\pgfqpoint{6.209045in}{3.542284in}}%
\pgfpathlineto{\pgfqpoint{6.221608in}{3.545977in}}%
\pgfpathlineto{\pgfqpoint{6.246734in}{3.545977in}}%
\pgfpathlineto{\pgfqpoint{6.271859in}{3.553363in}}%
\pgfpathlineto{\pgfqpoint{6.284422in}{3.553363in}}%
\pgfpathlineto{\pgfqpoint{6.296985in}{3.560749in}}%
\pgfpathlineto{\pgfqpoint{6.309548in}{3.557056in}}%
\pgfpathlineto{\pgfqpoint{6.322111in}{3.560749in}}%
\pgfpathlineto{\pgfqpoint{6.334673in}{3.560749in}}%
\pgfpathlineto{\pgfqpoint{6.347236in}{3.557056in}}%
\pgfpathlineto{\pgfqpoint{6.372362in}{3.557056in}}%
\pgfpathlineto{\pgfqpoint{6.384925in}{3.553363in}}%
\pgfpathlineto{\pgfqpoint{6.410050in}{3.553363in}}%
\pgfpathlineto{\pgfqpoint{6.422613in}{3.549670in}}%
\pgfpathlineto{\pgfqpoint{6.435176in}{3.557056in}}%
\pgfpathlineto{\pgfqpoint{6.460302in}{3.557056in}}%
\pgfpathlineto{\pgfqpoint{6.485427in}{3.564442in}}%
\pgfpathlineto{\pgfqpoint{6.497990in}{3.560749in}}%
\pgfpathlineto{\pgfqpoint{6.510553in}{3.564442in}}%
\pgfpathlineto{\pgfqpoint{6.523116in}{3.564442in}}%
\pgfpathlineto{\pgfqpoint{6.535678in}{3.560749in}}%
\pgfpathlineto{\pgfqpoint{6.548241in}{3.564442in}}%
\pgfpathlineto{\pgfqpoint{6.585930in}{3.564442in}}%
\pgfpathlineto{\pgfqpoint{6.598492in}{3.560749in}}%
\pgfpathlineto{\pgfqpoint{6.611055in}{3.560749in}}%
\pgfpathlineto{\pgfqpoint{6.623618in}{3.564442in}}%
\pgfpathlineto{\pgfqpoint{6.636181in}{3.564442in}}%
\pgfpathlineto{\pgfqpoint{6.661307in}{3.557056in}}%
\pgfpathlineto{\pgfqpoint{6.673869in}{3.564442in}}%
\pgfpathlineto{\pgfqpoint{6.698995in}{3.564442in}}%
\pgfpathlineto{\pgfqpoint{6.711558in}{3.571828in}}%
\pgfpathlineto{\pgfqpoint{6.724121in}{3.582907in}}%
\pgfpathlineto{\pgfqpoint{6.736683in}{3.582907in}}%
\pgfpathlineto{\pgfqpoint{6.749246in}{3.579214in}}%
\pgfpathlineto{\pgfqpoint{6.761809in}{3.579214in}}%
\pgfpathlineto{\pgfqpoint{6.774372in}{3.582907in}}%
\pgfpathlineto{\pgfqpoint{6.786935in}{3.582907in}}%
\pgfpathlineto{\pgfqpoint{6.799497in}{3.586600in}}%
\pgfpathlineto{\pgfqpoint{6.824623in}{3.586600in}}%
\pgfpathlineto{\pgfqpoint{6.837186in}{3.597678in}}%
\pgfpathlineto{\pgfqpoint{6.849749in}{3.593985in}}%
\pgfpathlineto{\pgfqpoint{6.874874in}{3.593985in}}%
\pgfpathlineto{\pgfqpoint{6.900000in}{3.601371in}}%
\pgfpathlineto{\pgfqpoint{6.900000in}{3.601371in}}%
\pgfusepath{stroke}%
\end{pgfscope}%
\begin{pgfscope}%
\pgfpathrectangle{\pgfqpoint{4.100000in}{0.880000in}}{\pgfqpoint{3.100000in}{3.080000in}}%
\pgfusepath{clip}%
\pgfsetroundcap%
\pgfsetroundjoin%
\pgfsetlinewidth{1.204500pt}%
\definecolor{currentstroke}{rgb}{1.000000,0.498039,0.054902}%
\pgfsetstrokecolor{currentstroke}%
\pgfsetstrokeopacity{0.850000}%
\pgfsetdash{}{0pt}%
\pgfpathmoveto{\pgfqpoint{4.400000in}{1.226811in}}%
\pgfpathlineto{\pgfqpoint{4.425126in}{1.655192in}}%
\pgfpathlineto{\pgfqpoint{4.462814in}{2.094652in}}%
\pgfpathlineto{\pgfqpoint{4.475377in}{2.168511in}}%
\pgfpathlineto{\pgfqpoint{4.487940in}{2.290378in}}%
\pgfpathlineto{\pgfqpoint{4.513065in}{2.467639in}}%
\pgfpathlineto{\pgfqpoint{4.525628in}{2.493489in}}%
\pgfpathlineto{\pgfqpoint{4.538191in}{2.511954in}}%
\pgfpathlineto{\pgfqpoint{4.550754in}{2.578427in}}%
\pgfpathlineto{\pgfqpoint{4.575879in}{2.652286in}}%
\pgfpathlineto{\pgfqpoint{4.601005in}{2.711373in}}%
\pgfpathlineto{\pgfqpoint{4.613568in}{2.748302in}}%
\pgfpathlineto{\pgfqpoint{4.626131in}{2.748302in}}%
\pgfpathlineto{\pgfqpoint{4.638693in}{2.777846in}}%
\pgfpathlineto{\pgfqpoint{4.651256in}{2.766767in}}%
\pgfpathlineto{\pgfqpoint{4.663819in}{2.803696in}}%
\pgfpathlineto{\pgfqpoint{4.676382in}{2.851705in}}%
\pgfpathlineto{\pgfqpoint{4.688945in}{2.888634in}}%
\pgfpathlineto{\pgfqpoint{4.701508in}{2.888634in}}%
\pgfpathlineto{\pgfqpoint{4.714070in}{2.903406in}}%
\pgfpathlineto{\pgfqpoint{4.726633in}{2.914485in}}%
\pgfpathlineto{\pgfqpoint{4.739196in}{2.936642in}}%
\pgfpathlineto{\pgfqpoint{4.751759in}{2.947721in}}%
\pgfpathlineto{\pgfqpoint{4.764322in}{2.973572in}}%
\pgfpathlineto{\pgfqpoint{4.789447in}{2.980957in}}%
\pgfpathlineto{\pgfqpoint{4.802010in}{2.992036in}}%
\pgfpathlineto{\pgfqpoint{4.814573in}{3.010501in}}%
\pgfpathlineto{\pgfqpoint{4.827136in}{3.032659in}}%
\pgfpathlineto{\pgfqpoint{4.839698in}{3.025273in}}%
\pgfpathlineto{\pgfqpoint{4.852261in}{3.010501in}}%
\pgfpathlineto{\pgfqpoint{4.864824in}{3.028966in}}%
\pgfpathlineto{\pgfqpoint{4.877387in}{3.028966in}}%
\pgfpathlineto{\pgfqpoint{4.889950in}{3.032659in}}%
\pgfpathlineto{\pgfqpoint{4.902513in}{3.025273in}}%
\pgfpathlineto{\pgfqpoint{4.915075in}{3.040045in}}%
\pgfpathlineto{\pgfqpoint{4.927638in}{3.058509in}}%
\pgfpathlineto{\pgfqpoint{4.940201in}{3.065895in}}%
\pgfpathlineto{\pgfqpoint{4.965327in}{3.095439in}}%
\pgfpathlineto{\pgfqpoint{4.977889in}{3.113903in}}%
\pgfpathlineto{\pgfqpoint{4.990452in}{3.113903in}}%
\pgfpathlineto{\pgfqpoint{5.003015in}{3.110210in}}%
\pgfpathlineto{\pgfqpoint{5.015578in}{3.117596in}}%
\pgfpathlineto{\pgfqpoint{5.028141in}{3.113903in}}%
\pgfpathlineto{\pgfqpoint{5.040704in}{3.113903in}}%
\pgfpathlineto{\pgfqpoint{5.065829in}{3.121289in}}%
\pgfpathlineto{\pgfqpoint{5.078392in}{3.132368in}}%
\pgfpathlineto{\pgfqpoint{5.090955in}{3.139754in}}%
\pgfpathlineto{\pgfqpoint{5.103518in}{3.136061in}}%
\pgfpathlineto{\pgfqpoint{5.116080in}{3.139754in}}%
\pgfpathlineto{\pgfqpoint{5.128643in}{3.136061in}}%
\pgfpathlineto{\pgfqpoint{5.141206in}{3.147140in}}%
\pgfpathlineto{\pgfqpoint{5.153769in}{3.154526in}}%
\pgfpathlineto{\pgfqpoint{5.166332in}{3.165604in}}%
\pgfpathlineto{\pgfqpoint{5.178894in}{3.165604in}}%
\pgfpathlineto{\pgfqpoint{5.191457in}{3.169297in}}%
\pgfpathlineto{\pgfqpoint{5.204020in}{3.176683in}}%
\pgfpathlineto{\pgfqpoint{5.216583in}{3.172990in}}%
\pgfpathlineto{\pgfqpoint{5.229146in}{3.172990in}}%
\pgfpathlineto{\pgfqpoint{5.241709in}{3.176683in}}%
\pgfpathlineto{\pgfqpoint{5.266834in}{3.191455in}}%
\pgfpathlineto{\pgfqpoint{5.279397in}{3.202534in}}%
\pgfpathlineto{\pgfqpoint{5.291960in}{3.217306in}}%
\pgfpathlineto{\pgfqpoint{5.304523in}{3.213613in}}%
\pgfpathlineto{\pgfqpoint{5.329648in}{3.220999in}}%
\pgfpathlineto{\pgfqpoint{5.342211in}{3.220999in}}%
\pgfpathlineto{\pgfqpoint{5.354774in}{3.213613in}}%
\pgfpathlineto{\pgfqpoint{5.367337in}{3.213613in}}%
\pgfpathlineto{\pgfqpoint{5.405025in}{3.235770in}}%
\pgfpathlineto{\pgfqpoint{5.417588in}{3.232077in}}%
\pgfpathlineto{\pgfqpoint{5.430151in}{3.235770in}}%
\pgfpathlineto{\pgfqpoint{5.455276in}{3.250542in}}%
\pgfpathlineto{\pgfqpoint{5.467839in}{3.261621in}}%
\pgfpathlineto{\pgfqpoint{5.480402in}{3.265314in}}%
\pgfpathlineto{\pgfqpoint{5.492965in}{3.265314in}}%
\pgfpathlineto{\pgfqpoint{5.505528in}{3.269007in}}%
\pgfpathlineto{\pgfqpoint{5.518090in}{3.276393in}}%
\pgfpathlineto{\pgfqpoint{5.530653in}{3.276393in}}%
\pgfpathlineto{\pgfqpoint{5.543216in}{3.272700in}}%
\pgfpathlineto{\pgfqpoint{5.568342in}{3.280086in}}%
\pgfpathlineto{\pgfqpoint{5.580905in}{3.276393in}}%
\pgfpathlineto{\pgfqpoint{5.593467in}{3.276393in}}%
\pgfpathlineto{\pgfqpoint{5.606030in}{3.291164in}}%
\pgfpathlineto{\pgfqpoint{5.618593in}{3.302243in}}%
\pgfpathlineto{\pgfqpoint{5.631156in}{3.302243in}}%
\pgfpathlineto{\pgfqpoint{5.656281in}{3.317015in}}%
\pgfpathlineto{\pgfqpoint{5.668844in}{3.328094in}}%
\pgfpathlineto{\pgfqpoint{5.693970in}{3.335480in}}%
\pgfpathlineto{\pgfqpoint{5.706533in}{3.335480in}}%
\pgfpathlineto{\pgfqpoint{5.731658in}{3.342866in}}%
\pgfpathlineto{\pgfqpoint{5.744221in}{3.339173in}}%
\pgfpathlineto{\pgfqpoint{5.756784in}{3.339173in}}%
\pgfpathlineto{\pgfqpoint{5.769347in}{3.342866in}}%
\pgfpathlineto{\pgfqpoint{5.832161in}{3.342866in}}%
\pgfpathlineto{\pgfqpoint{5.844724in}{3.346558in}}%
\pgfpathlineto{\pgfqpoint{5.882412in}{3.346558in}}%
\pgfpathlineto{\pgfqpoint{5.894975in}{3.350251in}}%
\pgfpathlineto{\pgfqpoint{5.920101in}{3.350251in}}%
\pgfpathlineto{\pgfqpoint{5.932663in}{3.346558in}}%
\pgfpathlineto{\pgfqpoint{5.945226in}{3.350251in}}%
\pgfpathlineto{\pgfqpoint{5.957789in}{3.350251in}}%
\pgfpathlineto{\pgfqpoint{5.970352in}{3.353944in}}%
\pgfpathlineto{\pgfqpoint{5.982915in}{3.350251in}}%
\pgfpathlineto{\pgfqpoint{6.008040in}{3.372409in}}%
\pgfpathlineto{\pgfqpoint{6.020603in}{3.376102in}}%
\pgfpathlineto{\pgfqpoint{6.058291in}{3.376102in}}%
\pgfpathlineto{\pgfqpoint{6.070854in}{3.383488in}}%
\pgfpathlineto{\pgfqpoint{6.095980in}{3.383488in}}%
\pgfpathlineto{\pgfqpoint{6.108543in}{3.387181in}}%
\pgfpathlineto{\pgfqpoint{6.121106in}{3.394567in}}%
\pgfpathlineto{\pgfqpoint{6.133668in}{3.394567in}}%
\pgfpathlineto{\pgfqpoint{6.146231in}{3.398260in}}%
\pgfpathlineto{\pgfqpoint{6.158794in}{3.398260in}}%
\pgfpathlineto{\pgfqpoint{6.183920in}{3.405645in}}%
\pgfpathlineto{\pgfqpoint{6.196482in}{3.413031in}}%
\pgfpathlineto{\pgfqpoint{6.209045in}{3.413031in}}%
\pgfpathlineto{\pgfqpoint{6.221608in}{3.420417in}}%
\pgfpathlineto{\pgfqpoint{6.234171in}{3.413031in}}%
\pgfpathlineto{\pgfqpoint{6.246734in}{3.413031in}}%
\pgfpathlineto{\pgfqpoint{6.259296in}{3.416724in}}%
\pgfpathlineto{\pgfqpoint{6.284422in}{3.416724in}}%
\pgfpathlineto{\pgfqpoint{6.296985in}{3.420417in}}%
\pgfpathlineto{\pgfqpoint{6.309548in}{3.416724in}}%
\pgfpathlineto{\pgfqpoint{6.359799in}{3.416724in}}%
\pgfpathlineto{\pgfqpoint{6.372362in}{3.420417in}}%
\pgfpathlineto{\pgfqpoint{6.384925in}{3.420417in}}%
\pgfpathlineto{\pgfqpoint{6.410050in}{3.427803in}}%
\pgfpathlineto{\pgfqpoint{6.422613in}{3.427803in}}%
\pgfpathlineto{\pgfqpoint{6.435176in}{3.431496in}}%
\pgfpathlineto{\pgfqpoint{6.460302in}{3.424110in}}%
\pgfpathlineto{\pgfqpoint{6.472864in}{3.427803in}}%
\pgfpathlineto{\pgfqpoint{6.485427in}{3.427803in}}%
\pgfpathlineto{\pgfqpoint{6.510553in}{3.435189in}}%
\pgfpathlineto{\pgfqpoint{6.535678in}{3.435189in}}%
\pgfpathlineto{\pgfqpoint{6.548241in}{3.442575in}}%
\pgfpathlineto{\pgfqpoint{6.560804in}{3.442575in}}%
\pgfpathlineto{\pgfqpoint{6.585930in}{3.449961in}}%
\pgfpathlineto{\pgfqpoint{6.623618in}{3.449961in}}%
\pgfpathlineto{\pgfqpoint{6.636181in}{3.457347in}}%
\pgfpathlineto{\pgfqpoint{6.673869in}{3.457347in}}%
\pgfpathlineto{\pgfqpoint{6.686432in}{3.461040in}}%
\pgfpathlineto{\pgfqpoint{6.698995in}{3.461040in}}%
\pgfpathlineto{\pgfqpoint{6.711558in}{3.464733in}}%
\pgfpathlineto{\pgfqpoint{6.812060in}{3.464733in}}%
\pgfpathlineto{\pgfqpoint{6.824623in}{3.468425in}}%
\pgfpathlineto{\pgfqpoint{6.849749in}{3.468425in}}%
\pgfpathlineto{\pgfqpoint{6.874874in}{3.475811in}}%
\pgfpathlineto{\pgfqpoint{6.887437in}{3.475811in}}%
\pgfpathlineto{\pgfqpoint{6.900000in}{3.479504in}}%
\pgfpathlineto{\pgfqpoint{6.900000in}{3.479504in}}%
\pgfusepath{stroke}%
\end{pgfscope}%
\begin{pgfscope}%
\pgfpathrectangle{\pgfqpoint{4.100000in}{0.880000in}}{\pgfqpoint{3.100000in}{3.080000in}}%
\pgfusepath{clip}%
\pgfsetroundcap%
\pgfsetroundjoin%
\pgfsetlinewidth{1.204500pt}%
\definecolor{currentstroke}{rgb}{0.172549,0.627451,0.172549}%
\pgfsetstrokecolor{currentstroke}%
\pgfsetstrokeopacity{0.850000}%
\pgfsetdash{}{0pt}%
\pgfpathmoveto{\pgfqpoint{4.400000in}{1.226811in}}%
\pgfpathlineto{\pgfqpoint{4.425126in}{1.655192in}}%
\pgfpathlineto{\pgfqpoint{4.462814in}{2.076187in}}%
\pgfpathlineto{\pgfqpoint{4.475377in}{2.161125in}}%
\pgfpathlineto{\pgfqpoint{4.487940in}{2.282992in}}%
\pgfpathlineto{\pgfqpoint{4.500503in}{2.382701in}}%
\pgfpathlineto{\pgfqpoint{4.513065in}{2.445481in}}%
\pgfpathlineto{\pgfqpoint{4.525628in}{2.489797in}}%
\pgfpathlineto{\pgfqpoint{4.538191in}{2.511954in}}%
\pgfpathlineto{\pgfqpoint{4.550754in}{2.545191in}}%
\pgfpathlineto{\pgfqpoint{4.563317in}{2.615356in}}%
\pgfpathlineto{\pgfqpoint{4.575879in}{2.615356in}}%
\pgfpathlineto{\pgfqpoint{4.588442in}{2.648593in}}%
\pgfpathlineto{\pgfqpoint{4.601005in}{2.696601in}}%
\pgfpathlineto{\pgfqpoint{4.613568in}{2.711373in}}%
\pgfpathlineto{\pgfqpoint{4.626131in}{2.733531in}}%
\pgfpathlineto{\pgfqpoint{4.638693in}{2.751995in}}%
\pgfpathlineto{\pgfqpoint{4.651256in}{2.785232in}}%
\pgfpathlineto{\pgfqpoint{4.663819in}{2.788925in}}%
\pgfpathlineto{\pgfqpoint{4.676382in}{2.788925in}}%
\pgfpathlineto{\pgfqpoint{4.688945in}{2.796311in}}%
\pgfpathlineto{\pgfqpoint{4.701508in}{2.844319in}}%
\pgfpathlineto{\pgfqpoint{4.714070in}{2.866476in}}%
\pgfpathlineto{\pgfqpoint{4.726633in}{2.881248in}}%
\pgfpathlineto{\pgfqpoint{4.751759in}{2.896020in}}%
\pgfpathlineto{\pgfqpoint{4.764322in}{2.892327in}}%
\pgfpathlineto{\pgfqpoint{4.776884in}{2.899713in}}%
\pgfpathlineto{\pgfqpoint{4.789447in}{2.910792in}}%
\pgfpathlineto{\pgfqpoint{4.802010in}{2.936642in}}%
\pgfpathlineto{\pgfqpoint{4.814573in}{2.947721in}}%
\pgfpathlineto{\pgfqpoint{4.839698in}{2.955107in}}%
\pgfpathlineto{\pgfqpoint{4.852261in}{2.951414in}}%
\pgfpathlineto{\pgfqpoint{4.864824in}{2.973572in}}%
\pgfpathlineto{\pgfqpoint{4.877387in}{2.992036in}}%
\pgfpathlineto{\pgfqpoint{4.889950in}{3.006808in}}%
\pgfpathlineto{\pgfqpoint{4.902513in}{3.010501in}}%
\pgfpathlineto{\pgfqpoint{4.915075in}{3.003115in}}%
\pgfpathlineto{\pgfqpoint{4.927638in}{2.988343in}}%
\pgfpathlineto{\pgfqpoint{4.940201in}{2.999422in}}%
\pgfpathlineto{\pgfqpoint{4.952764in}{3.003115in}}%
\pgfpathlineto{\pgfqpoint{4.965327in}{2.999422in}}%
\pgfpathlineto{\pgfqpoint{4.977889in}{3.006808in}}%
\pgfpathlineto{\pgfqpoint{4.990452in}{3.010501in}}%
\pgfpathlineto{\pgfqpoint{5.003015in}{3.017887in}}%
\pgfpathlineto{\pgfqpoint{5.015578in}{3.014194in}}%
\pgfpathlineto{\pgfqpoint{5.028141in}{3.036352in}}%
\pgfpathlineto{\pgfqpoint{5.040704in}{3.028966in}}%
\pgfpathlineto{\pgfqpoint{5.053266in}{3.043737in}}%
\pgfpathlineto{\pgfqpoint{5.065829in}{3.047430in}}%
\pgfpathlineto{\pgfqpoint{5.078392in}{3.062202in}}%
\pgfpathlineto{\pgfqpoint{5.090955in}{3.062202in}}%
\pgfpathlineto{\pgfqpoint{5.103518in}{3.076974in}}%
\pgfpathlineto{\pgfqpoint{5.116080in}{3.080667in}}%
\pgfpathlineto{\pgfqpoint{5.128643in}{3.069588in}}%
\pgfpathlineto{\pgfqpoint{5.141206in}{3.076974in}}%
\pgfpathlineto{\pgfqpoint{5.153769in}{3.076974in}}%
\pgfpathlineto{\pgfqpoint{5.166332in}{3.073281in}}%
\pgfpathlineto{\pgfqpoint{5.178894in}{3.080667in}}%
\pgfpathlineto{\pgfqpoint{5.191457in}{3.080667in}}%
\pgfpathlineto{\pgfqpoint{5.216583in}{3.073281in}}%
\pgfpathlineto{\pgfqpoint{5.229146in}{3.088053in}}%
\pgfpathlineto{\pgfqpoint{5.241709in}{3.088053in}}%
\pgfpathlineto{\pgfqpoint{5.279397in}{3.099132in}}%
\pgfpathlineto{\pgfqpoint{5.291960in}{3.106517in}}%
\pgfpathlineto{\pgfqpoint{5.304523in}{3.099132in}}%
\pgfpathlineto{\pgfqpoint{5.342211in}{3.110210in}}%
\pgfpathlineto{\pgfqpoint{5.354774in}{3.110210in}}%
\pgfpathlineto{\pgfqpoint{5.367337in}{3.121289in}}%
\pgfpathlineto{\pgfqpoint{5.392462in}{3.128675in}}%
\pgfpathlineto{\pgfqpoint{5.405025in}{3.128675in}}%
\pgfpathlineto{\pgfqpoint{5.417588in}{3.132368in}}%
\pgfpathlineto{\pgfqpoint{5.430151in}{3.128675in}}%
\pgfpathlineto{\pgfqpoint{5.455276in}{3.128675in}}%
\pgfpathlineto{\pgfqpoint{5.480402in}{3.136061in}}%
\pgfpathlineto{\pgfqpoint{5.492965in}{3.147140in}}%
\pgfpathlineto{\pgfqpoint{5.505528in}{3.143447in}}%
\pgfpathlineto{\pgfqpoint{5.518090in}{3.136061in}}%
\pgfpathlineto{\pgfqpoint{5.530653in}{3.136061in}}%
\pgfpathlineto{\pgfqpoint{5.555779in}{3.143447in}}%
\pgfpathlineto{\pgfqpoint{5.568342in}{3.150833in}}%
\pgfpathlineto{\pgfqpoint{5.593467in}{3.150833in}}%
\pgfpathlineto{\pgfqpoint{5.606030in}{3.161912in}}%
\pgfpathlineto{\pgfqpoint{5.618593in}{3.165604in}}%
\pgfpathlineto{\pgfqpoint{5.643719in}{3.165604in}}%
\pgfpathlineto{\pgfqpoint{5.656281in}{3.172990in}}%
\pgfpathlineto{\pgfqpoint{5.668844in}{3.176683in}}%
\pgfpathlineto{\pgfqpoint{5.681407in}{3.176683in}}%
\pgfpathlineto{\pgfqpoint{5.693970in}{3.180376in}}%
\pgfpathlineto{\pgfqpoint{5.731658in}{3.180376in}}%
\pgfpathlineto{\pgfqpoint{5.744221in}{3.184069in}}%
\pgfpathlineto{\pgfqpoint{5.781910in}{3.184069in}}%
\pgfpathlineto{\pgfqpoint{5.794472in}{3.187762in}}%
\pgfpathlineto{\pgfqpoint{5.807035in}{3.187762in}}%
\pgfpathlineto{\pgfqpoint{5.819598in}{3.180376in}}%
\pgfpathlineto{\pgfqpoint{5.832161in}{3.176683in}}%
\pgfpathlineto{\pgfqpoint{5.844724in}{3.180376in}}%
\pgfpathlineto{\pgfqpoint{5.869849in}{3.180376in}}%
\pgfpathlineto{\pgfqpoint{5.882412in}{3.184069in}}%
\pgfpathlineto{\pgfqpoint{5.894975in}{3.184069in}}%
\pgfpathlineto{\pgfqpoint{5.907538in}{3.187762in}}%
\pgfpathlineto{\pgfqpoint{5.970352in}{3.187762in}}%
\pgfpathlineto{\pgfqpoint{5.995477in}{3.195148in}}%
\pgfpathlineto{\pgfqpoint{6.008040in}{3.202534in}}%
\pgfpathlineto{\pgfqpoint{6.146231in}{3.202534in}}%
\pgfpathlineto{\pgfqpoint{6.158794in}{3.206227in}}%
\pgfpathlineto{\pgfqpoint{6.234171in}{3.206227in}}%
\pgfpathlineto{\pgfqpoint{6.246734in}{3.202534in}}%
\pgfpathlineto{\pgfqpoint{6.284422in}{3.202534in}}%
\pgfpathlineto{\pgfqpoint{6.296985in}{3.206227in}}%
\pgfpathlineto{\pgfqpoint{6.410050in}{3.206227in}}%
\pgfpathlineto{\pgfqpoint{6.435176in}{3.213613in}}%
\pgfpathlineto{\pgfqpoint{6.472864in}{3.213613in}}%
\pgfpathlineto{\pgfqpoint{6.485427in}{3.217306in}}%
\pgfpathlineto{\pgfqpoint{6.535678in}{3.217306in}}%
\pgfpathlineto{\pgfqpoint{6.548241in}{3.224691in}}%
\pgfpathlineto{\pgfqpoint{6.560804in}{3.224691in}}%
\pgfpathlineto{\pgfqpoint{6.573367in}{3.228384in}}%
\pgfpathlineto{\pgfqpoint{6.598492in}{3.228384in}}%
\pgfpathlineto{\pgfqpoint{6.611055in}{3.232077in}}%
\pgfpathlineto{\pgfqpoint{6.636181in}{3.232077in}}%
\pgfpathlineto{\pgfqpoint{6.648744in}{3.228384in}}%
\pgfpathlineto{\pgfqpoint{6.673869in}{3.228384in}}%
\pgfpathlineto{\pgfqpoint{6.686432in}{3.232077in}}%
\pgfpathlineto{\pgfqpoint{6.724121in}{3.232077in}}%
\pgfpathlineto{\pgfqpoint{6.736683in}{3.235770in}}%
\pgfpathlineto{\pgfqpoint{6.786935in}{3.235770in}}%
\pgfpathlineto{\pgfqpoint{6.799497in}{3.232077in}}%
\pgfpathlineto{\pgfqpoint{6.812060in}{3.235770in}}%
\pgfpathlineto{\pgfqpoint{6.824623in}{3.235770in}}%
\pgfpathlineto{\pgfqpoint{6.837186in}{3.239463in}}%
\pgfpathlineto{\pgfqpoint{6.900000in}{3.239463in}}%
\pgfpathlineto{\pgfqpoint{6.900000in}{3.239463in}}%
\pgfusepath{stroke}%
\end{pgfscope}%
\begin{pgfscope}%
\pgfsetrectcap%
\pgfsetmiterjoin%
\pgfsetlinewidth{0.803000pt}%
\definecolor{currentstroke}{rgb}{0.000000,0.000000,0.000000}%
\pgfsetstrokecolor{currentstroke}%
\pgfsetdash{}{0pt}%
\pgfpathmoveto{\pgfqpoint{4.100000in}{0.880000in}}%
\pgfpathlineto{\pgfqpoint{4.100000in}{3.960000in}}%
\pgfusepath{stroke}%
\end{pgfscope}%
\begin{pgfscope}%
\pgfsetrectcap%
\pgfsetmiterjoin%
\pgfsetlinewidth{0.803000pt}%
\definecolor{currentstroke}{rgb}{0.000000,0.000000,0.000000}%
\pgfsetstrokecolor{currentstroke}%
\pgfsetdash{}{0pt}%
\pgfpathmoveto{\pgfqpoint{7.200000in}{0.880000in}}%
\pgfpathlineto{\pgfqpoint{7.200000in}{3.960000in}}%
\pgfusepath{stroke}%
\end{pgfscope}%
\begin{pgfscope}%
\pgfsetrectcap%
\pgfsetmiterjoin%
\pgfsetlinewidth{0.803000pt}%
\definecolor{currentstroke}{rgb}{0.000000,0.000000,0.000000}%
\pgfsetstrokecolor{currentstroke}%
\pgfsetdash{}{0pt}%
\pgfpathmoveto{\pgfqpoint{4.100000in}{0.880000in}}%
\pgfpathlineto{\pgfqpoint{7.200000in}{0.880000in}}%
\pgfusepath{stroke}%
\end{pgfscope}%
\begin{pgfscope}%
\pgfsetrectcap%
\pgfsetmiterjoin%
\pgfsetlinewidth{0.803000pt}%
\definecolor{currentstroke}{rgb}{0.000000,0.000000,0.000000}%
\pgfsetstrokecolor{currentstroke}%
\pgfsetdash{}{0pt}%
\pgfpathmoveto{\pgfqpoint{4.100000in}{3.960000in}}%
\pgfpathlineto{\pgfqpoint{7.200000in}{3.960000in}}%
\pgfusepath{stroke}%
\end{pgfscope}%
\begin{pgfscope}%
\pgfsetbuttcap%
\pgfsetmiterjoin%
\definecolor{currentfill}{rgb}{1.000000,1.000000,1.000000}%
\pgfsetfillcolor{currentfill}%
\pgfsetlinewidth{1.003750pt}%
\definecolor{currentstroke}{rgb}{0.800000,0.800000,0.800000}%
\pgfsetstrokecolor{currentstroke}%
\pgfsetdash{}{0pt}%
\pgfpathmoveto{\pgfqpoint{5.888269in}{0.949444in}}%
\pgfpathlineto{\pgfqpoint{7.102778in}{0.949444in}}%
\pgfpathquadraticcurveto{\pgfqpoint{7.130556in}{0.949444in}}{\pgfqpoint{7.130556in}{0.977222in}}%
\pgfpathlineto{\pgfqpoint{7.130556in}{1.880000in}}%
\pgfpathquadraticcurveto{\pgfqpoint{7.130556in}{1.907777in}}{\pgfqpoint{7.102778in}{1.907777in}}%
\pgfpathlineto{\pgfqpoint{5.888269in}{1.907777in}}%
\pgfpathquadraticcurveto{\pgfqpoint{5.860492in}{1.907777in}}{\pgfqpoint{5.860492in}{1.880000in}}%
\pgfpathlineto{\pgfqpoint{5.860492in}{0.977222in}}%
\pgfpathquadraticcurveto{\pgfqpoint{5.860492in}{0.949444in}}{\pgfqpoint{5.888269in}{0.949444in}}%
\pgfpathclose%
\pgfusepath{stroke,fill}%
\end{pgfscope}%
\begin{pgfscope}%
\definecolor{textcolor}{rgb}{0.000000,0.000000,0.000000}%
\pgfsetstrokecolor{textcolor}%
\pgfsetfillcolor{textcolor}%
\pgftext[x=5.976040in, y=1.755771in, left, base]{\color{textcolor}\rmfamily\fontsize{10.000000}{12.000000}\selectfont Hyper-paramètre}%
\end{pgfscope}%
\begin{pgfscope}%
\definecolor{textcolor}{rgb}{0.000000,0.000000,0.000000}%
\pgfsetstrokecolor{textcolor}%
\pgfsetfillcolor{textcolor}%
\pgftext[x=5.916047in, y=1.613024in, left, base]{\color{textcolor}\rmfamily\fontsize{10.000000}{12.000000}\selectfont de régularisation C}%
\end{pgfscope}%
\begin{pgfscope}%
\pgfsetroundcap%
\pgfsetroundjoin%
\pgfsetlinewidth{1.204500pt}%
\definecolor{currentstroke}{rgb}{0.121569,0.466667,0.705882}%
\pgfsetstrokecolor{currentstroke}%
\pgfsetstrokeopacity{0.850000}%
\pgfsetdash{}{0pt}%
\pgfpathmoveto{\pgfqpoint{6.231635in}{1.467963in}}%
\pgfpathlineto{\pgfqpoint{6.509412in}{1.467963in}}%
\pgfusepath{stroke}%
\end{pgfscope}%
\begin{pgfscope}%
\definecolor{textcolor}{rgb}{0.000000,0.000000,0.000000}%
\pgfsetstrokecolor{textcolor}%
\pgfsetfillcolor{textcolor}%
\pgftext[x=6.620523in,y=1.419352in,left,base]{\color{textcolor}\rmfamily\fontsize{10.000000}{12.000000}\selectfont 1}%
\end{pgfscope}%
\begin{pgfscope}%
\pgfsetroundcap%
\pgfsetroundjoin%
\pgfsetlinewidth{1.204500pt}%
\definecolor{currentstroke}{rgb}{1.000000,0.498039,0.054902}%
\pgfsetstrokecolor{currentstroke}%
\pgfsetstrokeopacity{0.850000}%
\pgfsetdash{}{0pt}%
\pgfpathmoveto{\pgfqpoint{6.231635in}{1.274290in}}%
\pgfpathlineto{\pgfqpoint{6.509412in}{1.274290in}}%
\pgfusepath{stroke}%
\end{pgfscope}%
\begin{pgfscope}%
\definecolor{textcolor}{rgb}{0.000000,0.000000,0.000000}%
\pgfsetstrokecolor{textcolor}%
\pgfsetfillcolor{textcolor}%
\pgftext[x=6.620523in,y=1.225679in,left,base]{\color{textcolor}\rmfamily\fontsize{10.000000}{12.000000}\selectfont 10}%
\end{pgfscope}%
\begin{pgfscope}%
\pgfsetroundcap%
\pgfsetroundjoin%
\pgfsetlinewidth{1.204500pt}%
\definecolor{currentstroke}{rgb}{0.172549,0.627451,0.172549}%
\pgfsetstrokecolor{currentstroke}%
\pgfsetstrokeopacity{0.850000}%
\pgfsetdash{}{0pt}%
\pgfpathmoveto{\pgfqpoint{6.231635in}{1.080617in}}%
\pgfpathlineto{\pgfqpoint{6.509412in}{1.080617in}}%
\pgfusepath{stroke}%
\end{pgfscope}%
\begin{pgfscope}%
\definecolor{textcolor}{rgb}{0.000000,0.000000,0.000000}%
\pgfsetstrokecolor{textcolor}%
\pgfsetfillcolor{textcolor}%
\pgftext[x=6.620523in,y=1.032006in,left,base]{\color{textcolor}\rmfamily\fontsize{10.000000}{12.000000}\selectfont 30}%
\end{pgfscope}%
\end{pgfpicture}%
\makeatother%
\endgroup%

				\end{center}
				\caption{Perte et accuracy avec différentes valeurs de pondération de la pénalité L2 pour un classifeur svm pour l'ensemble d'entrainement et de test. \\\\Valeurs suivantes d'hyperparamètres : Taux d'apprentissage=0.0001, nombre d'epochs=200, taille de batch=5000}
			\end{figure}
		\end{reponse}

\end{enumerate}
\end{document}
