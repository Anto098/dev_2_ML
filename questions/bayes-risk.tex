\item \textbf{\enfr{Bayes Risk }{Risque de Bayes}}
\points{15 points}{10 points}

\newcommand{\lecture}[4]{
	\pagestyle{myheadings}
	\thispagestyle{plain}
	\newpage
	\setcounter{page}{1}
	\noindent
	\begin{center}
		\framebox{
			\vbox{\vspace{2mm}
				\hbox to 6.28in { {\bf ECE901 Spring 2007 Statistical Learning Theory \hfill Instructor: R. Nowak} }
				\vspace{6mm}
				\hbox to 6.28in { {\Large \hfill #1  \hfill} }
				\vspace{2mm}}
		}
	\end{center}
	\markboth{#1}{#1}
	\setlength{\headsep}{10mm}
	\vspace*{4mm}
}

%notation

\def\X{\ensuremath{{\mathcal X}}}
\def\Y{\ensuremath{{\mathcal Y}}}
\def\F{\ensuremath{{\mathcal F}}}
\def\R{\ensuremath{{\bf R}}}
\def\E{\mathbb{E}}
\def\sign{\ensuremath{\mbox{sign}}}
\def\ie{{\em i.e.,}}

\enfr{In this exercise, we will show that the Bayes classifier (assuming we use the true underlying target distribution) minimize the true risk over all possible classifiers.

	Recall that the goal of binary classification is to learn a mapping $f$ from the input space, $\X$, to to the class space, $\Y=\{0,1\}$.
	We can measure the loss of a classifier $f$ using the $0-1$ loss; \ie
	$$\ell(\hat{y},y) = \1{\hat{y}\neq y} = \left\{ \begin{array}{ll}
			1, \;\; & \text{if } \hat{y}\neq y \\
			0, \;\; & \text{otherwise}\end{array}
		\right.$$

	Recall that the true risk of $f$ is defined by
	$$R(f) = \E_{(x,y)\sim \mathcal{P}}\left[\ell(f(x),y)\right]$$
	where $\mathcal{P}$ is the underlying target distribution.

	Usually, we assume that $\mathcal{P}$ is unknown and we infer $f$ from a dataset drawn from $\mathcal{P}$. For this exercise, we will consider
	the Bayes classifier built using the target distribution $\mathcal{P}$, which is defined by
	$$f^{\star}(x) = \left\{ \begin{array}{ll}
			1,\;\; & \text{if }\eta(x) \geq 1/2 \\
			0,\;\; & otherwise\end{array}
		\right.$$
	where
	$\eta(x) \equiv P(Y=1 | X=x).$

	\medskip

	You will show that for any function $g:\X\to\Y$ we have $R(g) \geq R(f^{\star})$

	\begin{enumerate}
		\item First, show that $R(f) = P_{(x,y)\sim \mathcal{P}}(f(x)\neq y)$.
		\item Show that, for any $g:\X\to\Y$,
		      $$
			      P(g(x)\neq y \mid X=x)= 1- \left[ \1{g(x)=1}\eta(x) + \1{g(x)=0}\left(1-\eta(x)\right)
				      \right]$$
		\item Using the answer to the previous question, and the fact that $\1{g(x)=0}=1-\1{g(x)=1}$, show that, for any $g:\X\to\Y$,
		      $$P\left(g(x)\neq Y | X=x\right)-P\left(f^{\star}(x)\neq Y|X=x\right) = \left(2\eta(x)-1\right)\left(\1{f^{\star}(x)=1}-\1{g(x)=1}\right)
		      $$
		\item Finally, show that, for any $g:\X\to\Y$, $$ \left(2\eta(x)-1\right)\left(\1{f^{\star}(x)=1}-\1{g(x)=1}\right) \geq 0$$
		\item Conclude.
	\end{enumerate}
}
{
	Dans cet exercice, nous montrerons que le classificateur de Bayes (en supposant que nous utilisions la vraie distribution cible sous-jacente) minimise le risque réel sur tous les classificateurs possibles.

	Rappelons que le but de la classification binaire est d'apprendre une fonction $f$ de l'espace d'entrée, $\X$, vers l'espace des classes, $\Y=\{0,1\}$.
	On peut mesurer la qualité d'un classificateur $f$ en utilisant la fonction de coût $0-1$; \ie
	$$\ell(\hat{y},y) = \1{\hat{y}\neq y} = \left\{ \begin{array}{ll}
			1, \;\; & \text{if } \hat{y}\neq y \\
			0, \;\; & \text{otherwise}\end{array}
		\right.$$

	Rappelons que le vrai risque de $f$ est défini par
	$$R(f) = \E_{(x,y)\sim \mathcal{P}}\left[\ell(f(x),y)\right]$$
	où $\mathcal{P}$ est la distribution cible sous-jacente.

	Habituellement, nous supposons que $\mathcal{P}$ est inconnu et nous déduisons $f$ à partir d'un ensemble de données tirées de $\mathcal{P}$. Pour cet exercice, nous considérerons
	le classifieur de Bayes construit en utilisant la distribution cible $\mathcal{P}$, qui est définie par
	$$f^{\star}(x) = \left\{ \begin{array}{ll}
			1,\;\; & \text{si }\eta(x) \geq 1/2 \\
			0,\;\; & \text{sinon}\end{array}
		\right.$$
	où
	$\eta(x) \equiv P(Y=1 | X=x).$

	\medskip

	Vous montrerez que pour n'importe quelle fonction $g:\X\to\Y$ on a $R(g) \geq R(f^{\star})$

	\begin{enumerate}
		\item Tout d'abord, montrez que
		      \begin{equation} \label{eq:a} \tag{a}
			      R(f) = P_{(x,y)\sim \mathcal{P}}(f(x)\neq y)
		      \end{equation}

		      \begin{reponse}
			      \begin{align*}
				      R\left( f \right)
				       & =
				      \Esp_{(x,y) \sim \mathcal{P}}\left[\ell(f(x), y)\right]
				      \\
				       & =
				      \Esp_{(x,y) \sim \mathcal{P}}\left[\1{\hat{y} \neq y}\right]
				      \\
				       & =
				      \sum_{z \in \{0,1\}}\left(z \cdot P\left(\1{f(x) \neq y}\right) = z\right)
				      \\
				       & =
				      P\left(\1{f(x) \neq y } = 1 \right)
				      \\
				       & =
				      P_{\left(x,y\right) \sim \mathcal{P}}\left(f(x) \neq y)\right)
			      \end{align*}
		      \end{reponse}

		\item Montrez que, pour tout $g:\X\to\Y$,
		      \begin{equation} \label{eq:b} \tag{b}
			      P(g(x)\neq y \mid X=x)= 1- \left[ \1{g(x)=1}\eta(x) + \1{g(x)=0}\left(1-\eta(x)\right)
				      \right]
		      \end{equation}

		      \begin{reponse}
			      sachant que:
			      \begin{equation} \label{eq:*} \tag{*}
				      \Esp_{(x, y) \sim \mathcal{P}}\left[\1{f(x) \neq y}\right] = P\left(f(x) = y\right)
			      \end{equation}
			      On a:
			      \begin{align*}
				      P\left(g(x) \neq y \mid X = x\right)
				                                        & = 1 - P(g(x) = y \mid X = x)                                                                                                              \\
				                                        & = 1 - \left(P(Y = 1, g(x) = 1 \mid X = x\right) + P\left(Y = 0, g(x) = 0 \mid X = x)\right)                                               \\
				      \overset{\ref{eq:*}}              & {=} 1 - (\Esp\left[\1{y = 1} \cdot \1{g(x) = 1} \mid X = x\right] +\Esp\left[\1{y = 0} \cdot \1{g(x) = 0} \mid X = x\right])              \\
				      \overset{\text{indep}}            & {=}             1 - (\1{g(x) = 1} \cdot \Esp\left[\1{y = 1} \mid X = x\right] + \1{g(x) = 0} \cdot \Esp\left[\1{y = 0} \mid X = x\right]) \\
				      \overset{\ref{eq:*}}              & {=} 1 - ((\1{g(x) = 1} \cdot P(Y = 1 \mid X = x) + (\1{g(x) = 0} \cdot P(Y = 0 \mid X = x)))                                              \\
				      \overset{\Delta \text{ de } \eta} & {=} 1 - \left(\1{g(x) = 1} \cdot \eta(x) + \1{g(x) = 0} \cdot (1 - \eta(x)\right)
			      \end{align*}
		      \end{reponse}

		\item En utilisant la réponse à la question précédente et le fait que $\1{g(x)=0}=1-\1{g(x)=1}$, montrez que, pour toute fonction $g:\X\to\Y$,
		      \begin{equation} \label{eq:c}
			      \tag{c}
			      P\left(g(x)\neq Y | X=x\right)-P\left(f^{\star}(x)\neq Y|X=x\right) = \left(2\eta(x)-1\right)\left(\1{f^{\star}(x)=1}-\1{g(x)=1}\right)
		      \end{equation}

		      \begin{reponse}
			      En notant $a\left( x\right) = \1{g(x) = 1}$ et $b\left( x\right) = \1{f^{\star}(x) = 1}$
			      \begin{align*}
				         & P\left(g(x) \neq Y \mid X = x\right) - P\left(f^{\star}(x) \neq Y \mid X = x\right)                                                                                                                      \\
				      =  &
				      \left(1 - ((\1{g(x) = 1} \eta(x) + \1{g(x) = 0} (1 - \eta(x)))\right)-\left(1 - ((\1{f^{\star}(x) = 1} \eta(x) + \1{f^{\star}(x) = 0}(1 - \eta(x)))\right)                                                   \\
				      =  & \left(1 - ((\1{g(x) = 1} \eta(x) + (1 - \1{g(x) = 1}) (1 - \eta(x)))\right)                                                                                                                               \\
				      {} & -\left(1 - ((\1{f^{\star}(x) = 1} \eta(x) + (1 - \1{f^{\star}(x) = 1}) (1 - \eta(x)))\right)                                                                                                            \\
				      =  & \left(1 - (a\left( x\right) \eta + (1 - a\left( x\right))(1 - \eta\left(x\right)))\right)-\left(1 - (b\left( x\right) \eta\left(x\right) + (1 - b\left( x\right))(1 - \eta\left(x\right)))\right)         \\
				      =  & \left(1 - (a\left( x\right) \eta\left(x\right) + (1 - \eta\left(x\right) - a\left( x\right) + a\left( x\right) \eta\left(x\right)))\right)                                                                \\
				      {} & -\left(1 - (b\left( x\right) \eta\left(x\right) + (1 - \eta\left(x\right) - b\left( x\right) + b\left( x\right) \eta\left(x\right)))\right)                                                               \\
				      =  & \left(1 - (2 a\left( x\right) \eta\left(x\right) + 1 - \eta\left(x\right) - a\left( x\right))\right)-\left(1 - (2 b\left( x\right) \eta\left(x\right) + 1 - \eta\left(x\right) - b\left( x\right))\right) \\
				      =  & \left(-2 a\left( x\right) \eta\left(x\right) + \eta\left(x\right) + a\left( x\right)\right)-\left(-2 b\left( x\right) \eta\left(x\right) + \eta\left(x\right) + b\left( x\right)\right)                   \\
				      =  & 2 b\left( x\right) \eta\left(x\right) - 2 a\left( x\right) \eta\left(x\right) + b\left( x\right) -a\left( x\right)                                                                                        \\
				      =  & 2 \eta\left(x\right) (b\left( x\right) - a\left( x\right)) +\left(b\left( x\right)-a\left( x\right)\right)                                                                                                \\
				      =  & (2 \eta\left(x\right) - 1) (b\left( x\right) - a\left( x\right))                                                                                                                                          \\
				      =  & (2 \eta\left( x\right) - 1)(\1{f^{\star}(x) = 1} - \1{g(x) = 1})
			      \end{align*}
		      \end{reponse}

		\item Montrez que, pour tout $g:\X\to\Y$,
		      \begin{equation} \label{eq:d} \tag{d}
			      \left(2\eta(x)-1\right)\left(\1{f^{\star}(x)=1}-\1{g(x)=1}\right) \geq 0
		      \end{equation}

		      \begin{reponse}
			      Nous avons 2 cas :

			      \begin{itemize}
				      \item Si $\eta(x) \geq \frac{1}{2}$ :

				            Alors $\1{f^{\star}(x)=1} = \1{\eta(x) \geq \frac{1}{2}} = 1$ d'où

				            \begin{equation*}
					            \underbrace{(2 \eta(x)-1)}_{\geq 0} \underbrace{(\underbrace{\1{f^{\star}(x)=1}}_{1}-\underbrace{\1{g(x)=1}}_{0 \text{ ou } 1})}_{\geq 0} \geq 0
				            \end{equation*}

				      \item Si $\eta(x) < \frac{1}{2}$ :

				            Alors $\1{f^{\star}(x)=1} = \1{\eta(x) \geq \frac{1}{2}} = 0$ d'où
				            \begin{equation*}
					            \underbrace{(2 \eta(x)-1)}_{< 0} \underbrace{(\underbrace{\1{f^{\star}(x)=1}}_{0}-\underbrace{\1{g(x)=1}}_{0 \text{ ou } 1})}_{\leq 0} \geq 0
				            \end{equation*}
			      \end{itemize}

			      Pour tout $x$, on a donc $\left(2\eta(x)-1\right)\left(\1{f^{\star}(x)=1}-\1{g(x)=1}\right) \geq 0$
		      \end{reponse}

		\item Conclure.

		      \begin{reponse}
			      \begin{gather*}
				      \left(2\eta(x)-1\right)\left(\1{f^{\star}(x)=1}-\1{g(x)=1}\right) \geq 0\\
				      \overset{\ref{eq:c}}{\Leftrightarrow}\\
				      P\left(g(x)\neq Y | X=x\right)-P\left(f^{\star}(x)\neq Y|X=x\right) \geq 0\\
				      \Leftrightarrow\\
				      P\left(g(x)\neq Y | X=x\right) \geq P\left(f^{\star}(x)\neq Y|X=x\right)\\
				      \overset{\ref{eq:a}}{\Leftrightarrow}\\
				      R\left(g\right) \geq R\left(f^{\star}\right)
			      \end{gather*}
		      \end{reponse}

	\end{enumerate}
}
