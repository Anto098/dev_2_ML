\item \textbf{\enfr{Derivatives and Gradients }{Dérivés et gradients}}
\points{IFT6390 - Non demandé}{14 points}

\enfr{In this question, we will look at derivatives and gradients. Chapter 5 of Mathematics for Machine Learning book can be used as reference for this question.}{Dans cette question, nous examinerons les dérivées et les gradients. Le Chapitre 5 de ``Mathematics for Machine Learning'' peut être utilisé comme référence pour cette question.}

\begin{enumerate}
	\item{[2 points] \enfr{
		            Compute the derivative $f^{'}\left(x\right)$ for:\\
		            $$f\left(x\right)=\log\left(x^4\right)\sin\left(x^3\right)$$\\
	            }
	            {Calculer la dérivée $f^{'}\left(x\right)$ pour: \\
		            $$f\left(x\right)=\log\left(x^4\right)\sin\left(x^3\right)$$\\}
	      }
	      
	      \begin{reponse}
	      	\begin{align*}
	      		f^{'}\left(x\right)
	      		&= \log\left(x^4\right)^\prime \sin\left(x^3\right) + \sin\left(x^3\right)^\prime \log\left(x^4\right)\\
	      		&= \frac{4}{x} \sin\left(x^3\right) + 3x^2\cos\left(x^3\right)\log\left(x^4\right)
	      	\end{align*}
	      \end{reponse}

	\item { [2 points] \enfr{Compute the derivative $f^{'}\left(x\right)$ for:\\
		      $$f\left(x\right)=\exp\left(\frac{-1}{2\sigma }\left(x-\mu \right)^2\right)$$
		      Here $\sigma$,$\mu$ $\in \mathbb{R}$.\\~\\
	      }
	      { Calculer la dérivée $f^{'}\left(x\right)$ pour:\\ $$f\left(x\right)=\exp\left(\frac{-1}{2\sigma }\left(x-\mu \right)^2\right)$$
		      où $\sigma,\mu \in \mathbb{R}.$\\~\\}
	      }
	      
	      \begin{reponse}
	      	\begin{align*}
	      		f^{'}\left(x\right)
	      		&= \left(\frac{-1}{2\sigma }\left(x-\mu \right)^2\right)^\prime \exp\left(\frac{-1}{2\sigma }\left(x-\mu \right)^2\right)\\
	      		&= \frac{-1}{2\sigma } 2 \left(x -\mu \right) \exp\left(\frac{-1}{2\sigma }\left(x-\mu \right)^2\right)\\
	      		&= \frac{\mu-x}{\sigma} \exp\left(\frac{-1}{2\sigma }\left(x-\mu \right)^2\right)
	      	\end{align*}
	      \end{reponse}

	\item { [4 points] \enfr{Consider the following functions:\\
		      $$f_1\left(x\right)=\sin\left(x_1\right)\cos\left(x_2\right), \:x\:\in \:\mathbb{R}^2$$
		      $$f_2\left(x,y\right)=x^Ty\:$$
		      Here $\:x,y\:\in \:\mathbb{R}^n$.\\
		      $$f_3\left(x\right)=xx^T\:$$
		      Here $\:x\:\in \:\mathbb{R}^n$.\\~\\
		      i. What are the dimensions of $\frac{\partial f_i}{\partial x}$?\\
		      ii. Compute the jacobians.\\~\\

	      }{Considérez les fonctions suivantes:\\
		      $$f_1\left(x\right)=\sin\left(x_1\right)\cos\left(x_2\right), \:x\:\in \:\mathbb{R}^2$$
		      $$f_2\left(x, y\right)=x^{\top}y\:$$
		      Ici $\:x,y\:\in \:\mathbb{R}^n$.\\
		      $$f_3\left(x\right)=xx^\top\:$$
		      Ici $\:x\:\in \:\mathbb{R}^n$.\\~\\
		      i. Quelles sont les dimensions de $\frac{\partial f_i}{\partial x}$?\\
		      ii. Calculer le jacobien.\\~\\}
	      }

	\item { [6 points] \enfr{Compute the derivatives $\frac{df}{dx}$ of the following functions:\\
	      i. Use the chain rule. Provide the dimensions of every derivative.\\
	      $$f\left(z\right)=exp\left(-\frac{1}{2}z\right)$$
	      $$z=g\left(y\right)=y^TS^{-1}y$$
	      $$y=h\left(x\right)=x-\mu$$
	      Here $ x, \mu \in \mathbb{R}^D, S \in \mathbb{R}^{D \times D}$.\\~\\
	      ii. $$f\left(x\right)=tr\left(xx^{\top}+\sigma I\right)\:$$
	      Here $\:x\in \mathbb{R}^D$ and $tr\left(A\right)$ is the trace of A.\\~\\
	      iii. Use the chain rule to compute the derivatives and provide the dimensions of the partial derivative as well.\\
	      $$f=\tanh\left(z\right)$$
	      Here $f \in \mathbb{R}^M$.
	      $$z=Ax+b\:$$
	      Here $\:x\:\in \mathbb{R}^N,\:A\in \:\mathbb{R}^{M \times N}\:,\:b\:\in \:\mathbb{R}^M$.\\~\\

	      }
	      {Calculer les dérivées $\frac{df}{dx}$ des fonctions suivantes:\\
	      i. Utilisez la règle de la chaîne. Donnez les dimensions de chaque dérivée.\\
	      $$f\left(z\right)=\exp\left(-\frac{1}{2}z\right)$$
	      $$z=g\left(y\right)=y^{\top}S^{-1}y$$
	      $$y=h\left(x\right)=x-\mu$$
	      où $x, \mu \in \mathbb{R}^D, S \in \mathbb{R}^{D \times D}$.\\~\\
	      ii. $$f\left(x\right)=tr\left(xx^{\top}+\sigma I\right)\:$$
	      Ici $\:x\in \mathbb{R}^D$ and $tr\left(A\right)$ est la trace de A.\\~\\
	      iii. Utilisez la règle de la chaîne pour calculer les dérivées et fournir également les dimensions de la dérivée partielle.\\
	      $$f=\tanh\left(z\right)$$
	      Ici $f \in \mathbb{R}^M$.
	      $$z=Ax+b\:$$
	      Ici $\:x\:\in \mathbb{R}^N,\:A\in \:\mathbb{R}^{M \times N}\:,\:b\:\in \:\mathbb{R}^M$.\\~\\
	      }
	      }

\end{enumerate}